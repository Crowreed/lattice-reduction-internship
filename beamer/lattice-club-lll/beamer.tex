\documentclass{beamer}
\usepackage[english]{babel}
\usetheme{Madrid}
\usecolortheme{seahorse}
\usepackage[T1]{fontenc}
\usepackage{lmodern}
\usepackage[utf8]{inputenc}
\usepackage[english,ruled,vlined]{algorithm2e}

\usepackage{caption}
\usepackage{tikz}
\usetikzlibrary{decorations.pathreplacing,arrows.meta}
\usetikzlibrary{fit}
\usetikzlibrary{calc}
\usepackage{ragged2e}
\usepackage{parskip}
\setlength{\parindent}{0pt}
\setlength{\parskip}{2pt}

\newenvironment{smallalgo}[2]{
    \begin{center}
        \begin{minipage}{0.95\linewidth}
            \begin{algorithm}[H]
                 \captionsetup{labelformat=empty}
                 \caption*{\textit{#1}}
                \vspace{4pt}
                \LinesNumbered
                \DontPrintSemicolon
            }{
            \end{algorithm}
        \end{minipage}
    \end{center}
}
\usepackage[backend=biber,style=authoryear]{biblatex}
\usepackage{csquotes}
\setlength{\bibitemsep}{1\baselineskip}                                            
\addbibresource{references/references.bib}
\usepackage[most]{tcolorbox}
\tcbuselibrary{skins, breakable}
\usepackage{mdframed}
\usepackage{amsmath, amsthm, amssymb}
\usepackage{mathtools}
\usepackage{mathrsfs}
\usepackage{bm}
\hfuzz=200pt % Je t'interdis de te plaindre pour tout dépassement inférieur à 200pt (7cm)
\newcommand{\SVP}{\mathrm{SVP}}
\newcommand{\LLL}{\mathrm{LLL}}
\newcommand{\CVP}{\mathrm{CVP}}
\newcommand{\Z}{\mathbb{Z}}
\renewcommand{\u}{\mathbf{u}}
\renewcommand{\v}{\mathbf{v}}
\newcommand{\bg}{\mathbf{g}}
\renewcommand{\b}{\mathbf{b}}
\newcommand{\N}{\mathbb{N}}
\newcommand{\Q}{\mathbb{Q}}
\newcommand{\C}{\mathbb{C}}
\newcommand{\F}{\mathbb{F}}
\newcommand{\R}{\mathbb{R}}
\newcommand{\K}{\mathbb{K}}
\newcommand{\LL}{\mathscr{L}}
\newcommand{\OO}{\mathcal{O}}

\DeclareMathOperator{\rdeg}{rdeg}
\DeclareMathOperator{\MM}{MM}
\DeclareMathOperator{\M}{M}
\DeclareMathOperator{\lcoeff}{lcoeff}
\DeclareMathOperator{\rang}{rang}
\DeclareMathOperator{\size}{size}
\DeclareMathOperator{\Gram}{Gram}
\providecommand{\x}{\times}
\newcommand{\eqjust}[1]{\overset{(#1)}{=}}
\newcommand{\leqjust}[1]{\overset{(#1)}{\leq}}
\newcommand{\nint}[1]{\left\lceil #1 \right\rfloor} 
\newcommand{\ps}[2]{\left\langle #1 , #2 \right\rangle}
\newcommand{\norm}[1]{\left\| #1 \right\|}
\newcommand{\gscoeff}[2]{\frac{\ps{#1}{#2}}{\norm{#2}^2} #2}
\newtheoremstyle{definitionstyle}
{10pt}
{10pt}
{\normalfont}
{}
{\bfseries}
{.}
{1em}
{}
\theoremstyle{definitionstyle}
\newtheoremstyle{examplestyle}
{10pt}
{10pt}
{\normalfont}
{}
{\itshape}
{.}
{1em}
{}
\theoremstyle{examplestyle}
\newtheorem*{counterexample}{Contre exemple}
\newtheorem*{notation}{Notation}
\newtheorem*{remark}{Remarque}
\newtheoremstyle{propositionstyle}
{10pt}
{10pt}
{\normalfont}
{}
{\bfseries}
{.}
{1em}
{}
\usepackage[absolute,overlay]{textpos}
\renewcommand*{\mkbibnamefamily}[1]{#1}
\renewcommand*{\mkbibnamegiven}[1]{#1}
% Force le retour à la ligne pour les liens et DOIs
\setcounter{biburllcpenalty}{7000}
\setcounter{biburlucpenalty}{8000}
\setcounter{biburlnumpenalty}{9000}

\title{The LLL Algorithm: Lattice Basis Reduction and applications to Approximate Shortest Vector Problem}
\author{Lucas Petit}
\date{May 26, 2025}

\begin{document}

%%%%%%%%%%%%%%%%%%%%%%%%%%%%%%%%%%%%%%%%%%%%%%%%%%%%%%%%%%%%%%%%%%%%%%%%%%%%%%%%%%%%%%%%%%%%%%%%%%%%%%%%%%%%%%%%%%%%%%%%%%%%%%%%%%%%%%%%%%%%%%%%%%%%%%%%%%%%%%%

\begin{frame}
    \titlepage
\end{frame}

%%%%%%%%%%%%%%%%%%%%%%%%%%%%%%%%%%%%%%%%%%%%%%%%%%%%%%%%%%%%%%%%%%%%%%%%%%%%%%%%%%%%%%%%%%%%%%%%%%%%%%%%%%%%%%%%%%%%%%%%%%%%%%%%%%%%%%%%%%%%%%%%%%%%%%%%%%%%%%%

\begin{frame}{Recap: Euclidean Space and Inner Product}
    \uncover<2->
    {
        We consider a real finite-dimensional vector space $\R^n$ equipped with the standard \textbf{Euclidean inner product}:
        \[
            \ps{\u}{\v} := \sum_{i=1}^{n} \u_i \v_i
        \]
    }

    \uncover<3->
    {
        This inner product induces the \textbf{Euclidean norm}:
        \[
            \|\u\|_2 = \sqrt{\ps{\u}{\u}} = \sqrt{\sum_{i=1}^{n} \u_i^2}
        \]
    }
\end{frame}

%%%%%%%%%%%%%%%%%%%%%%%%%%%%%%%%%%%%%%%%%%%%%%%%%%%%%%%%%%%%%%%%%%%%%%%%%%%%%%%%%%%%%%%%%%%%%%%%%%%%%%%%%%%%%%%%%%%%%%%%%%%%%%%%%%%%%%%%%%%%%%%%%%%%%%%%%%%%%%%

\begin{frame}{Recap: Euclidean Lattice}
    \uncover<2->
    {
        A \textbf{Euclidean lattice} $\LL$ is a discrete additive subgroup of $\R^n$.
    }

    \begin{itemize}
        \uncover<3->
        {
        \item[$\bullet$] \textbf{Additive subgroup:}
              \[
                  \mathbf{0} \in \LL\text{, } \mathbf{x} + \mathbf{y} \in \LL \text{, } -\mathbf{x} \in \LL \text{ for all } \mathbf{x}, \mathbf{y} \in \LL.
              \]
              }
              \uncover<4->
              {
              \vspace{-0.3cm}
        \item[$\bullet$]  \textbf{Discrete:} For every $\mathbf{x} \in \LL$, there exists $\varepsilon > 0$ such that
              \[
                  \mathcal{B}(\mathbf{x}, \varepsilon) \cap \LL = \{\mathbf{x}\}
              \]
              where $\mathcal{B}(\mathbf{x}, \varepsilon)$ denotes the open ball of radius $\varepsilon$ centered at $\mathbf{x}$.
              }
    \end{itemize}

    \uncover<5->
    {
        \begin{figure}
            \visible<6->
            {
                \includegraphics[width=0.25\textwidth]{images/lattice_5.png}
            }
            \visible<7->
            {
                \includegraphics[width=0.25\textwidth]{images/lattice_6.png}
            }
            \visible<8->
            {
                \includegraphics[width=0.25\textwidth]{images/lattice_7.png}
            }

            \caption{Example of lattice in $\R^2$}
        \end{figure}
    }
\end{frame}

%%%%%%%%%%%%%%%%%%%%%%%%%%%%%%%%%%%%%%%%%%%%%%%%%%%%%%%%%%%%%%%%%%%%%%%%%%%%%%%%%%%%%%%%%%%%%%%%%%%%%%%%%%%%%%%%%%%%%%%%%%%%%%%%%%%%%%%%%%%%%%%%%%%%%%%

\begin{frame}{Recap: Lattice Bases}

    \uncover<2->
    {
        Any lattice $\LL \subseteq \R^n$ admits a maximal $\Z$-linearly independent family $(\b_i)_{1 \leq i \leq m}$, with \( m \leq n \) such that:
        \[
            \LL = \bigoplus_{i=1}^{m} \Z \b_i = \left\{ a_1 \b_1 + \cdots + a_m \b_m \mid a_i \in \Z \right\}
        \]
    }

    \uncover<3->
    {
        This family is called a \textbf{basis} of the lattice $\LL$.
    }

    \uncover<4->
    {
        \begin{figure}
            \visible<5->
            {
                \includegraphics[width=0.25\textwidth]{images/lattice_1.png}
            }
            \visible<6->
            {
                \includegraphics[width=0.25\textwidth]{images/lattice_2.png}
            }
            \visible<7->
            {
                \includegraphics[width=0.25\textwidth]{images/lattice_4.png}
            }

            \caption{Example of lattice with different basis in $\R^2$}
        \end{figure}
    }
\end{frame}

%%%%%%%%%%%%%%%%%%%%%%%%%%%%%%%%%%%%%%%%%%%%%%%%%%%%%%%%%%%%%%%%%%%%%%%%%%%%%%%%%%%%%%%%%%%%%%%%%%%%%%%%%%%%%%%%%%%%%%%%%%%%%%%%%%%%%%%%%%%%%%%%%%%%%%%%%%%%%%%

\begin{frame}{Two different bases of the same lattice}

    \uncover<2->
    {
        \begin{figure}
            \centering
            \begin{minipage}{0.4\textwidth}
                \centering
                \includegraphics[width=\linewidth]{images/lattice_3.png}
                \caption*{\shortstack{short, nearly orthogonal vectors\\\textbf{looks good}}}
            \end{minipage}
            \hfill
            \begin{minipage}{0.4\textwidth}
                \centering
                \includegraphics[width=\linewidth]{images/lattice_4.png}
                \caption*{\shortstack{long, skewed basis vectors\\\textbf{looks bad}}}
            \end{minipage}
        \end{figure}
    }

    \uncover<3->
    {
        Can we formalize this?
    }

    \uncover<4->
    {
        $\rightarrow$ notion of \textbf{quasi-orthogonal} (or \textbf{reduced}) bases.
    }
\end{frame}

%%%%%%%%%%%%%%%%%%%%%%%%%%%%%%%%%%%%%%%%%%%%%%%%%%%%%%%%%%%%%%%%%%%%%%%%%%%%%%%%%%%%%%%%%%%%%%%%%%%%%%%%%%%%%%%%%%%%%%%%%%%%%%%%%%%%%%%%%%%%%%%%%%%%%%%%%%%%%%%

\begin{frame}{Recap: Orthogonal Bases and Gram-Schmidt Process}

    \uncover<2->
    {
        A basis \( (\b_i)_{1 \leq i \leq n} \) of \( \R^n \) is called \textbf{orthogonal} if
        \[
            \langle \b_i, \b_j \rangle = 0 \quad \text{for all } i \neq j.
        \]
    }
    \uncover<3->
    {
        \begin{figure}
            \visible<4->
            {
                \includegraphics[width=0.3\textwidth]{images/lattice_8.png}
            }
            \visible<5->
            {
                \includegraphics[width=0.3\textwidth]{images/lattice_2.png}
            }
            \visible<6->
            {
                \includegraphics[width=0.3\textwidth]{images/lattice_3.png}
            }

            \caption{Orthogonal or not orthogonal basis}
        \end{figure}
    }

    \uncover<7->
    {
        How an we compute an orthogonal basis ?
    }
    \uncover<8->
    {

        $\rightarrow$ \textbf{Gram-Schmidt orthogonalization process}
    }
\end{frame}

%%%%%%%%%%%%%%%%%%%%%%%%%%%%%%%%%%%%%%%%%%%%%%%%%%%%%%%%%%%%%%%%%%%%%%%%%%%%%%%%%%%%%%%%%%%%%%%%%%%%%%%%%%%%%%%%%%%%%%%%%%%%%%%%%%%%%%%%%%%%%%%%%%%%%%%%%%%%%%%

\begin{frame}{Recap: Gram--Schmidt orthogonalization}

    \uncover<2->
    {
        Let \( (\b_i)_{1 \leq i \leq n} \) be a basis of \( \R^n \). The associated orthogonal basis \( (\b_i^*)_{1 \leq i \leq n} \) is constructed via the \textbf{Gram–Schmidt orthogonalization process}:
        \[
            \b_1^* \coloneqq \b_1,
            \quad
            \b_i^* \coloneqq \b_i - \sum_{j=1}^{i-1} \mu_{i,j} \b_j^*,
            \quad
            \mu_{i,j} \coloneqq \frac{\langle \b_i, \b_j^* \rangle}{\|\b_j^*\|^2}.
        \]
    }

    \uncover<3>
    {
        \centering
        \includegraphics[width=0.4\textwidth]{images/slide07.png}

    }
\end{frame}

%%%%%%%%%%%%%%%%%%%%%%%%%%%%%%%%%%%%%%%%%%%%%%%%%%%%%%%%%%%%%%%%%%%%%%%%%%%%%%%%%%%%%%%%%%%%%%%%%%%%%%%%%%%%%%%%%%%%%%%%%%%%%%%%%%%%%%%%%%%%%%%%%%%%%%%%%%%%%%%

\begin{frame}{Recap: Gram--Schmidt orthogonalization}

    \uncover<2->
    {
        The coefficients \( \mu_{i,j} \) are called \textbf{Gram–Schmidt coefficients}.

        \[
            \begin{pmatrix}
                \b_1   \\
                \b_2   \\
                \vdots \\
                \b_n
            \end{pmatrix}
            =
            \begin{pmatrix}
                1         & 0      & \cdots      & 0      \\
                \mu_{2,1} & \ddots & \ddots      & \vdots \\
                \vdots    & \ddots & \ddots      & 0      \\
                \mu_{n,1} & \cdots & \mu_{n,n-1} & 1
            \end{pmatrix}
            \times
            \begin{pmatrix}
                \b_1^* \\
                \b_2^* \\
                \vdots \\
                \b_n^*
            \end{pmatrix}
        \]


    }

    \uncover<3>
    {
        The resulting family \( (\b_i^*)_{1 \leq i \leq n} \) is orthogonal.
    }
\end{frame}

%%%%%%%%%%%%%%%%%%%%%%%%%%%%%%%%%%%%%%%%%%%%%%%%%%%%%%%%%%%%%%%%%%%%%%%%%%%%%%%%%%%%%%%%%%%%%%%%%%%%%%%%%%%%%%%%%%%%%%%%%%%%%%%%%%%%%%%%%%%%%%%%%%%%%%%%%%%%%%%

\begin{frame}{Example: Gram--Schmidt Orthogonalization}
    \pause Let
    \[
        B =
        \begin{pmatrix}
            -2 & 2 & 1 \\
            3  & 0 & 2 \\
            2  & 2 & 0
        \end{pmatrix}
    \]

    \vspace{1em}

    \pause \textbf{Step 1 : } \(\b_1^* \pause := \pause \b_1 \pause := \pause (-2, 2, 1)\), \(\pause \| \b_1^* \|^2 \pause = \pause 2^2+2^2+1 \pause = \pause 9\)

    \pause \textbf{Step 2 : } \( \mu_{2, 1} \pause = \frac{\ps{\b_2}{\b_1^*}}{\|\b_1^*\|^2} \pause = -\frac{4}{9} \pause \)

    \( \b_2^* \pause := \pause \b_2 - \mu_{2, 1} \b_1^* \pause = (3, 0, 2) + \frac{4}{9} (-2, 2, 1) \pause = \left(\frac{19}{9}, \frac{8}{9}, \frac{22}{9}\right) \)

    \pause \textbf{Step 3 :} \( \mu_{3, 1} = 0\) , \( \mu_{3, 2} = \frac{54}{101}, \pause \b_3^* = \left(\frac{88}{101}, \frac{154}{101}, -\frac{132}{101}\right)\)

    \pause
    \[
        \overbrace{
            \begin{pmatrix}
                -2 & 2 & 1 \\
                3  & 0 & 2 \\
                2  & 2 & 0
            \end{pmatrix}
        }^{B}
        = \pause
        \overbrace{
            \begin{pmatrix}
                1            & 0              & 0 \\
                -\frac{4}{9} & 1              & 0 \\
                0            & \frac{54}{101} & 1
            \end{pmatrix}
        }^{U}
        \x
        \overbrace{
            \begin{pmatrix}
                -2             & 2               & 1                \\
                \frac{19}{9}   & \frac{8}{9}     & \frac{22}{9}     \\
                \frac{88}{101} & \frac{154}{101} & -\frac{132}{101}
            \end{pmatrix}
        }^{B^*}
    \]
\end{frame}








%%%%%%%%%%%%%%%%%%%%%%%%%%%%%%%%%%%%%%%%%%%%%%%%%%%%%%%%%%%%%%%%%%%%%%%%%%%%%%%%%%%%%%%%%%%%%%%%%%%%%%%%%%%%%%%%%%%%%%%%%%%%%%%%%%%%%%%%%%%%%%%%%%%%%%%%%%%%%%%

\begin{frame}{Size Reduction of a Basis}
    \begin{textblock*}{12cm}(0.5cm, 1.2cm)
        \uncover<2->
        {
            \textbf{Problem:} The Gram–Schmidt orthogonal basis of \( B \) is generally not a basis of the lattice \( \LL(B) \).
        }
    \end{textblock*}

    \begin{textblock*}{12cm}(3.5cm, 2.7cm)
        \only<3>{\includegraphics[width=0.5\textwidth]{images/lattice_00.png}}
        \only<4>{\includegraphics[width=0.5\textwidth]{images/lattice_01.png}}
    \end{textblock*}
\end{frame}

%%%%%%%%%%%%%%%%%%%%%%%%%%%%%%%%%%%%%%%%%%%%%%%%%%%%%%%%%%%%%%%%%%%%%%%%%%%%%%%%%%%%%%%%%%%%%%%%%%%%%%%%%%%%%%%%%%%%%%%%%%%%%%%%%%%%%%%%%%%%%%%%%%%%%%%%%%%%%%%

\begin{frame}{Size Reduction of a Basis}
    \begin{textblock*}{12cm}(0.5cm, 1.2cm)
        \uncover<2->
        {
            We want a basis of $\LL$ that \emph{approximates} the Gram–Schmidt basis as closely as possible:
        }
    \end{textblock*}

    \begin{textblock*}{12cm}(4cm, 1.9cm)
        \only<3>{\includegraphics[width=0.35\textwidth]{images/slide11_0.png}}
        \only<4>{\includegraphics[width=0.35\textwidth]{images/slide11_1.png}}
        \only<5>{\includegraphics[width=0.35\textwidth]{images/slide11_2.png}}
        \only<6>{\includegraphics[width=0.35\textwidth]{images/slide11_3.png}}
        \only<7>{\includegraphics[width=0.35\textwidth]{images/slide11_4.png}}
        \only<8>{\includegraphics[width=0.35\textwidth]{images/slide11_4.png}}
        \only<9->{\includegraphics[width=0.35\textwidth]{images/slide11_5.png}}
    \end{textblock*}

    \begin{textblock*}{12cm}(0.5cm,6cm)
        \uncover<8->
        {
            We define the \textbf{nearest integer}, as
            \(
            \nint{x} \coloneqq \left\lfloor x + \tfrac{1}{2} \right\rfloor.
            \)
        }
    \end{textblock*}

    \begin{textblock*}{6cm}(8.3cm,3.5cm)

        \uncover<10->
        {
            \( \left|\nint{x}-x\right| \leq \frac{1}{2} \text{ for all } x \in \R \)
        }
    \end{textblock*}

    \begin{textblock*}{12cm}(0.5cm,7.3cm)
        \uncover<11->
        {
            \textbf{Definition:}
            A basis is said to be \textbf{size-reduced} if:
            \[
                \max_{1 \leq j < i \leq n} |\mu_{i,j}| \leq \frac{1}{2}
            \]
        }


    \end{textblock*}


\end{frame}

%%%%%%%%%%%%%%%%%%%%%%%%%%%%%%%%%%%%%%%%%%%%%%%%%%%%%%%%%%%%%%%%%%%%%%%%%%%%%%%%%%%%%%%%%%%%%%%%%%%%%%%%%%%%%%%%%%%%%%%%%%%%%%%%%%%%%%%%%%%%%%%%%%%%%%%%%%%%%%%

\begin{frame}{Why Size Reduction is Not Enough}


    \begin{figure}
        \centering
        \includegraphics[width=0.4\textwidth]{images/lattice_not_sufficient.png}
        \caption*{\small \textbf{A size-reduced basis.}}
    \end{figure}
    \pause
    \[
        \overbrace{
            \begin{pmatrix}
                3  & 4  \\
                -2 & -1
            \end{pmatrix}
        }^{B}
        =
        \overbrace{
            \begin{pmatrix}
                1            & 0 \\
                -\frac{2}{5} & 1
            \end{pmatrix}
        }^{U}
        \cdot
        \overbrace{
            \begin{pmatrix}
                3            & 4           \\
                -\frac{4}{5} & \frac{3}{5}
            \end{pmatrix}
        }^{B^*}
    \]
    \pause \textit{Length reduction alone \textbf{does not imply} almost-orthogonality!}
\end{frame}

%%%%%%%%%%%%%%%%%%%%%%%%%%%%%%%%%%%%%%%%%%%%%%%%%%%%%%%%%%%%%%%%%%%%%%%%%%%%%%%%%%%%%%%%%%%%%%%%%%%%%%%%%%%%%%%%%%%%%%%%%%%%%%%%%%%%%%%%%%%%%%%%%%%%%%%%%%%%%%%

\begin{frame}{Definition: Lovász condition}
    \begin{textblock*}{12cm}(0.5cm,1.5cm)
        \uncover<2->
        {
            Ideally, we would like to find a basis \( (\b_i)_{1 \leq i \leq n} \) of the lattice \( \LL \) such that:
            \[
                \| \b_1 \| = \lambda_1(\LL), \quad
                \| \b_2 \| = \lambda_2(\LL), \quad \ldots, \quad
                \| \b_n \| = \lambda_n(\LL)
            \]
        }
    \end{textblock*}

    \begin{textblock*}{12cm}(0.5cm,3.3cm)
        \uncover<3->
        {
            This would imply \( \| \b_1 \| \leq \cdots \leq \| \b_n \|\), but is it hard to find a such basis.
        }
    \end{textblock*}

    \begin{textblock*}{12cm}(0.5cm,4cm)
        \uncover<4->
        {
            A basis \( (\b_i)_{1 \leq i \leq m} \) satisfies the \textbf{original Lovász condition} if:
            \[
                \|\b_i^*\|^2 \leq 2\|\b_{i+1}^*\|^2 \quad \text{for all } 1 \leq i < n
            \]
        }
    \end{textblock*}

    \begin{textblock*}{\paperwidth}(0cm,5.5cm)
        \centering
        \only<5-10>{\includegraphics[width=0.28\textwidth]{images/lattice_not_sufficient.png}}
        \only<11>{\includegraphics[width=0.28\textwidth]{images/lattice_LLLreduced.png}}
    \end{textblock*}

    \begin{textblock*}{2cm}(5.1cm,7.5cm)
        \only<5-7> {\( \b_2 \)}
        \only<8-11> {\( \b_1 \)}
    \end{textblock*}

    \begin{textblock*}{2cm}(5.7cm,6.1cm)
        \only<11> {\( \b_2 \)}
    \end{textblock*}

    \begin{textblock*}{2cm}(7.2cm,5.5cm)
        \only<5-7> {\( \b_1 \)}
        \only<8-10> {\( \b_2 \)}
    \end{textblock*}

    \begin{textblock*}{\paperwidth}(0cm,8cm)
        \centering
        \only<6-7>{\textcolor{red}{\( \|\b_1^*\|^2 > 2\|\b_{2}^*\|^2 \)}}

        \only<7>{We can swap \( \b_1 \) and \( \b_2 \).}
    \end{textblock*}

    \begin{textblock*}{\paperwidth}(0cm,8cm)
        \centering
        \only<9-10>{\textcolor{green}{\( \|\b_1^*\|^2 \leq 2\|\b_{2}^*\|^2 \)}}

        \only<10>{We can size-reduce \( \b_1 \) and \( \b_2 \)!}
    \end{textblock*}

    % poruquoi opas faire une transition de lovasz à la formule plus simple et de dire que c'est equivalent pour notre propos

    %+ EXEMPLE ET CONTRE EXEMPLE

\end{frame}


%\begin{frame}
%    Si on veut
%    \[
%    \|b_i\|^2 \leq \|b_{i+1}\|^2
%    \]
%    En réecrivant dans la base de Gram--Schmidt :
%    \[
%    \|b_i^*\|^2 + \sum_{j < i} \mu_{i,j}^2 \|b_j^*\|^2 < \|b_{i+1}^*\|^2 + \mu_{i+1,i}^2 \|b_i^*\|^2 + \sum_{j < i} \mu_{i+1,j}^2 \|b_j^*\|^2
%    \]
%    
%    En ignorant ces deux sommes, on retrouve la condition de Lovasz pour \( \delta = 1\): 
%    
%    \[
%    \|b_i^*\|^2  < \|b_{i+1}^*\|^2 + \mu_{i+1,i}^2 \|b_i^*\|^2 
%    \]
%\end{frame}
%%%%%%%%%%%%%%%%%%%%%%%%%%%%%%%%%%%%%%%%%%%%%%%%%%%%%%%%%%%%%%%%%%%%%%%%%%%%%%%%%%%%%%%%%%%%%%%%%%%%%%%%%%%%%%%%%%%%%%%%%%%%%%%%%%%%%%%%%%%%%%%%%%%%%%%%%%%%%%%

\begin{frame}{Definition: \( \LLL \)-- reduced Basis}

    A basis is called \( \LLL \)--reduced if:
    \begin{itemize}
        \item[$\bullet$] It is size-reduced;
        \item[$\bullet$] It satisfies the Lovász condition.
    \end{itemize}

    % EXEMPLE DE DIFFERENTES BASES REDUITE OU NON AVEC INTERET CONDITION DE LOVASZ
\end{frame}

%%%%%%%%%%%%%%%%%%%%%%%%%%%%%%%%%%%%%%%%%%%%%%%%%%%%%%%%%%%%%%%%%%%%%%%%%%%%%%%%%%%%%%%%%%%%%%%%%%%%%%%%%%%%%%%%%%%%%%%%%%%%%%%%%%%%%%%%%%%%%%%%%%%%%%%%%%%%%%%

\begin{frame}{Recap:The $\gamma-\SVP$ Problem}

    Definitions of \( \lambda_1 \), \( \lambda_2 \), \( \ldots \) are detailed in \parencite{Boudgoust2023}.

    \vspace{1cm}
    \pause \textbf{Approximate Shortest Vector Problem (\( \gamma-\SVP \))} \\
    \vspace{0.5em}

    Given a basis \( B \) of a lattice \( \LL \subset \R^n \) and an approximation factor \( \gamma > 0 \), \textbf{find a non-zero vector} \( \v \in \LL \setminus \{\mathbf{0}\} \) such that:
    \[
        \|\v\|_2 \leq \gamma \cdot \lambda_1(\LL)
    \]

    \vspace{1em}

    \pause
    \begin{tabular}{@{}ll}
        \(\gamma = 1\)                & exact \( \SVP \) — NP-hard                          \\
        \(\gamma = \mathrm{poly}(n)\) & relevant for \textbf{lattice-based cryptography}    \\
        \(\gamma = 2^{\OO(n)}\)       & solvable in \textbf{polynomial time} via \( \LLL \) \\
    \end{tabular}
\end{frame}

%%%%%%%%%%%%%%%%%%%%%%%%%%%%%%%%%%%%%%%%%%%%%%%%%%%%%%%%%%%%%%%%%%%%%%%%%%%%%%%%%%%%%%%%%%%%%%%%%%%%%%%%%%%%%%%%%%%%%%%%%%%%%%%%%%%%%%%%%%%%%%%%%%%%%%%%%%%%%%%

\begin{frame}{Lemma}
    \begin{textblock*}{12cm}(0.5cm, 1.2cm) % {largeur} (x,y)
        \uncover<2->
        {
            \textbf{Lemma.} For any \( \b \in \LL \setminus \{\mathrm{0}\} \) we have:
            \vspace{-0.1cm}
            \[
                \displaystyle \|\b\| \geq \min_{1 \leq i \leq n} \|\b_i\|
            \]
        }
    \end{textblock*}

    \begin{textblock*}{12cm}(0.5cm,2.8cm)
        \uncover<3->
        {
            \textbf{Proof.} Let \( (\b_i)_{1 \leq i \leq n} \) of the lattice \( \LL \), and write:
            \vspace{-0.1cm}
            \[
                \b = \sum_{i=1}^n \lambda_i \b_i \in \LL \setminus \{0\}, \quad \lambda_i \in \Z.
            \]
            Let \( k \) be the largest index such that \( \lambda_k \neq 0 \).
            We can write
        }
    \end{textblock*}

    \begin{textblock*}{12cm}(0.5cm,5.5cm)
        \only<4>
        {
            \[
                \b = \sum_{i=1}^{n} \lambda_i \b_i
            \]
        }
    \end{textblock*}

    \begin{textblock*}{12cm}(0.5cm,5.5cm)
        \only<5>
        {
            \[
                \b = \sum_{i=1}^{k} \lambda_i \b_i
            \]
        }
    \end{textblock*}

    \begin{textblock*}{12cm}(0.5cm,5.5cm)
        \only<6>
        {
            \[
                \b = \sum_{i=1}^{k} \lambda_i \left( \b_i^* + \sum_{j=1}^{i} \mu_{ij} \b_j^* \right)
            \]
        }
    \end{textblock*}

    \begin{textblock*}{12cm}(0.5cm,5.5cm)
        \only<7>
        {
            \[
                \b = \sum_{i=1}^{k} \lambda_i \b_i^* + \lambda_i \sum_{j=1}^{i} \mu_{ij} \b_j^*
            \]
        }
    \end{textblock*}

    \begin{textblock*}{12cm}(0.5cm,5.5cm)
        \only<8>
        {
            \[
                \b = \lambda_k \b_k^* + \sum_{i < k} \lambda_i \sum_{j=1}^{i} \mu_{ij} \b_j^*
            \]
        }
    \end{textblock*}

    \begin{textblock*}{12cm}(0.5cm,5.5cm)
        \only<9->
        {
            \[
                \b = \lambda_k \mathbf{b}_k^* + \sum_{i < k} \nu_i \b_i^*, \quad \nu_i \in \R
            \]
        }
    \end{textblock*}

    \begin{textblock*}{12cm}(0.5cm,6.5cm)
        \only<10->
        {
            Hence
            \[
                \|\b\|^2 = \lambda_k^2 \|\b_k^*\|^2 + \sum_{i<k} \nu_i^2 \|\b_i^*\|^2
            \]

            \[
                \displaystyle
                \only<9->{\geq \lambda_k^2 \|\b_k^*\|^2}
                \only<10->{\geq \|\b_k^*\|^2}
                \only<11->{\geq \min_{1 \leq i \leq n} \|\b_i\|}
            \]
        }
    \end{textblock*}



\end{frame}


%%%%%%%%%%%%%%%%%%%%%%%%%%%%%%%%%%%%%%%%%%%%%%%%%%%%%%%%%%%%%%%%%%%%%%%%%%%%%%%%%%%%%%%%%%%%%%%%%%%%%%%%%%%

\begin{frame}{Theorem: Bound on First Vector in a Reduced Basis}

    \begin{textblock*}{12cm}(0.5cm, 1.2cm)
        \uncover<2->
        {
            \textbf{Theorem.} Let \( (\b_i)_{1 \leq i \leq n} \) be a reduced basis of a lattice \( \LL \subseteq \R^n \), and let \( \b \in \LL \setminus \{0\} \). Then:
            \[
                \|\b_1\| \leq 2^{(n-1)/2} \cdot \|\b\|.
            \]
        }
    \end{textblock*}

    \begin{textblock*}{12cm}(0.5cm,2.5cm)
        \uncover<3>
        {
            \textbf{Proof.}
            \[
                \|\b_1\|^2  =  \|\b_1^*\|^2\leq 2 \|\b_2^*\|^2  \leq 2^2 \|\b_3^*\|^2  \leq \cdots \leq  2^{n-1} \|\b_n^*\|^2.
            \]
            Thus,
            \[
                \|\b\|  \geq \min\{\|\b_1^*\|, \dots, \|\b_n^*\|\}  \geq 2^{-(n-1)/2} \|\b_1\|
            \]
            \qed
        }
    \end{textblock*}

    \begin{textblock*}{12cm}(0.5cm,2.5cm)

        \uncover<5->
        {
            \textbf{Corollary.}
            \[
                \|\b_1\| \leq 2^{(n-1)/2} \cdot \lambda_1(\LL).
            \]
        }

        \uncover<6->
        {
            \textbf{Interpretation.} The vector \( \b_1 \) of a reduced basis \textbf{solves} \(  2^{(n-1)/2} - \SVP\).
        }

        \uncover<7->
        {
            How can we compute a reduced basis in practice?
        }
        \uncover<8>
        {

            $\rightarrow$ Use the \( \LLL \) \parencite{Lenstra1982}(Lenstra, Lenstra, Lovasz) algorithm!
        }
    \end{textblock*}
\end{frame}

%%%%%%%%%%%%%%%%%%%%%%%%%%%%%%%%%%%%%%%%%%%%%%%%%%%%%%%%%%%%%%%%%%%%%%%%%%%%%%%%%%%%%%%%%%%%%%%%%%%%%%%%%%%%%%%%%%%%%%%%%%%%%%%%%%%%%%%%%%%%%%%%%%%%%%%%%%%%%%%

\begin{frame}{LLL Algorithm}

    \uncover<2->
    {
        \begin{smallalgo}{LLL}{algo:LLL_MCA}
            \LinesNumbered
            \DontPrintSemicolon
            \KwIn{A basis \( B = (\b_1, \ldots, \b_n) \)}
            \KwOut{An LLL-reduced basis \( G = (\bg_1, \ldots, \bg_n) \)}

            \( G \leftarrow copy(B)\)


            \tikz[remember picture, baseline=(gs.base)]{
                \node[inner sep=1pt] (gs) {
                    \( (G^*, U) \leftarrow \) Gram-Schmidt \( G \)\;
                };
            }


            \While{\( i \leq n \)}{

                \For{\( j = i-1, i-2, \ldots, 1 \)}{
                    \tikz[remember picture, baseline=(size.base)]{
                        \node[inner sep=1pt] (size) {
                            \( \bg_i \leftarrow \bg_i - \nint{\mu_{i,j}} \, \bg_j \), update \( (G^*, U) \)\;
                        };
                    }
                }

                \If{
                    \( i > 1 \) \textbf{and}
                    \tikz[remember picture, baseline=(lov.base)]{
                        \node[inner sep=1pt] (lov) {
                            \( \|\bg_{i-1}^*\|^2 > 2 \|\bg_i^*\|^2 \)
                        };
                    }
                }{
                    Swap \( \bg_{i-1} \) and \( \bg_i \), update \((G^*, U)\)\;
                    \( i \leftarrow i - 1 \)\;
                }
                \Else{
                    \( i \leftarrow i + 1 \)\;
                }
            }

            \KwRet{\( G \)}
        \end{smallalgo}
    }



    % Encadrés en overlay
    \begin{tikzpicture}[remember picture, overlay]

        % Encadré vert gram-schmidt
        \visible<3->{
            \node[draw=green, thick, rounded corners, fit=(gs), inner sep=0pt] {};
            \node[anchor=west] at ([xshift=0.3cm]gs.east) {\textbf{\textcolor{green}{Gram-Schmidt}}};
        }

        % Encadré bleu size-reduction
        \visible<4->{
            \node[draw=blue, thick, rounded corners, fit=(size), inner sep=0pt] {};
            \node[anchor=west] at ([xshift=0.3cm]size.east) {\textbf{\textcolor{blue}{Size Reduction}}};
        }

        % Encadré rouge Lovász
        \visible<5->{
            \node[draw=red, thick, rounded corners, fit=(lov), inner sep=0pt] {};
            \node[anchor=west] at ([xshift=1.3cm]lov.east) {\textbf{\textcolor{red}{Lovász Condition}}};
        }

    \end{tikzpicture}

\end{frame}


%%%%%%%%%%%%%%%%%%%%%%%%%%%%%%%%%%%%%%%%%%%%%%%%%%%%%%%%%%%%%%%%%%%%%%%%%%%%%%%%%%%%%%%%%%%%%%%%%%%%%%%%%%%%%%%%%%%%%%%%%%%%%%%%%%%%%%%%%%%%%%%%%%%%%%%%%%%%%%%


\begin{frame}{Example}
    \pause Let's compute a LLL reduced basis of \( \LL (B) \) with
    \[
        B \coloneqq
        \begin{pmatrix}
            -2 & 2 & 1 \\
            3  & 0 & 2 \\
            2  & 2 & 0
        \end{pmatrix}
    \]

    \pause We start by compute its Gram-Schmidt decomposition :

    \pause We did it previously!

    \pause
    \[
        \overbrace{
            \begin{pmatrix}
                -2 & 2 & 1 \\
                3  & 0 & 2 \\
                2  & 2 & 0
            \end{pmatrix}
        }^{B}
        =
        \overbrace{
            \begin{pmatrix}
                1            & 0              & 0 \\
                -\frac{4}{9} & 1              & 0 \\
                0            & \frac{54}{101} & 1
            \end{pmatrix}
        }^{U}
        \x
        \overbrace{
            \begin{pmatrix}
                -2             & 2               & 1                \\
                \frac{19}{9}   & \frac{8}{9}     & \frac{22}{9}     \\
                \frac{88}{101} & \frac{154}{101} & -\frac{132}{101}
            \end{pmatrix}
        }^{B^*}
    \]
\end{frame}

%%%%%%%%%%%%%%%%%%%%%%%%%%%%%%%%%%%%%%%%%%%%%%%%%%%%%%%%%%%%%%%%%%%%%%%%%%%%%%%%%%%%%%%%%%%%%%%%%%%%%%%%%%%%%%%%%%%%%%%%%%%%%%%%%%%%%%%%%%%%%%%%%%%%%%%%%%%%%%%

\begin{frame}{Example}
    \uncover<2->
    {
        \[
            \begin{tikzpicture}[remember picture,baseline=(E1.base)]
                \node (E1)
                {
                    \(
                    \underbrace
                    {
                        \begin{pmatrix}
                            -2 & 2 & 1 \\
                            3  & 0 & 2 \\
                            2  & 2 & 0
                        \end{pmatrix}
                    }_{G}
                    =
                    \underbrace
                    {
                        \begin{pmatrix}
                            1            & 0              & 0 \\
                            -\frac{4}{9} & 1              & 0 \\
                            0            & \frac{54}{101} & 1
                        \end{pmatrix}
                    }_{U}
                    \cdot
                    \underbrace
                    {
                        \begin{pmatrix}
                            -2             & 2               & 1                \\
                            \frac{19}{9}   & \frac{8}{9}     & \frac{22}{9}     \\
                            \frac{88}{101} & \frac{154}{101} & -\frac{132}{101}
                        \end{pmatrix}
                    }_{G^*}
                    \)
                };
            \end{tikzpicture}
        \]
    }

    \vspace{2cm}

    \uncover<5->
    {
        \[
            \begin{tikzpicture}[remember picture,baseline=(E2.base)]
                \node (E2)
                {
                    \(
                    \underbrace
                    {
                        \begin{pmatrix}
                            -2 & 2 & 1  \\
                            3  & 0 & 2  \\
                            -1 & 2 & -2
                        \end{pmatrix}
                    }_{G}
                    =
                    \underbrace
                    {
                        \begin{pmatrix}
                            1            & 0               & 0 \\
                            -\frac{4}{9} & 1               & 0 \\
                            0            & -\frac{47}{101} & 1
                        \end{pmatrix}
                    }_{U}
                    \cdot
                    \underbrace
                    {
                        \begin{pmatrix}
                            -2             & 2               & 1                \\
                            \frac{19}{9}   & \frac{8}{9}     & \frac{22}{9}     \\
                            \frac{88}{101} & \frac{154}{101} & -\frac{132}{101}
                        \end{pmatrix}
                    }_{G^*}
                    \)
                };
            \end{tikzpicture}
        \]
    }

    \only<3>
    {
        \begin{tikzpicture}[remember picture,overlay]
            \draw[->,
                thick,
                shorten >=2pt,
                shorten <=2pt,
                >=Latex
            ]
            (E1.south) to
            node[midway, left] {\textbf{\textcolor{blue}{Size Reduction}}}
            node[midway, right] {$\bg_3 \leftarrow \bg_3-\nint{\frac{54}{101}} \cdot \bg_2$}
            (E2.north);
        \end{tikzpicture}
    }

    \only<4->
    {
        \begin{tikzpicture}[remember picture,overlay]
            \draw[->,
                thick,
                % change l’angle de courbure
                shorten >=2pt,           % décale la pointe
                shorten <=2pt,           % décale la base
                >=Latex                  % style de la pointe
            ]
            (E1.south) to
            node[midway, left] {\textbf{\textcolor{blue}{Size Reduction}}}
            node[midway, right] {$\bg_3 \leftarrow \bg_3-1 \cdot \bg_2$}
            (E2.north);
        \end{tikzpicture}
    }


\end{frame}

\begin{frame}{Example}

    \uncover<3->
    {
        \[
            \begin{tikzpicture}[remember picture,baseline=(E2.base)]
                \node (E2)
                {
                    \(
                    \underbrace{
                        \begin{pmatrix}
                            -2 & 2 & 1  \\
                            -1 & 2 & -2 \\
                            3  & 0 & 2
                        \end{pmatrix}
                    }_{G}
                    =
                    \underbrace{
                        \begin{pmatrix}
                            1            & 0              & 0 \\
                            \frac{4}{9}  & 1              & 0 \\
                            -\frac{4}{9} & -\frac{47}{65} & 1
                        \end{pmatrix}
                    }_{U}
                    \cdot
                    \underbrace{
                        \begin{pmatrix}
                            -2              & 2             & 1              \\
                            -\frac{1}{9}    & \frac{10}{9}  & -\frac{22}{9}  \\
                            \frac{132}{165} & \frac{22}{13} & -\frac{44}{65}
                        \end{pmatrix}
                    }_{G^*}
                    \)
                };
            \end{tikzpicture}
        \]
    }

    \vspace{2cm}

    \[
        \begin{tikzpicture}[remember picture,baseline=(E1.base)]
            \node (E1) {
                \(
                \underbrace{
                    \begin{pmatrix}
                        -2 & 2 & 1  \\
                        3  & 0 & 2  \\
                        -1 & 2 & -2
                    \end{pmatrix}
                }_{G}
                =
                \underbrace{
                    \begin{pmatrix}
                        1            & 0               & 0 \\
                        -\frac{4}{9} & 1               & 0 \\
                        0            & -\frac{47}{101} & 1
                    \end{pmatrix}
                }_{U}
                \cdot
                \underbrace{
                    \begin{pmatrix}
                        -2             & 2               & 1                \\
                        \frac{19}{9}   & \frac{8}{9}     & \frac{22}{9}     \\
                        \frac{88}{101} & \frac{154}{101} & -\frac{132}{101}
                    \end{pmatrix}
                }_{G^*}
                \)
            };
        \end{tikzpicture}
    \]




    \uncover<2->
    {
        \begin{tikzpicture}[remember picture,overlay]
            \draw[->,
                thick,
                % change l’angle de courbure
                shorten >=2pt,           % décale la pointe
                shorten <=2pt,           % décale la base
                >=Latex                  % style de la pointe
            ]
            (E1.north) to
            node[midway, left] {\textbf{\textcolor{red}{Swap}}}
            node[midway, right] {$\bg_3 \leftrightarrow \bg_2$}
            (E2.south);
        \end{tikzpicture}
    }


\end{frame}

%%%%%%%%%%%%%%%%%%%%%%%%%%%%%%%%%%%%%%%%%%%%%%%%%%%%%%%%%%%%%%%%%%%%%%%%%%%%%%%%%%%%%%%%%%%%%%%%%%%%%%%%%%%%%%%%%%%%%%%%%%%%%%%%%%%%%%%%%%%%%%%%%%%%%%%%%%%%%%%

\begin{frame}{Example}
    \uncover<2->
    {
        \[
            \begin{tikzpicture}[remember picture,baseline=(E1.base)]
                \node (E1)
                {
                    \(
                    \underbrace
                    {
                        \begin{pmatrix}
                            -2 & 2 & 1  \\
                            -1 & 2 & -2 \\
                            3  & 0 & 2
                        \end{pmatrix}
                    }_{G}
                    =
                    \underbrace
                    {
                        \begin{pmatrix}
                            1            & 0              & 0 \\
                            \frac{4}{9}  & 1              & 0 \\
                            -\frac{4}{9} & -\frac{47}{65} & 1
                        \end{pmatrix}
                    }_{U}
                    \cdot
                    \underbrace
                    {
                        \begin{pmatrix}
                            -2              & 2             & 1              \\
                            -\frac{1}{9}    & \frac{10}{9}  & -\frac{22}{9}  \\
                            \frac{132}{165} & \frac{22}{13} & -\frac{44}{65}
                        \end{pmatrix}
                    }_{G^*}
                    \)
                };
            \end{tikzpicture}
        \]
    }


    \vspace{2cm}

    \uncover<5->
    {
        \[
            \begin{tikzpicture}[remember picture,baseline=(E2.base)]
                \node (E2)
                {
                    \(
                    \underbrace{
                        \begin{pmatrix}
                            -2 & 2 & 1  \\
                            -1 & 2 & -2 \\
                            2  & 2 & 0
                        \end{pmatrix}
                    }_{G}
                    =
                    \underbrace{
                        \begin{pmatrix}
                            1           & 0             & 0 \\
                            \frac{4}{9} & 1             & 0 \\
                            0           & \frac{18}{65} & 1
                        \end{pmatrix}
                    }_{U}
                    \cdot
                    \underbrace{
                        \begin{pmatrix}
                            -2              & 2             & 1              \\
                            -\frac{1}{9}    & \frac{10}{9}  & -\frac{22}{9}  \\
                            \frac{132}{165} & \frac{22}{13} & -\frac{44}{65}
                        \end{pmatrix}
                    }_{G^*}
                    \)
                };
            \end{tikzpicture}
        \]
    }

    \only<3>
    {
        \begin{tikzpicture}[remember picture,overlay]
            \draw[->,
                thick,
                % change l’angle de courbure
                shorten >=2pt,           % décale la pointe
                shorten <=2pt,           % décale la base
                >=Latex                  % style de la pointe
            ]
            (E1.south) to
            node[midway, left] {\textbf{\textcolor{blue}{Size Reduction}}}
            node[midway, right] {$\bg_3 \leftarrow \bg_3-\nint{-\frac{47}{65}}\bg_2$}
            (E2.north);
        \end{tikzpicture}
    }
    \only<4->
    {
        \begin{tikzpicture}[remember picture,overlay]
            \draw[->,
                thick,
                % change l’angle de courbure
                shorten >=2pt,           % décale la pointe
                shorten <=2pt,           % décale la base
                >=Latex                  % style de la pointe
            ]
            (E1.south) to
            node[midway, left] {\textbf{\textcolor{blue}{Size Reduction}}}
            node[midway, right] {$\bg_3 \leftarrow \bg_3+1\bg_2$}
            (E2.north);
        \end{tikzpicture}
    }
\end{frame}

%%%%%%%%%%%%%%%%%%%%%%%%%%%%%%%%%%%%%%%%%%%%%%%%%%%%%%%%%%%%%%%%%%%%%%%%%%%%%%%%%%%%%%%%%%%%%%%%%%%%%%%%%%%%%%%%%%%%%%%%%%%%%%%%%%%%%%%%%%%%%%%%%%%%%%%%%%%%%%%

\begin{frame}{LLL: Example of a Reduced Basis}

    \pause We obtain the following LLL reduced basis:
    \[
        \bg_{\text{reduced}} =
        \begin{pmatrix}
            -2 & 2 & 1  \\
            -1 & 2 & -2 \\
            2  & 2 & 0
        \end{pmatrix}
    \]
    \pause
    The vector \( (2, 2, 0) \) is a shortest nonzero vector in the lattice, hence:

    \[
        \lambda_1(\LL) = 2\sqrt{2}.
    \]

\end{frame}

%%%%%%%%%%%%%%%%%%%%%%%%%%%%%%%%%%%%%%%%%%%%%%%%%%%%%%%%%%%%%%%%%%%%%%%%%%%%%%%%%%%%%%%%%%%%%%%%%%%%%%%%%%%%%%%%%%%%%%%%%%%%%%%%%%%%%%%%%%%%%%%%%%%%%%%%%%%%%%%



%\begin{frame}{Let's Try It in SageMath!}
%    
%    \centering
%    \Huge
%    \textbf{Pause.}\\[1em]
%    \normalsize
%    Let’s try out the LLL algorithm in \texttt{SageMath}\\
%    to approximate a solution to the Shortest Vector Problem.
%\end{frame}

%%%%%%%%%%%%%%%%%%%%%%%%%%%%%%%%%%%%%%%%%%%%%%%%%%%%%%%%%%%%%%%%%%%%%%%%%%%%%%%%%%%%%%%%%%%%%%%%%%%%%%%%%%%%%%%%%%%%%%%%%%%%%%%%%%%%%%%%%%%%%%%%%%%%%%%%%%%%%%%

\begin{frame}{LLL Complexity}
    \uncover<2->
    {
        \begin{smallalgo}{LLL}{algo:LLL_MCA}
            \LinesNumbered
            \DontPrintSemicolon
            \KwIn{A basis \( B = (\b_1, \ldots, \b_n) \)}
            \KwOut{An LLL-reduced basis \( G = (\bg_1, \ldots, \bg_n) \)}
            \( G \leftarrow copy(B)\)\;
            \tikz[remember picture, baseline=(gs.base)]{\node[inner sep=1pt] (gs){\( (G^*, U) \leftarrow \) Gram-Schmidt \( G \)\;};}

            \tikz[remember picture] \node (whilestart) {}; \While{\( i \leq n \)}{
                \tikz[remember picture] \node (forstart) {};\For{\( j = i-1, i-2, \ldots, 1 \)}{
                    \tikz[remember picture, baseline=(size.base)]{
                        \node[inner sep=1pt] (size){\( \bg_i \leftarrow \bg_i - \nint{\mu_{i,j}} \, \bg_j \), update \( (G^*, U) \)\;};}}\tikz[remember picture]\node(forend){};
                \tikz[remember picture]\node(lovstart){};
                \If{\( i > 1 \) \textbf{and} \( \|\bg_{i-1}^*\|^2 > 2 \|\bg_i^*\|^2 \)}{
                    \tikz[remember picture, baseline=(swap.base)]{\node[inner sep=1pt] (swap){Swap \( \bg_{i-1} \) and \( \bg_i \), update \((G^*, U)\)\;}; }
                    \( i \leftarrow i - 1 \)}
                \Else{\( i \leftarrow i + 1 \)\;} \tikz[remember picture] \node (lovend) {};}\KwRet{\( G \)}
        \end{smallalgo}
    }

    \begin{tikzpicture}[remember picture, overlay]
        \visible<3->
        {
            \node[draw=green, thick, rounded corners, fit=(gs), inner sep=0pt] {};
            \node[anchor=west] at ([xshift=0cm]gs.east) {\textbf{\textcolor{green}{ \( \OO(n^3) \)}}};
        }

        \visible<4->
        {
            \node[draw=blue, thick, rounded corners, fit=(size), inner sep=0pt] {};
            \node[anchor=west] at ([xshift=0cm]size.east) {\textbf{\textcolor{blue}{\( \OO(n) \)}}};
        }

        \visible<5->
        {
            \coordinate (A) at ($(forstart) + (0.1cm, 0cm)$);
            \coordinate (B) at ($(forend) + (7.3cm, +0.5cm)$);
            \node[draw=blue, thick, rounded corners, minimum height=0.5cm, inner sep=6pt, fit=(A)(B)] {};
            \node[anchor=west, text=blue, font=\bfseries] at ($(B) + (0.2cm, 0.3cm)$) {\( \OO(n^2) \)};
        }

        \visible<6->
        {
            \node[draw=red, thick, rounded corners, fit=(swap), inner sep=0pt] {};
            \node[anchor=west] at ([xshift=0cm]swap.east) {\textbf{\textcolor{red}{\( \OO(n) \)}}};
        }

        \visible<7->
        {
            \coordinate (A) at ($(lovstart) + (-0.3cm, 0cm)$);
            \coordinate (B) at ($(lovstart) + (6.9cm, -2.5cm)$);
            \node[draw=red, thick, rounded corners, minimum height=0.5cm, inner sep=6pt, fit=(A)(B)] {};
            \node[anchor=west, text=red, font=\bfseries] at ($(B) + (0.2cm, 0cm)$) {\( \OO(n) \)};
        }

        \visible<8->
        {
            \coordinate (A) at ($(whilestart) + (+0.4cm, 0cm)$);
            \coordinate (B) at ($(whilestart) + (2.5cm, -0.1cm)$);
            \node[draw=purple, thick, rounded corners, minimum height=0.5cm, inner sep=6pt, fit=(A)(B)] {};
            \node[anchor=west, text=purple, font=\bfseries] at ($(B) + (0.2cm, 0.1cm)$) {How much?};
        }
    \end{tikzpicture}
\end{frame}

%%%%%%%%%%%%%%%%%%%%%%%%%%%%%%%%%%%%%%%%%%%%%%%%%%%%%%%%%%%%%%%%%%%%%%%%%%%%%%%%%%%%%%%%%%%%%%%%%%%%%%%%%%%%%%%%%%%%%%%%%%%%%%%%%%%%%%%%%%%%%%%%%%%%%%%%%%%%%%%

\begin{frame}{Correctness}

    \vspace{0.5em}

    \textbf{Key idea:}
    Clearly, if the algorithm \( \LLL \) \textbf{terminates}, the returned basis is by construction \( \LLL \)-reduced.

    \vspace{1em}

    \textit{Therefore, it remains \textbf{to prove} that \( \LLL \) \textbf{always terminates}.}

\end{frame}

%%%%%%%%%%%%%%%%%%%%%%%%%%%%%%%%%%%%%%%%%%%%%%%%%%%%%%%%%%%%%%%%%%%%%%%%%%%%%%%%%%%%%%%%%%%%%%%%%%%%%%%%%%%%%%%%%%%%%%%%%%%%%%%%%%%%%%%%%%%%%%%%%%%%%%%%%%%%%%%

\begin{frame}{How can we prove the termination of the algorithm?}

    \begin{textblock*}{12cm}(0.5cm,2cm)
        \only<1>{\includegraphics[width=1\textwidth]{images/1.png}}
    \end{textblock*}

    \begin{textblock*}{12cm}(0.5cm,2cm)
        \only<2>{\includegraphics[width=1\textwidth]{images/2.png}}
    \end{textblock*}

    \begin{textblock*}{12cm}(0.5cm,2cm)
        \only<3>{\includegraphics[width=1\textwidth]{images/3.png}}
    \end{textblock*}

    \begin{textblock*}{12cm}(0.5cm,2cm)
        \only<4>{\includegraphics[width=1\textwidth]{images/4.png}}
    \end{textblock*}

    \begin{textblock*}{12cm}(0.5cm,2cm)
        \only<5>{\includegraphics[width=1\textwidth]{images/5.png}}
    \end{textblock*}
\end{frame}

%%%%%%%%%%%%%%%%%%%%%%%%%%%%%%%%%%%%%%%%%%%%%%%%%%%%%%%%%%%%%%%%%%%%%%%%%%%%%%%%%%%%%%%%%%%%%%%%%%%%%%%%%%%%%%%%%%%%%%%%%%%%%%%%%%%%%%%%%%%%%%%%%%%%%%%%%%%%%%%

\begin{frame}{How can we prove the termination of the algorithm?}
    \pause
    Let
    \(
    \bg_k =
    \begin{pmatrix}
        \bg_1  \\
        \bg_2  \\
        \vdots \\
        \bg_n
    \end{pmatrix}
    \).
    \pause
    We define
    \(
    d_k := \det(\bg_k \cdot \bg_k^t)
    \). \pause

    \( \rightarrow \) will be used to control the progress of the algorithm.
    \pause

    We have \pause
    \[
        \displaystyle d_k \pause = \det\left(\bg_k\bg_k^t\right)\pause=\det\left(U_k \bg_k^* (\bg_k^*)^t U_k^t\right)\pause = \det\left(\bg_k^* (\bg_k^*)^t\right)\pause = \prod_{1 \leq l \leq k} \| \mathbf{g}_l^* \|^2
    \]
    \pause If we \textbf{swap} \( \bg_i \) and \( \bg_{i-1}\) :

    \( \| \mathbf{d}_{i-1}^*\| \) decrease by a \( \frac{3}{4}\) factor, so \( \mathbf{d_{i-1}}\) decrease by a \( \frac{3}{4}\) factor.
\end{frame}

%%%%%%%%%%%%%%%%%%%%%%%%%%%%%%%%%%%%%%%%%%%%%%%%%%%%%%%%%%%%%%%%%%%%%%%%%%%%%%%%%%%%%%%%%%%%%%%%%%%%%%%%%%%%%%%%%%%%%%%%%%%%%%%%%%%%%%%%%%%%%%%%%%%%%%%%%%%%%%%

\begin{frame}{How can we prove the termination of the algorithm?}
    \pause
    We define
    \(
    \displaystyle \Z \ni D := \prod_{k=1}^{n-1} d_k > 1
    \)

    \pause \( \rightarrow \) After each swap, \( D \) decrease by a \( \frac{3}{4} \) factor.


    \pause Let \( D_0 \) be the value of \( D \) a the start of \( \LLL \), we have
    \[
        \displaystyle D_0 =\pause
        \prod_{k=1}^{n-1} d_k = \pause
        \prod_{k=1}^{n-1} \prod_{1 \leq l \leq k} \| \bg_l^* \|^2 = \pause
        \prod_{k=1}^{n-1} \| \bg_k^* \|^{2(n-k)}\pause
    \]
    \[
        \leq \prod_{k=1}^{n-1} \| \bg_k \|^{2(n-k)} \pause\leq
        \prod_{k=1}^{n-1} \left(\max_{1 \leq i \leq n} \|\bg_i\| \right)^{2(n-k)}
        \pause\leq \left(\max_{1 \leq i \leq n} \|\bg_i\| \right)^{n(n-1)}
    \]


    \pause\textbf{Termination proof}
    \[
        \pause \displaystyle 1 \leq \underbrace{\cdots}_{ \displaystyle \OO \left( \log \left(\max_{1 \leq i \leq n} \|\bg_i\| \right) \right) \text{ steps }} \leq \frac{4}{3} D_1 \leq D_0 \leq \left(\max_{1 \leq i \leq n} \|\bg_i\| \right)^{n(n-1)}
    \]
\end{frame}

%%%%%%%%%%%%%%%%%%%%%%%%%%%%%%%%%%%%%%%%%%%%%%%%%%%%%%%%%%%%%%%%%%%%%%%%%%%%%%%%%%%%%%%%%%%%%%%%%%%%%%%%%%%%%%%%%%%%%%%%%%%%%%%%%%%%%%%%%%%%%%%%%%%%%%%%%%%%%%%

\begin{frame}{LLL Complexity}
    \uncover<2->
    {
        \begin{smallalgo}{LLL}{algo:LLL_MCA}
            \LinesNumbered
            \DontPrintSemicolon
            \KwIn{A basis \( B = (\b_1, \ldots, \b_n) \)}
            \KwOut{An LLL-reduced basis \( G = (\bg_1, \ldots, \bg_n) \)}

            \( G \leftarrow copy(B)\)


            \tikz[remember picture, baseline=(gs.base)]
            {
                \node[inner sep=1pt] (gs)
                {
                    \( (G^*, U) \leftarrow \) Gram-Schmidt \( G \)\;
                };
            }

            \tikz[remember picture] \node (whilestart) {};
            \While{\( i \leq n \)}{
                \tikz[remember picture] \node (forstart) {};
                \For{\( j = i-1, i-2, \ldots, 1 \)}{
                    \tikz[remember picture, baseline=(size.base)]
                    {
                        \node[inner sep=1pt] (size)
                        {
                            \( \bg_i \leftarrow \bg_i - \nint{\mu_{i,j}} \, \bg_j \), update \( (G^*, U) \)\;
                        };
                    }
                }
                \tikz[remember picture]\node(forend){};
                \tikz[remember picture]\node(lovstart){};
                \If{
                    \( i > 1 \) \textbf{and} \( \|\bg_{i-1}^*\|^2 > 2 \|\bg_i^*\|^2 \)
                }
                {
                    \tikz[remember picture, baseline=(swap.base)]
                    {
                        \node[inner sep=1pt] (swap)
                        {
                            Swap \( \bg_{i-1} \) and \( \bg_i \), update \((G^*, U)\)\;
                        };
                    }
                    \( i \leftarrow i - 1 \)\;
                }
                \Else{
                    \( i \leftarrow i + 1 \)\;
                }
                \tikz[remember picture] \node (lovend) {};
            }
            \KwRet{\( G \)}
        \end{smallalgo}
    }

    \begin{tikzpicture}[remember picture, overlay]
        \visible<3->
        {
            \node[draw=green, thick, rounded corners, fit=(gs), inner sep=0pt] {};
            \node[anchor=west] at ([xshift=0cm]gs.east) {\textbf{\textcolor{green}{ \( \OO(n^3) \)}}};
        }

        \visible<4->
        {
            \node[draw=blue, thick, rounded corners, fit=(size), inner sep=0pt] {};
            \node[anchor=west] at ([xshift=0cm]size.east) {\textbf{\textcolor{blue}{\( \OO(n) \)}}};
        }

        \visible<5->
        {
            \coordinate (A) at ($(forstart) + (0.1cm, 0cm)$);
            \coordinate (B) at ($(forend) + (7.3cm, +0.5cm)$);
            \node[draw=blue, thick, rounded corners, minimum height=0.5cm, inner sep=6pt, fit=(A)(B)] {};
            \node[anchor=west, text=blue, font=\bfseries] at ($(B) + (0.2cm, 0.3cm)$) {\( \OO(n^2) \)};
        }

        \visible<6->
        {
            \node[draw=red, thick, rounded corners, fit=(swap), inner sep=0pt] {};
            \node[anchor=west] at ([xshift=0cm]swap.east) {\textbf{\textcolor{red}{\( \OO(n) \)}}};
        }

        \visible<7->
        {
            \coordinate (A) at ($(lovstart) + (-0.3cm, 0cm)$);
            \coordinate (B) at ($(lovstart) + (6.9cm, -2.5cm)$);
            \node[draw=red, thick, rounded corners, minimum height=0.5cm, inner sep=6pt, fit=(A)(B)] {};
            \node[anchor=west, text=red, font=\bfseries] at ($(B) + (0.2cm, 0cm)$) {\( \OO(n) \)};
        }

        \visible<8->
        {
            \coordinate (A) at ($(whilestart) + (+0.4cm, 0cm)$);
            \coordinate (B) at ($(whilestart) + (2.5cm, -0.1cm)$);
            \node[draw=purple, thick, rounded corners, minimum height=0.5cm, inner sep=6pt, fit=(A)(B)] {};
            \node[anchor=west, text=purple, font=\bfseries] at ($(B) + (0.2cm, 0.1cm)$) {\( \OO(n^2 log(A)) \)};
        }
    \end{tikzpicture}
\end{frame}

%%%%%%%%%%%%%%%%%%%%%%%%%%%%%%%%%%%%%%%%%%%%%%%%%%%%%%%%%%%%%%%%%%%%%%%%%%%%%%%%%%%%%%%%%%%%%%%%%%%%%%%%%%%%%%%%%%%%%%%%%%%%%%%%%%%%%%%%%%%%%%%%%%%%%%%%%%%%%%%

\begin{frame}{Theorem: Complexity of BasisReduction}
    \pause \textbf{Theorem.}

    \pause \( \bullet ~~ \LLL \) uses \( \displaystyle \OO \left(n^2 \log \left(\max_{1 \leq i \leq n} \|\b_i\| \right)\right) \) loop iterations.

    \pause \( \bullet ~~ \LLL \) uses \( \displaystyle \OO \left(n^2 \right) \) arithmetic operations over rationals per iteration.

    \pause \( \bullet ~~ U \) represented with rationals of bit-lengths \( \displaystyle \OO \left(n \log \left(  \max_{1 \leq i \leq n} \|\b_i\| \right) \right) \)

    \pause \( \Rightarrow \LLL \) uses \( \displaystyle \widetilde{\OO}\left( n^5 \log^2 \left(  \max_{1 \leq i \leq n} \|\b_i\| \right) \right) \) bit operations.

    \vspace{1cm}
    \pause \textbf{Theorem.}

    \pause $\rightarrow$ \( \LLL \) \textbf{compute} a reduced basis in \textbf{polynomial time}.

    \pause $\rightarrow$ \( \LLL \) \textbf{solve} \( 2^{\OO(n)}- \SVP \) in \textbf{polynomial time}.

\end{frame}

%%%%%%%%%%%%%%%%%%%%%%%%%%%%%%%%%%%%%%%%%%%%%%%%%%%%%%%%%%%%%%%%%%%%%%%%%%%%%%%%%%%%%%%%%%%%%%%%%%%%%%%%%%%%%%%%%%%%%%%%%%%%%%%%%%%%%%%%%%%%%%%%%%%%%%%%%%%%%%%

\begin{frame}[c]
    \centering
    \Huge
    \textbf{Thank you for your attention!}

    \vspace{2em}
    \LARGE
    Questions?
\end{frame}
\end{document}
