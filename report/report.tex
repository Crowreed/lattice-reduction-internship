%%%%%%%%%%%%%%%%%%%%%%%%%%%%%%%%%%%%%%%%%%%%%%%%%%%%%%%%%%%%%%%%%%%%%%%%%%%%%%%%%%%%%%%%%%%%%%%%%%%%%%%%%%%%%%%%%%%%%%%%%%%%%%%%%%%%%%%%%%%%%%%%%%%%%%%%%%%%%%%
%                                                                                                                                                             %
%                                                            ▗▄▄▖ ▗▄▄▄▖▗▄▄▖  ▗▄▖ ▗▄▄▖▗▄▄▄▖                                                                    %
%                                                            ▐▌ ▐▌▐▌   ▐▌ ▐▌▐▌ ▐▌▐▌ ▐▌ █                                                                      %
%                                                            ▐▛▀▚▖▐▛▀▀▘▐▛▀▘ ▐▌ ▐▌▐▛▀▚▖ █                                                                      %
%                                                            ▐▌ ▐▌▐▙▄▄▖▐▌   ▝▚▄▞▘▐▌ ▐▌ █                                                                      %
%                                                                                                                                                             %
%%%%%%%%%%%%%%%%%%%%%%%%%%%%%%%%%%%%%%%%%%%%%%%%%%%%%%%%%%%%%%%%%%%%%%%%%%%%%%%%%%%%%%%%%%%%%%%%%%%%%%%%%%%%%%%%%%%%%%%%%%%%%%%%%%%%%%%%%%%%%%%%%%%%%%%%%%%%%%%
%                                                                                                                                                             %
% Auteur :          Lucas Noirot                                                                                                                              %
% Éditeur utilisé : TeXstudio                                                                                                                                 %
% Compilateur :     pdfLaTeX (dernière version au moment de la rédaction)                                                                                     %
%                                                                                                                                                             %
% Ce document est libre de droit. Toute personne est libre de l'utiliser, le modifier ou le redistribuer, sans restriction.                                   %
% Licence : Creative Commons CC0 – https://creativecommons.org/publicdomain/zero/1.0/                                                                         %
%                                                                                                                                                             %
%%%%%%%%%%%%%%%%%%%%%%%%%%%%%%%%%%%%%%%%%%%%%%%%%%%%%%%%%%%%%%%%%%%%%%%%%%%%%%%%%%%%%%%%%%%%%%%%%%%%%%%%%%%%%%%%%%%%%%%%%%%%%%%%%%%%%%%%%%%%%%%%%%%%%%%%%%%%%%%
%                                                                                                                                                             %
% Les sous-fichiers sont :                                                                                                                                    %
%                                                                                                                                                             %
% preamble.tex                     cover_page.tex                           acknowledgement.tex                            notations_complexity_acronyms.tex  %
%                                                                                                                                                             %
% introduction.tex                 conclusion_and_future_work.tex           polynomial_lattice_reduction_integer_case.tex                                     %
%                                                                                                                                                             %
% lattice_basics.tex               lattice_reduction.tex                    polynomial_lattice_reduction.tex               readin\bg_guide.tex                  %
%                                                                                                                                                             %
% euclidean_lattice_reduction.tex  lattices_by_relations_not_generator.tex  appendix.tex                                                                      %
%                                                                                                                                                             %
% Pas de fichier parent                                                                                                                                       %
%                                                                                                                                                             %
%%%%%%%%%%%%%%%%%%%%%%%%%%%%%%%%%%%%%%%%%%%%%%%%%%%%%%%%%%%%%%%%%%%%%%%%%%%%%%%%%%%%%%%%%%%%%%%%%%%%%%%%%%%%%%%%%%%%%%%%%%%%%%%%%%%%%%%%%%%%%%%%%%%%%%%%%%%%%%%
%                                                                                                                                                             %
\documentclass[a4paper,12pt]{report}  % Déclare un document de type "report" (idéal pour les rapports longs)                                                  %
%                                     % a4paper : Définit le format du papier en A4 (210 × 297 mm)                                                            %
%                                     % 12pt : Définit la taille de police principale à 12 points (standard pour les rapports)                                %
%                                                                                                                                                             %
\input{preamble}                                                                                                                                              %
%                                                                                                                                                             %
%%%%%%%%%%%%%%%%%%%%%%%%%%%%%%%%%%%%%%%%%%%%%%%%%%%%%%%%%%%%%%%%%%%%%%%%%%%%%%%%%%%%%%%%%%%%%%%%%%%%%%%%%%%%%%%%%%%%%%%%%%%%%%%%%%%%%%%%%%%%%%%%%%%%%%%%%%%%%%%
%                                                                                                                                                             %
\titleformat                                                                                                                                                  %
{\chapter}                                                                                                                                          % command %
[display]                                                                                                                                             % shape %
{\Large}                                                                                                                                             % format %
{                                                                                                                                                             %
	\rule{\widthof{CHAPITRE}}{0.6pt}\vspace{-0.3cm}\\                                                                                                         %
	\makebox[0pt][l]{CHAPITRE \Huge\textbf{\thechapter}}\vspace{-0.6cm} \\                                                                                    %
	\rule{\widthof{CHAPITRE \Huge\textbf{\thechapter}}}{0.6pt}                                                                                                %
}                                                                                                                                                     % label %
{1ex}                                                                                                                                                   % sep %
{\huge\textit }                                                                                                                                               %
[                                                                                                                                                             %
]                                                                                                                                                % after-code %
%                                                                                                                                                             %
%%%%%%%%%%%%%%%%%%%%%%%%%%%%%%%%%%%%%%%%%%%%%%%%%%%%%%%%%%%%%%%%%%%%%%%%%%%%%%%%%%%%%%%%%%%%%%%%%%%%%%%%%%%%%%%%%%%%%%%%%%%%%%%%%%%%%%%%%%%%%%%%%%%%%%%%%%%%%%%
%                                                                                                                                                             %
\begin{document}                                                                                                                                              %
%                                                                                                                                                             %
\pagenumbering{gobble} % désactive la numérotation


\include{cover_page}                                     % Page de couverture (non numérotée, hors table des matières)                                        %
%                                                                                                                                                             %
\include{acknowledgements}                               % Remerciements (non numérotée, hors table des matières)                                             %
%                                                                                                                                                             %
\tableofcontents                                         % Table des matières                                                                                 %
%                                                                                                                                                             %
\listofalgorithms
\addcontentsline{toc}{chapter}{Liste des algorithmes}

\listoffigures
\addcontentsline{toc}{chapter}{Table des figures}

%                                                                                                                                                             %
\include{notations_complexity_acronyms}                  % Notations, complexité, acronymes — pose les conventions du mémoire                                 %

\include{reading_guide}                                  % Guide de lecture — explication du plan, structure, conseils pour le lecteur                        %

\chapter*{Déroulé du stage et motivations}
\addcontentsline{toc}{chapter}{Déroulé du stage et motivations}

Ce stage de Master 2 s’est déroulé au sein de l’équipe \textsc{ECO} (Exact COmputing) du LIRMM, à l’Université de Montpellier. Il s’inscrit dans le cadre d’un stage académique en informatique théorique, axé sur l’étude de la réduction de réseaux euclidiens, un sujet en calcul formel et en cryptographie.

L’ensemble du code développé durant le stage est disponible à l’adresse suivante : \texttt{[lien vers dépôt/code]}.

\section*{Objectifs et déroulé du stage}

L’objectif principal du stage était d'essayer d'apater des techniques de réductions de réseaux euclidiens avec des techniques issues du monde polynomiale, en s’appuyant à la fois sur les fondements théoriques et sur des expérimentations.

\begin{itemize}
    \item[\( \bullet \)] \textbf{Phase 1: Compréhension des réseaux euclidiens et de l’algorithme LLL} \\
    Le début du stage a été consacré à l’étude approfondie de l’algorithme de Lenstra–Lenstra–Lovász (LLL), de sa preuve de correction ainsi que de la bibliographie associée sur les réseaux euclidiens. Implémentation de LLL en SageMath. Cette étape a également permis de préparer la rédaction du rapport intermédiaire.
    
    \item[\( \bullet \)] \textbf{Phase 2: Compréhension des réseaux polynomiaux et algorithmes de réduction} \\
    Dans un second temps, le travail s’est tourné vers le cas des réseaux polynomiaux : lecture de la littérature, compréhension des algorithmes (\textsc{basis}, \textsc{PM-basis}), et mise en perspective avec le cas euclidien. Implémentation dans SageMath.
    
    \item[\( \bullet \)] \textbf{Phase 3: Rapport intermédiaire et retour encadrant} \\
    La remise du rapport intermédiaire a permis de faire le point sur les avancées, de mieux cadrer le positionnement du stage, et de structurer l’état de l’art.
    
    \item[\( \bullet \)] \textbf{Phase 4: Réflexions d’adaptation au cas entier} \\
    Durant la seconde moitié du stage, des pistes d’adaptation d’algorithmes issus du cas polynomial ont été explorées pour les réseaux entiers.
\end{itemize}

\section*{Participation à la vie du laboratoire}

Durant le stage, j’ai également eu l’occasion de m’impliquer dans l’environnement de recherche local :
\begin{itemize}
    \item[\( \bullet \)] Présentation d’un exposé de 1h30 (en anglais) sur l’algorithme LLL et sa preuve lors du \textit{Lattice Club}, un séminaire interne dédié aux réseaux.
    \item[\( \bullet \)] Participation hebdomadaire aux séances du Lattice Club, animées notamment par Katarina et d’autres intervenants, où différents aspects de la cryptographie à base de réseaux ont été abordés.
    \item[\( \bullet \)] Présence régulière aux séminaires du laboratoire organisés les mardis.
\end{itemize}



\include{introduction}                                   % Introduction générale (non numérotée)                                                              %
%                                                                                                                                                             %



\pagenumbering{arabic}
%                                                                                                                                                             %

%%%%%%%%%%%%%%%%%%%%%%%%%%%%%%%%%%%%%%%%%%%%%%%%%%%%%%%%%%%%%%%%%%%%%%%%%%%%%%%%%%%%%%%%%%%%%%%%%%%%%%%%%%%%%%%%%%%%%%%%%%%%%%%%%%%%%%%%%%%%%%%%%%%%%%%%%%%%%%%
%                                                                                                                                                             %
%                                                       ▗▄▄▖▗▖ ▗▖ ▗▄▖ ▗▄▄▖▗▄▄▄▖▗▄▄▄▖▗▄▄▖     ▗▄▄▄▖                                                            %
%                                                      ▐▌   ▐▌ ▐▌▐▌ ▐▌▐▌ ▐▌ █  ▐▌   ▐▌ ▐▌      █                                                              %
%                                                      ▐▌   ▐▛▀▜▌▐▛▀▜▌▐▛▀▘  █  ▐▛▀▀▘▐▛▀▚▖      █                                                              %
%                                                      ▝▚▄▄▖▐▌ ▐▌▐▌ ▐▌▐▌    █  ▐▙▄▄▖▐▌ ▐▌    ▗▄█▄▖                                                            %
%                                                                                                                                                             %
%                                           ▗▖    ▗▄▖▗▄▄▄▖▗▄▄▄▖▗▄▄▄▖ ▗▄▄▖▗▄▄▄▖    ▗▄▄▖  ▗▄▖  ▗▄▄▖▗▄▄▄▖ ▗▄▄▖ ▗▄▄▖                                              %
%                                           ▐▌   ▐▌ ▐▌ █    █    █  ▐▌   ▐▌       ▐▌ ▐▌▐▌ ▐▌▐▌     █  ▐▌   ▐▌                                                 %
%                                           ▐▌   ▐▛▀▜▌ █    █    █  ▐▌   ▐▛▀▀▘    ▐▛▀▚▖▐▛▀▜▌ ▝▀▚▖  █  ▐▌    ▝▀▚▖                                              %
%                                           ▐▙▄▄▖▐▌ ▐▌ █    █  ▗▄█▄▖▝▚▄▄▖▐▙▄▄▖    ▐▙▄▞▘▐▌ ▐▌▗▄▄▞▘▗▄█▄▖▝▚▄▄▖▗▄▄▞▘                                              %
%                                                                                                                                                             %
%%%%%%%%%%%%%%%%%%%%%%%%%%%%%%%%%%%%%%%%%%%%%%%%%%%%%%%%%%%%%%%%%%%%%%%%%%%%%%%%%%%%%%%%%%%%%%%%%%%%%%%%%%%%%%%%%%%%%%%%%%%%%%%%%%%%%%%%%%%%%%%%%%%%%%%%%%%%%%%
%                                                                                                                                                             %
% Les sous-fichiers sont :                                                                                                                                    %
%                                                                                                                                                             %
% polynomial_lattice_basics.tex     euclidean_lattice_basics.tex                                                                                              %
%                                                                                                                                                             %
% Le fichier parent est : report.tex                                                                                                                          %
%                                                                                                                                                             %
%%%%%%%%%%%%%%%%%%%%%%%%%%%%%%%%%%%%%%%%%%%%%%%%%%%%%%%%%%%%%%%%%%%%%%%%%%%%%%%%%%%%%%%%%%%%%%%%%%%%%%%%%%%%%%%%%%%%%%%%%%%%%%%%%%%%%%%%%%%%%%%%%%%%%%%%%%%%%%%


%                                                                                                                                                             %
%%%%%%%%%%%%%%%%%%%%%%%%%%%%%%%%%%%%%%%%%%%%%%%%%%%%%%%%%%%%%%%%%%%%%%%%%%%%%%%%%%%%%%%%%%%%%%%%%%%%%%%%%%%%%%%%%%%%%%%%%%%%%%%%%%%%%%%%%%%%%%%%%%%%%%%%%%%%%%%
%                                                                                                                                                             %
%                                                ▗▄▄▖▗▄▄▄▖ ▗▄▄▖▗▄▄▄▖▗▄▄▄▖ ▗▄▖ ▗▖  ▗▖    ▗▄▄▄▖        ▗▄▄▄▖                                                    %
%                                               ▐▌   ▐▌   ▐▌     █    █  ▐▌ ▐▌▐▛▚▖▐▌      █            █                                                      %
%                                                ▝▀▚▖▐▛▀▀▘▐▌     █    █  ▐▌ ▐▌▐▌ ▝▜▌      █            █                                                      %
%                                               ▗▄▄▞▘▐▙▄▄▖▝▚▄▄▖  █  ▗▄█▄▖▝▚▄▞▘▐▌  ▐▌    ▗▄█▄▖        ▗▄█▄▖                                                    %
%                                                                                                                                                             %
%                ▗▄▄▄▖▗▖ ▗▖ ▗▄▄▖▗▖   ▗▄▄▄▖▗▄▄▄ ▗▄▄▄▖ ▗▄▖ ▗▖  ▗▖    ▗▖    ▗▄▖▗▄▄▄▖▗▄▄▄▖▗▄▄▄▖ ▗▄▄▖▗▄▄▄▖    ▗▄▄▖  ▗▄▖  ▗▄▄▖▗▄▄▄▖ ▗▄▄▖ ▗▄▄▖                       %
%                ▐▌   ▐▌ ▐▌▐▌   ▐▌     █  ▐▌  █▐▌   ▐▌ ▐▌▐▛▚▖▐▌    ▐▌   ▐▌ ▐▌ █    █    █  ▐▌   ▐▌       ▐▌ ▐▌▐▌ ▐▌▐▌     █  ▐▌   ▐▌                          %
%                ▐▛▀▀▘▐▌ ▐▌▐▌   ▐▌     █  ▐▌  █▐▛▀▀▘▐▛▀▜▌▐▌ ▝▜▌    ▐▌   ▐▛▀▜▌ █    █    █  ▐▌   ▐▛▀▀▘    ▐▛▀▚▖▐▛▀▜▌ ▝▀▚▖  █  ▐▌    ▝▀▚▖                       %
%                ▐▙▄▄▖▝▚▄▞▘▝▚▄▄▖▐▙▄▄▖▗▄█▄▖▐▙▄▄▀▐▙▄▄▖▐▌ ▐▌▐▌  ▐▌    ▐▙▄▄▖▐▌ ▐▌ █    █  ▗▄█▄▖▝▚▄▄▖▐▙▄▄▖    ▐▙▄▞▘▐▌ ▐▌▗▄▄▞▘▗▄█▄▖▝▚▄▄▖▗▄▄▞▘                       %
%                                                                                                                                                             %
%%%%%%%%%%%%%%%%%%%%%%%%%%%%%%%%%%%%%%%%%%%%%%%%%%%%%%%%%%%%%%%%%%%%%%%%%%%%%%%%%%%%%%%%%%%%%%%%%%%%%%%%%%%%%%%%%%%%%%%%%%%%%%%%%%%%%%%%%%%%%%%%%%%%%%%%%%%%%%%
%                                                                                                                                                             %
% Pas de sous-fichiers                                                                                                                                        %
%                                                                                                                                                             %
% Le fichier parent est : lattice_basics.tex                                                                                                                  %
%                                                                                                                                                             %
%%%%%%%%%%%%%%%%%%%%%%%%%%%%%%%%%%%%%%%%%%%%%%%%%%%%%%%%%%%%%%%%%%%%%%%%%%%%%%%%%%%%%%%%%%%%%%%%%%%%%%%%%%%%%%%%%%%%%%%%%%%%%%%%%%%%%%%%%%%%%%%%%%%%%%%%%%%%%%%
%                          



\chapter{Réseaux euclidiens}

\lettrine{\textbf{L}}{'étude} des réseaux euclidiens puise ses racines dans les travaux mathématiques du XVIII\textsuperscript{e} siècle, notamment ceux de Leonhard Euler sur l'organisation géométrique des points dans l'espace. C'est cependant en 1891 qu'Hermann Minkowski établit véritablement les fondements modernes avec l'introduction de la théorie géométrique des nombres, reliant explicitement les réseaux à divers problèmes d'optimisation et de minimisation. Ses résultats joueront ultérieurement un rôle déterminant dans le développement de la cryptographie moderne. Néanmoins, c'est au cours du XX\textsuperscript{e} siècle, et plus particulièrement à partir des années 1990, que les réseaux euclidiens connaissent une véritable intégration dans la cryptographie. Les travaux de chercheurs tels que Ajtai, Dwork ou Regev marquent alors un tournant décisif, en démontrant que les difficultés algorithmiques intrinsèques aux réseaux peuvent constituer une base solide pour la conception de nouveaux systèmes cryptographiques résistants aux attaques conventionnelles et quantiques. Ce chapitre vise précisément à introduire de manière approfondie les concepts fondamentaux liés aux réseaux euclidiens.
                                                                                                                                   %
\vspace{-0.1cm}                                                                                         %
\section{Définitions et exemples}
\vspace{-0.1cm}
Nous considérons un espace euclidien, c’est-à-dire un espace vectoriel réel de dimension finie muni d’un produit scalaire, noté $\langle \bf, \bg \rangle$. Dans ce chapitre, nous utiliserons le produit scalaire usuel défini par $\langle \bf, \bg \rangle := \bf \cdot \bg^t = \sum_{i=1}^{n} \bf_i \bg_i$, lequel induit la norme-$2$ donnée par $\|\bf\|_2 = \sqrt{\sum_{i=1}^{n} \bf_i^2}$. Pour alléger les notations, nous omettrons l’indice $2$. Ce chapitre se contentera d’exposer les résultats classiques, sans chercher à les redémontrer. Rappelons qu’au sein de $\R^n$, toutes les normes sont équivalentes, ce qui signifie qu’aucune ne change la nature intrinsèque des problèmes que nous aborderons.  Nous nous concentrerons néanmoins sur la norme-$2$ pour sa simplicité géométrique, car elle offre une mesure intuitive des longueurs et des angles, point essentiel pour étudier la structure des réseaux euclidiens. Afin de simplifier l’écriture, nous désignerons par $\mathcal{B} = (\bb_i)_{1 \leq i \leq n}$ une base de $\R^n$ et par $\mathcal{B}^* = (\bb^*_i)_{1 \leq i \leq n}$ sa base orthogonalisée par le procédé de Gram-Schmidt. Les matrices $B$ et $B^*$ (sans calligraphie) seront composées (en ligne) des vecteurs de la famille $(\bb_i)_{1 \leq i \leq n}$ et $(\bb^*_i)_{1 \leq i \leq n}$ respectivement.

\vspace{0.2cm}

On commence par introduire la notion de réseau euclidien. Pour un rappel sur les groupes ou les modules le lecteur est invité à lire les annexes correspondantes. Le point de vue de module nous sera utile dans la suite de ce manuscrit.

\vspace{-0.2cm}

\begin{definition}
	Soit $n \in \N^*$, $\bb_1, \ldots, \bb_n \in \R^n$. Les définitions suivantes sont équivalentes.\footnote{Plusieurs définitions équivalentes d’un réseau euclidien coexistent dans la littérature, selon que l’on se place ou non dans un espace euclidien $\R^n$, ou dans un espace $\R^n$ équipé explicitement d’une forme quadratique définie positive. Dans tous les cas, l’idée générale reste la même : un réseau euclidien est un sous-ensemble discret de $\R^n$ formé par toutes les combinaisons linéaires entières d’un ensemble de vecteurs générateurs.}
	\begin{itemize}
		\item[$\bullet$] Un \textbf{réseau euclidien} $\LL$ est un sous-groupe discret additif de $\R^n$.
        \begin{itemize}
            \item Sous-groupe additif : $\mathbf{0} \in \LL$, et pour tout $\bx, \by \in \LL$, $\bx + \by, -\bx \in \LL$
            \item Discret : $\forall \bx \in \LL, \exists \varepsilon > 0$ tel que $\mathcal{B}(\bx, \varepsilon) \cap \LL = \{\bx\}$ (où $\mathcal{B}(\bx, \varepsilon)$ désigne la boule ouverte de rayon $\varepsilon$ centrée en $\bx$).
        \end{itemize}
		\item[$\bullet$] Un \textbf{réseau euclidien} $\LL$ est un $\Z$-module libre de type fini de $\R^n$.
	\end{itemize}
\end{definition}

\begin{definition}
    Un \textbf{sous-réseau} $\LL'$ de $\LL$ est un sous groupe de $\LL$, on notera $\LL' \subseteq \LL$.
\end{definition}

\begin{example}
    Les entiers de Gauss, définit par $\Z[i]:= \Z \oplus i \Z$ forment un réseau de rang $2$ dans $\C$, c'est même un anneau.
\end{example}

\begin{example}
    Un exemple plus exotique, $\Z \oplus \sqrt{2} \cdot \Z$ est un réseau de rang $2$ dans $\R$.
\end{example}

\begin{counterexample}
    $\Q$ n'est pas un réseau euclidien , car $\Q$ est dense dans $\R$, ce qui brise la discrétude, bien que ce soit un sous-groupe de $\R$.
\end{counterexample}

En particulier on peut montrer qu'il existe une famille $\Z$-libre maximale $(\bb_i)_{1 \leq i \leq m}$ dans $\LL$ telle que 
$$\LL = \bigoplus\limits_{1 \leq i \leq m} \Z \bb_i:=\{a_1\bb_1 + \cdots a_m\bb_m : a_i \in \Z\}$$
Cette famille est appelée \textbf{base} de $\LL$, si on note $B$ la matrice de la famille $(\bb_i)_{1 \leq i \leq m}$ on notera $\LL(B)$ le réseau de base $B$, donc \textbf{engendré} par la famille $(\bb_i)_{1 \leq i \leq m}$. L'entier $m$ est commun à toutes les bases de $\LL$ et on l'appelle \textbf{rang} de $\LL$. Lorsque $n=m$, on dit que le réseau est de \textbf{rang plein}.

\begin{example}
     On a la suite d'inclusions $2\Z \subset \Z \subset \frac{1}{2}\Z$. \footnote{On rappelle que pour \( a \in \R \), on a \( a\Z = \{ an \in \R | n \in \Z \} \).} Bien que $\rang(2\Z) = \rang(\Z) = \rang(\frac{1}{2}\Z)$, ces ensembles sont distincts. Cela montre que, contrairement aux espaces vectoriels, avoir une relation d'inclusion et avoir le même rang ne suffit pas à garantir l'égalité des réseaux.
\end{example}

Existe-t-il une notion de \textbf{bonne} base ? Nous verrons qu'une base idéale est celle qui est en un sens la plus orthogonale possible. Il n'existe pas toujours de base strictement orthogonale, ce qui justifie la notion de quasi-orthogonalité. Nous allons rajouter des façons de mesurer la qualité d'une base dans le chapitre suivant et introduire la notion de réduction, qui consistera à trouver une telle base.

\begin{figure}[h]
    \centering
    \begin{subfigure}[b]{0.25\textwidth}
        \centering
        \includegraphics[width=\textwidth]{images/lattice_0_2.png}
        \caption{$B = \begin{pmatrix}
                3 & -1 \\ 
                2 & 0
            \end{pmatrix}$.}
    \end{subfigure}
    \hspace{0.05\textwidth}
    \begin{subfigure}[b]{0.25\textwidth}
        \centering
        \includegraphics[width=\textwidth]{images/lattice_1_2.png}
        \caption{$B' = \begin{pmatrix}
                1 & 1 \\ 
                1 & -1
            \end{pmatrix}$}
    \end{subfigure}
    \caption{Deux bases pour le même réseau de $\R^2$.}
    \label{fig:bases_lattice}
\end{figure}

\begin{definition}
    La \textbf{taille}\footnote{\textbf{taille} et \textbf{volume} sont synonymes, mais l'usage le plus courant dans la littérature anglaise est "déterminant du réseau" ou "volume".} d'un réseau $\LL(B)$ est $|\det(B)|$ et est noté $|\LL|$. La taille d'un réseau est indépendante de la base choisie.
\end{definition} 



\begin{remark}
    On peut définir de façon équivalente $|\LL| := \sqrt{ \det( \Gram(B))}$ où $\Gram(B) = B^t \cdot B$. On peut voir $|\LL|$ comme le \textbf{volume du domaine fondamental} de $\left\{ \sum\limits_{i=0}^{n} \lambda_i \bb_i \mid 0 \leq \lambda_i < 1 \right\}.$
\end{remark}
\begin{theoreme}[Inégalité d'Hadamard]
    Soit $B \in M_n(\K)$ et $\LL(B)$ un réseau.
    Alors 
    $$|\LL|=|\det(B)| \leq \prod_{i=1}^{n} \| \bb_i \|$$
    
    La borne est atteinte si, et seulement si $(\bb_i)_{1 \leq i \leq n}$ est une famille orthogonale.	
\end{theoreme}
Toutes les bases d'un réseau euclidien diffèrent d'une transformation de déterminant $\pm 1$. L'ensemble de ces transformations est connu sous le nom de groupe unimodulaire. Un ensemble de points non alignés dans un réseau ne constitue pas une base si son déterminant est différent de $\pm |\LL|$.

\begin{proposition}
    Soit $\LL$ et $\LL'$ deux réseaux de rang $n$ de base $B$ et $B'$. Alors $\LL$ = $\LL'$ si et seulement si il existe $ U \in \mathrm{GL}_n(\Z)$ tel que $B'=BU$. Où $\mathrm{GL}_n(\Z)= \{M \in \Z^{n \x n} | \det (M) = \pm 1\}$.
\end{proposition}

\begin{remark}
	On a l'action de groupe
	\begin{align*}
		\mathrm{GL}_n(\Z) \x \mathrm{GL}_n(\R) &\longrightarrow \mathrm{GL}_n(\R) \\
		(U, \mathbf{B}) &\longmapsto \mathbf{B}U
	\end{align*}	
	Un réseau est exactement une orbite de cette action. Nous verrons dans le chapitre suivant que la réduction de réseaux consiste à trouver un bon représentant pour chaque orbite.
\end{remark}

Nous présentons maintenant deux \textbf{invariants} fondamentaux d'un réseau : 

\begin{itemize}
	\item[$\bullet$] La \textbf{taille du vecteur minimal} du réseau, notée $\lambda_1(\LL)$.
	\item[$\bullet$] Le \textbf{volume du réseau}, aussi appelé la \textbf{taille du réseau} souvent désigné par $|\LL|$.
\end{itemize}



\begin{proposition}
	Soit $\LL$, $\LL'$  deux réseaux de $\R^n$ tel que $\LL' \subseteq \LL$ alors $\frac{|\LL'|}{|\LL|} \in \N$.
\end{proposition}

\begin{remark}
	Ce résultat est une conséquence directe du théorème de Lagrange en théorie des groupes.
\end{remark}


\begin{definition}
	On appelle \textbf{minimum d'un réseau} \footnote{on peut aussi le définir comme $\lambda_1(\LL) = \min\{r>0 : |\mathcal{B}(r)\cap\LL|>1\} \in \R_+$} $\LL$ la quantité	
	\[
	\lambda_1(\LL) = \min_{\substack{v \in \LL \\ v \neq 0}} \|v\|\]
\end{definition}

Plus généralement, pour $k \in \{1, \ldots, n\}$, on pose $\lambda_k(\LL)$ le plus petit réel $r$ tel qu'il existe $k$ vecteurs $\R$-linéairement indépendants dans $\LL$ de norme au plus $r$.

\begin{remark}
	$\lambda_1(\LL)$ correspond à la distance minimale entre deux points quelconques de $ \LL $.
\end{remark}

\begin{example}
	Soit $\LL$ le réseau de la figure ~\ref{fig:bases_lattice}. On a $|\LL| = 2$ et $\lambda_1( \LL) = \lambda_2( \LL) = \sqrt{2}$.
\end{example}

\begin{theoreme}[Premier théorème de Minkowski]
	Pour tout $n \in \N^*$, il existe une constante $C_n > 0$ telle que pour tout réseau $\LL$ de $\R^n$, on a :
	
	\[
	\lambda_1(\LL) \leq C_n |\LL|^{1/n}
	\]
	

\end{theoreme}
	On peut prendre cette constante égale à $C_n=(2/\sqrt{\pi}) \Gamma(n/2+1)^{1/n}$. \footnote{La fonction \(\Gamma\), appelée fonction gamma, généralise la notion de factorielle aux nombres réels (et complexes). Elle est définie, pour tout \(z \in \mathbb{C}\), par la formule suivante : \[
    \Gamma(z) = \int_0^{+\infty} t^{z-1} e^{-t} \, dt.
    \]}
On appelle \textbf{constante de Hermite-Minkowski} le carré de la constante optimale possible pour cette inégalité, noté $\gamma _n$, en particulier, on a $\gamma _n \leq C_n$.
En développant, on a 
\[
\gamma_n = \sup_{\dim(\LL) = n} \gamma (\LL) , \quad  \text{ où }\gamma (\LL) := \frac{\lambda _1 (\LL) ^2}{\det (\LL) ^{2/n}}
\]

\begin{proposition}
    \[
    \gamma_n = 4 \left(\frac{ \Delta _n}{V_n}\right)^{2/n} \quad \forall n \in \N^*
    \]
    où \( V_n = \frac{\pi ^{n/2}}{\Gamma(\frac{n}{2} + 1)} \) et \( \Delta_n \) est la densité d'un empilement compact \footnote{Un empilement compact d'une collection d'objets est un agencement de ces objets de telle sorte qu'ils occupent le moins d'espace possible.} de densité maximum d'hypersphères.
\end{proposition} % expliquer c'est quoi un empilement comapct

On ne connaît pas la valeur exacte de \( \gamma_n \) pour tout \( n \). Seuls les valeurs dans le tableau suivant sont connues de manières exactes.\footnote{
    \textbf{En dimension} $n = 1$, n'importe quel réseau de dimension 1 atteint cette borne.  
    \textbf{En dimension} $n = 2$, le réseau optimal est celui des entiers d’Eisenstein, également appelé réseau en nid d’abeille.  
    \textbf{En dimension} $n = 3$, le réseau optimal est présenté dans \parencite{SP3}, mais sa démonstration  nécessite environ 130 pages de minimisation de fonctions analytiques.  
    Des avancées majeures ont été obtenues pour les \textbf{dimensions} $n = 8$ et $n = 24$ grâce à la mathématicienne ukrainienne Maryna Viazovska, qui a démontré l’optimalité du réseau $E_8$ \parencite{SP8} et, en collaboration, celle du réseau de Leech \parencite{SP24}.
}

\begin{figure}[h]
    \centering
    \begin{tabular}{|c|c|c|c|c|c|c|c|c|c|}
        \hline
        $n$ & $1$ & $2$ & $3$ & $4$ & $5$ & $6$ & $7$ & $8$ & $24$ \\
        \hline
        $\gamma_n^n$ & $1$ & $\frac{4}{3}$ & $2$ & $4$ & $8$ & $\frac{64}{3}$ & $64$ & $256$ & $4^{24}$ \\
        \hline
    \end{tabular}

    \caption{Valeurs connues de $ \gamma _n$}
\end{figure}

On sait que \( ( \gamma _n)_n \) est une suite d'ordre de croissance linéaire mais on ne sait pas si elle est croissante.

\begin{example}
	Le \textbf{réseau d'Eisenstein}, ou \textbf{réseau en nid d'abeille}, est un réseau de \(\R^2\) de rang $2$, engendré par la base 
    \(
    \begin{pmatrix}
		1 & \frac{1 + \sqrt{3}}{2} \\
		1 & \frac{1 - \sqrt{3}}{2}
	\end{pmatrix} \). 
	On a \(|\LL| = \sqrt{3},  \lambda_1 = \sqrt{2}, \gamma(\LL) = \frac{2}{\sqrt{3}}\).
    En traçant des sphères de centre les points du réseaux et de rayon \(\sqrt{2}\), on obtient l'empilement compact le plus dense en dimension $2$.
	
	\begin{figure}[h]
		\centering
		\includegraphics[scale=0.4]{images/hexagonal_lattice.png}
		\caption{Empilement hexagonal dans \(\R^2\)}
	\end{figure}
\end{example}

\vspace{-1cm}

\begin{proposition}
	Soit $\LL \subset \R^n$ un réseau de base $(\bb_i)_{1 \leq i \leq n}$ et sa base de Gram-Schmidt associée $(\bb_i^*)_{1 \leq i \leq n}$.
	Alors pour tout $\bv \in \LL \backslash \{\mathbf{0}\}$, on a $$\lambda_1(\LL) \geq \|\bv\| \geq \min_{1 \le i \le n} \|\bb_i^*\|$$ 

\end{proposition}

Tous ces résultats et inégalités vont nous donner des critères de mesure de la qualité de réduction d'une base dans le chapitre sur la réduction de réseaux euclidiens.

\section{Réseaux définis par relations plutôt que par générateurs}

\begin{definition}
    Soit $\LL \subset \R^n$ un réseau de base $(\bb_i)_{1 \leq i \leq n}$. Le \textbf{dual} de \( \LL \) est défini par
    \[
    \LL^\vee:=\{ \bx \in \R^n ~|~ \forall \by \in \LL, \langle \bx, \by \rangle \in \Z\}.
    \]
\end{definition}

\begin{proposition}
    On a $\LL = \LL(\bb_1, \ldots, \bb_m)$ si et seulement si $\LL^\vee= \LL(\bb_1^\vee, \ldots, \bb_n^\vee)$,
    
    où \( \bb_i^\vee \) vérifie \( \left \langle \bb_i^\vee, \bb_i \right \rangle = \delta_{i,j}\). \footnote{ \( \delta_{i,j} \) est le symbole de Kronecker, il vaut \( 1 \) si \( i = j \) et \( 0 \) sinon.}
\end{proposition}

\begin{proposition}
    $\LL^\vee$ est un réseau de base : 
    
    \begin{itemize}
        \item [$\bullet$] $(B^t)^{-1}$ lorsque le réseau est de rang plein.
        \item[$\bullet$] $ B(B^t B)^{-1} $ lorsque le réseau n'est pas de rang plein.
    \end{itemize}
\end{proposition}

\begin{proposition} De la proposition précédente découle les propriétés suivantes :
    \begin{itemize}
        \item[$\bullet$] $\rang(\LL) = \rang(\LL^\vee)$.
        \item[$\bullet$] $|\LL^\vee| = |\LL|^{-1}$.
        \item[$\bullet$] \( (\LL ^\vee)^\vee = \LL \)
    \end{itemize}
\end{proposition}

\begin{definition}
    On dit que \( \LL \) est \textbf{auto-dual} si \( \LL = \LL^\vee \).
\end{definition}

\begin{proposition} On a les propriétés suivantes :
    \begin{itemize}
        \item[$\bullet$] $(a \LL)^\vee = \frac{1}{a} \LL^\vee$ pour tout $a \in \R^*$.
        \item[$\bullet$] $(\Z u)^\vee= \frac{1}{ \|u\|} \Z u$ pour tout $u \in \R^m \backslash \{ \mathbf{0} \}$.
    \end{itemize}
\end{proposition}

\begin{example}
    Le réseau dual de $\Z^n$ est $\Z^n$ et est donc auto-dual.
\end{example}

\begin{figure}[h]
    \centering
    \begin{subfigure}[b]{0.25\textwidth}
        \centering
        \includegraphics[width=\textwidth]{images/lattice_0_5.png}
        \caption{\( 2 \Z^2 \)}
    \end{subfigure}
    \hspace{0.05\textwidth}
    \begin{subfigure}[b]{0.25\textwidth}
        \centering
        \includegraphics[width=\textwidth]{images/lattice_1_5.png}
        \caption{\( \frac12 \Z^2  \)}
    \end{subfigure}
    \caption{Un réseau et son dual.}
    \label{fig:dual}
\end{figure}


\begin{proposition}
    Soit $\LL_1$, $\LL_2$ des réseaux, alors
    $$(\LL_1 \oplus \LL_2)^\vee = \LL_1^\vee \oplus \LL_2^\vee$$
\end{proposition}

\begin{lemma}
    Soit $\LL$ un réseau de dimension $n$. On a
    \begin{itemize}
        \item[$\bullet$] $\lambda_1(\LL) \cdot \lambda_1(\LL^\vee) \leq n$,
        \item[$\bullet$] $\lambda_1(\LL) \cdot \lambda_n(\LL^\vee) \geq 1$.
    \end{itemize}
\end{lemma}

\parencite{Banaszczyk1993} a démontré une relation encore plus forte entre les minima d'un réseau et ceux de son dual, connue sous le nom de théorème de transfert.

\begin{theoreme}[Théorème de transfert]
    Soit $\LL$ un réseau de dimension $n$. On a
    \[
    1 \leq \lambda_1(\LL) \cdot \lambda_n(\LL^\vee) \leq n.
    \]
\end{theoreme}

\begin{definition}
    On définit le réseau euclidien \( A_n \subset \R^{n+1} \) par :
    \[A_n = \left\{ (x_1, \dots, x_{n+1}) \in \Z^{n+1} \;\middle|\; \sum_{i=1}^{n+1} x_i = 0 \right\}\]
\end{definition}

\begin{example}
    On a :
    \begin{align*}
        A_0 &= \{ 0 \} \subset \R, \\
        A_1 &= \{ (x, -x) \in \Z^2 \}, \text{ donc } A_1 \text{ est de rang } 1\\
        A_2 &= \{ (x, y, z) \in \Z^3 \mid x + y + z = 0 \}, \\
        &= \langle a_1, a_2 \rangle \quad \text{où } 
        a_1 = (1, -1, 0), \quad a_2 = (0, 1, -1), \\
        &\text{donc } A_2 \text{ est de rang } 2.
    \end{align*}
\end{example}

\begin{proposition}
    Pour tout $n \in \N$, \( A_n \) a pour matrice génératrice :
    
    \[
    B_n := 
    \begin{pmatrix}
        1 & 0 & 0 & \cdots & 0 \\
        -1 & 1 & 0 & \cdots & 0 \\
        0 & -1 & 1 & \cdots & 0 \\
        \vdots & \vdots & \ddots & \ddots & \vdots \\
        0 & 0 & \cdots & -1 & 1 \\
    \end{pmatrix} 
    \in M_n(\Z)
    \]
\end{proposition}

\begin{proposition}
    \( A_n \) est un réseau euclidien de rang \( n \), pour tout \( n \in \N \).
\end{proposition}

Il existe une classe particulière de réseaux qui joue un rôle important en cryptographie.
\begin{definition}[Réseau $q$-aire]
    Un réseau $\LL$ est un réseau $q$-aire si 
    \[
    q\mathbb{Z}^n \subseteq \LL \subseteq \mathbb{Z}^n.
    \]
\end{definition}

 
Étant donnée une matrice $\mathbf{A} \in \Z_q^{m \times n}$ pour certains entiers $n, m, q \in \N$, on peut définir deux réseaux :

\[
\LL_q(\mathbf{A}) = \left\{ \by \in \Z^m : \by = \mathbf{A}\mathbf{s} \bmod q \text{ pour un certain } \mathbf{s} \in \Z^n \right\}
\]

\[
\LL_q^\perp(\mathbf{A}^T) = \left\{ \by \in \Z^m : \mathbf{A}^T \by = 0 \bmod q \right\}
\]

Les deux réseaux sont de dimension $m$. Le premier est engendré par les lignes de $\mathbf{A}$ et a pour déterminant $q^{m-n}$, tandis que le second contient tous les vecteurs orthogonaux aux lignes de $\mathbf{A}$ et a pour déterminant $q^n$.

De plus, ils sont liés par la dualité des réseaux, c’est-à-dire :
\[
\LL_q^\perp(\mathbf{A}^T) = q \cdot \LL_q(\mathbf{A})^\vee \quad \text{et} \quad \LL_q(\mathbf{A}) = q \cdot \LL_q^\perp(\mathbf{A}^T)^\vee.
\]

\section{Quelques problèmes algorithmiques liés aux réseaux euclidiens}

Cette section s’inspire largement des travaux de \textcite{Boudgoust2023}, auxquels le lecteur intéressé pourra se référer pour un traitement plus approfondi. On définira deux problèmes algorithmiques important sur les réseaux euclidiens. Il y en a beaucoup plus, \parencite{stephens-davidowitz_latticeproblems} donne un aperçu des réductions entre ces problèmes.

\subsection{Des problèmes faciles}

Certains problèmes liés aux réseaux euclidiens sont relativement simples à résoudre, notamment la vérification de l’appartenance d’un vecteur à un réseau donné, ou encore la décision de l’égalité de deux bases de réseaux. Le lecteur intéressé pourra s’exercer en tentant de résoudre ces problèmes.

\begin{problem}[\textbf{Adhésion}]
    Étant donné une base $B$ d’un réseau $\LL$, et $\bv \in \R^n$, décider si $\bv \in \LL$.
\end{problem}

\begin{problem}[\textbf{Équivalence}]
    Étant donné deux bases $B$ et $B'$, décider si $\LL (B) = \LL (B')$,
\end{problem}

\subsection{Le problème du vecteur le plus court}


Considérons le problème suivant, paramétré par la dimension $n$ du réseau :


\begin{problem}[\textbf{Shortest Vector Problem (SVP)}, \textbf{NP-complet}, \parencite{Ajtai1996}.]
    Étant donné une base $B$ d’un réseau $\LL$, trouver un vecteur $\bv \neq \mathbf{0}$ tel que $\|\bv\|_2 = \lambda_1(\LL)$.
\end{problem}

On ne connaît que des algorithmes demandant au moins un nombre exponentiel d'opérations pour résoudre ce problème, même en utilisant des algorithmes quantiques. Les algorithmes de type \textbf{énumération} \footnote{ils énumèrent tous les vecteurs du réseau qui sont dans une certaine boule bien choisie, en pratique ils sont utilisés jusqu'aux dimensions $n \approx 80$. On peut leur ajouter des optimisations et des heuristiques.} et les algorithmes de type \textbf{crible} \footnote{on génère deux listes d'éléments du réseau, puis on construit la liste de toutes les différences entre les éléments des deux listes. On espère obtenir des vecteurs plus court. On recommence le procédé. Le temps d'exécution est en $2^{\OO(n)}$.} se démarquent pour ce problème. Le calcul d'un plus court vecteur dans un réseau euclidien de $\R^n$ est en général un problème difficile qui sert de fondation à de nombreuses primitives cryptographiques. On s’intéresse souvent à la version approximative :


\begin{problem}[\textbf{SIVP$_\gamma$}, où $\gamma > 0$]
    Étant donné une base $B$ du réseau $\LL$, trouver des vecteur $\bv_1, \ldots \bv_n \neq \mathbf{0}$ linéairement indépendants tel que $\|\bv_i\|_2  \leq \gamma \cdot \lambda_i(\LL)$ pour tout \( i \).
\end{problem}

\begin{figure}[H]
    \centering
    \includegraphics[width=0.25\textwidth]{images/lattice_0_8.png}
    \caption{ Une instance de SIVP. }
    \label{fig:sivp}
\end{figure}

L’état des connaissances actuelles est le suivant :

\begin{itemize}
	\item[$\bullet$] Pour $\gamma = \OO(1)$, le problème est prouvé \textbf{NP-complet}, \parencite{Ajtai1996}. %pas d'algo exp ?
	\item[$\bullet$] Pour $\gamma = \text{poly}(n)$, il existe des algorithmes en \textbf{temps exponentiel}.
	\item[$\bullet$] Pour $\gamma = 2^{\OO(n)}$, l’algorithme \textbf{LLL} \parencite{Lenstra1982} permet de le résoudre en \textbf{temps polynomial}.
\end{itemize}

\begin{problem}[\textbf{GapSVP$_\gamma$}, où $\gamma > 0$]
    Étant donné une base $B$ du réseau $\LL$, et $r \in \R_+^*$. Décider si \( \lambda_1(\LL) \leq r \) (instance positive) ou \( \lambda_1(\LL) > \gamma \cdot r \) (instance négative).
\end{problem}

\begin{theoreme}[\cite{Banaszczyk1993},\cite{stephens-davidowitz_latticeproblems}].
    
    $\mathrm{SVP}_\gamma$ n'est pas plus simple que $\mathrm{GapSVP}_\gamma$.
\end{theoreme}

\begin{conjecture}
    Il n'existe aucun algorithme classique ou quantique en temps polynomial qui approxime les problèmes de réseaux 
    $\mathrm{SVP}_\gamma$, $\mathrm{GapSVP}_\gamma$ ou à un facteur polynomial près $\gamma$ (pour tous les réseaux d'entrée possibles).
\end{conjecture}

\subsection{Le problème du vecteur le plus proche}
Un autre problème important concerne la recherche de vecteurs proches d'une cible dans un réseau.

\begin{problem}[\textbf{Closest Vector Problem (CVP)}, \textbf{NP-complet}, \parencite{Ajtai1996}.]
    Étant donnés $t \in \R^n$, un réseau $\LL(B)$, trouver $\bv \in \LL$ tel que 
    $$\displaystyle \|t - \bv\|_2  = d(t, \LL) := \min_{ v \in \LL } \{ \|t - \bv\|_2  \}.$$
\end{problem}


Le problème $\mathrm{CVP}$ est en général difficile pour un réseau arbitraire. Cependant, pour certaines familles spécifiques de réseaux, comme $\Z^n$, des algorithmes en temps polynomial sont connus. La qualité de la base choisie joue un rôle crucial dans la résolution du problème. De même, on peut considérer une version approximative :


\begin{problem}[\textbf{CVP$_\gamma$}, $\gamma > 0$]
    Étant donnés $t \in \R^n$, un réseau $\LL(B)$, trouver $\bv \in \LL$ tel que 
    \(
    \|t - \bv\|_2  \leq \gamma \cdot d(t, \LL).
    \)
\end{problem}

\begin{problem}[\textbf{GapCVP$_\gamma$}, $\gamma > 0$]
    Étant donné \( r \in \R_+^* \), $t \in \R^n$, un réseau $\LL(B)$.
    Décider si il existe \( \bv \in \LL \) tel que \( \|t - \bv\|_2  \leq r \) (instance positive) ou \( \|t - \bv\|_2  > \gamma \cdot r \) (instance négative) 
\end{problem}

\begin{theoreme}[\cite{Goldreich1999}].
    
    $\mathrm{GapSVP}_\gamma$ se réduit à $\mathrm{GapCVP}_\gamma$ en temps polynomial.
\end{theoreme}


\begin{theoreme}
	Il existe un algorithme qui résout \textbf{CVP$_{\exp(n)}$} en temps polynomial via l'algorithme \textbf{LLL}.
\end{theoreme}

L'efficacité des algorithmes dépend grandement de la qualité de la base du réseau euclidien choisie. Le chapitre sur la réduction abordera des techniques pour améliorer la base, via l'algorithme \textbf{LLL}.\footnote{En dimension fixée, résoudre \textbf{exactement} le problème \textbf{SVP} pour la norme $\|\cdot\|_\infty$ fournit en fait une $\gamma$-approximation (avec $\gamma$ dépendant de la dimension) pour le problème \textbf{SVP} dans la norme $\|\cdot\|_2$. Il existe notamment une constante $C$ telle que $\|v\|_2 \leq C \|v\|_\infty$ pour tout $v \in \R^n$. Ainsi, un vecteur minimisant $\|v\|_\infty$ donne un vecteur $\sqrt{n}$-proche du vecteur réellement le plus court en norme euclidienne.}
                        % Réseaux euclidiens : définitions fondamentales, exemples, problèmes algorithmiques (SVP, CVP, etc.) %
%               
\chapter{Réduction de réseaux euclidiens}                                                                                                                                     %
%%%%%%%%%%%%%%%%%%%%%%%%%%%%%%%%%%%%%%%%%%%%%%%%%%%%%%%%%%%%%%%%%%%%%%%%%%%%%%%%%%%%%%%%%%%%%%%%%%%%%%%%%%%%%%%%%%%%%%%%%%%%%%%%%%%%%%%%%%%%%%%%%%%%%%%%%%%%%%%
%                                                                                                                                                             %
%                                        ▗▄▄▖▗▄▄▄▖ ▗▄▄▖▗▄▄▄▖▗▄▄▄▖ ▗▄▖ ▗▖  ▗▖    ▗▄▄▄▖▗▄▄▄▖        ▗▄▄▄▖▗▄▄▄▖                                                  %
%                                       ▐▌   ▐▌   ▐▌     █    █  ▐▌ ▐▌▐▛▚▖▐▌      █    █            █    █                                                    %
%                                        ▝▀▚▖▐▛▀▀▘▐▌     █    █  ▐▌ ▐▌▐▌ ▝▜▌      █    █            █    █                                                    %
%                                       ▗▄▄▞▘▐▙▄▄▖▝▚▄▄▖  █  ▗▄█▄▖▝▚▄▞▘▐▌  ▐▌    ▗▄█▄▖▗▄█▄▖        ▗▄█▄▖▗▄█▄▖                                                  %
%                                                                                                                                                             %
%         ▗▄▄▄▖▗▖ ▗▖ ▗▄▄▖▗▖   ▗▄▄▄▖▗▄▄▄ ▗▄▄▄▖ ▗▄▖ ▗▖  ▗▖    ▗▖    ▗▄▖▗▄▄▄▖▗▄▄▄▖▗▄▄▄▖ ▗▄▄▖▗▄▄▄▖    ▗▄▄▖ ▗▄▄▄▖▗▄▄▄ ▗▖ ▗▖ ▗▄▄▖▗▄▄▄▖▗▄▄▄▖ ▗▄▖ ▗▖  ▗▖              %
%         ▐▌   ▐▌ ▐▌▐▌   ▐▌     █  ▐▌  █▐▌   ▐▌ ▐▌▐▛▚▖▐▌    ▐▌   ▐▌ ▐▌ █    █    █  ▐▌   ▐▌       ▐▌ ▐▌▐▌   ▐▌  █▐▌ ▐▌▐▌     █    █  ▐▌ ▐▌▐▛▚▖▐▌              %
%         ▐▛▀▀▘▐▌ ▐▌▐▌   ▐▌     █  ▐▌  █▐▛▀▀▘▐▛▀▜▌▐▌ ▝▜▌    ▐▌   ▐▛▀▜▌ █    █    █  ▐▌   ▐▛▀▀▘    ▐▛▀▚▖▐▛▀▀▘▐▌  █▐▌ ▐▌▐▌     █    █  ▐▌ ▐▌▐▌ ▝▜▌              %
%         ▐▙▄▄▖▝▚▄▞▘▝▚▄▄▖▐▙▄▄▖▗▄█▄▖▐▙▄▄▀▐▙▄▄▖▐▌ ▐▌▐▌  ▐▌    ▐▙▄▄▖▐▌ ▐▌ █    █  ▗▄█▄▖▝▚▄▄▖▐▙▄▄▖    ▐▌ ▐▌▐▙▄▄▖▐▙▄▄▀▝▚▄▞▘▝▚▄▄▖  █  ▗▄█▄▖▝▚▄▞▘▐▌  ▐▌              %
%                                                                                                                                                             %
%%%%%%%%%%%%%%%%%%%%%%%%%%%%%%%%%%%%%%%%%%%%%%%%%%%%%%%%%%%%%%%%%%%%%%%%%%%%%%%%%%%%%%%%%%%%%%%%%%%%%%%%%%%%%%%%%%%%%%%%%%%%%%%%%%%%%%%%%%%%%%%%%%%%%%%%%%%%%%%
%                                                                                                                                                             %
% Pas de sous-fichiers                                                                                                                                        %
%                                                                                                                                                             %
% Le fichier parent est : lattice_reduction.tex                                                                                                               %
%                                                                                                                                                             %
%%%%%%%%%%%%%%%%%%%%%%%%%%%%%%%%%%%%%%%%%%%%%%%%%%%%%%%%%%%%%%%%%%%%%%%%%%%%%%%%%%%%%%%%%%%%%%%%%%%%%%%%%%%%%%%%%%%%%%%%%%%%%%%%%%%%%%%%%%%%%%%%%%%%%%%%%%%%%%%

\lettrine{D}{ans} ce chapitre, nous nous concentrerons sur un algorithme de réduction de réseaux euclidiens s’exécutant en temps polynomial : l’algorithme $\mathrm{LLL}$, du nom de ses auteurs A. Lenstra, H. Lenstra et L. Lovász. L'algorithme $\mathrm{LLL}$ possède de nombreuses applications, notamment en cryptanalyse de schémas basés sur le problème du sac à dos, en factorisation efficace de polynômes, ou encore dans le calcul rapide de décompositions en forme normale d'Hermite (HNF). Le lecteur pourra consulter \parencite{Havas1998} pour un aperçu plus complet de ses usages.  Nous ne traiterons pas d’autres algorithmes de réductions de réseaux euclidiens comme $\mathrm{BKZ}$, afin de rester dans un cadre plus élémentaire. L’algorithme $\mathrm{LLL}$ repose sur une idée simple mais puissante : il produit une approximation entière de la décomposition de Gram-Schmidt et réorganise les vecteurs de la base pour en améliorer la structure. Un rappel sur le procédé d’orthogonalisation de Gram-Schmidt est dans l’annexe dédiée. Celle-ci ne sera pas rappelée ici afin de préserver la concision du texte. Au cours de mon stage j'ai réalisé une présentation de l'algorithme LLL, ainsi qu'une idée rapide de sa preuve, que vous pourrez trouver ici.

\section{La base réduite}

Dans les problèmes liés aux réseaux euclidiens, une base orthogonale représenterait une base idéale.

\begin{problem}[\textbf{Problème}]
    La base \( B^* \) ( de $\R^n$ ) n'est généralement pas une base du réseau \( \LL(B) \).
\end{problem}

\begin{figure}[H]
    \centering
    \includegraphics[width=0.23\textwidth]{images/lattice_0_9.png}
    \caption{ Base de Gram-Schmidt qui n'est pas une base de \( \LL \). }
    \label{fig:GS}
\end{figure}

Nous cherchons une base de \( \LL \) qui \emph{approxime} la base de Gram--Schmidt aussi fidèlement que possible.

\begin{notation}
    On définit l'entier le plus proche de $x \in \R$ par $\lceil x\rfloor  = \lfloor x+1/2 \rfloor$.
\end{notation}

\begin{proposition}
    On a $|x-\lceil x\rfloor| \leq \frac12$ pour tout $x \in \R$.
\end{proposition}

Dans la suite de cette explication, le lecteur est invité à porter une attention particulière aux * qui dénotent un vecteur de la base de Gram-Schmidt.
L'opération d'orthogonaliation de Gram-Schmidt nous donne :
\begin{equation}
    \mathbf{b}^*_2 \coloneqq \mathbf{b}_2 - \mu_{2,1} \mathbf{b}_1^* \notin \LL
\end{equation}

Mais \( \mathbf{b}_1^* = \mathbf{b}_1 \in \LL \), en prenant \( k \in \Z \), on a
\(
\mathbf{b}_2 - k \mathbf{b}_1 \in \LL
\), 
on pourrait donc réaliser l'opération
\[
\mathbf{b}_2 \coloneqq \mathbf{b}_2 - k \mathbf{b}_1 \in \LL
\]
On peut donc choisir \( k \in \Z \) qui minimise \( \left \langle \bb_2, \bb_1 \right \rangle \), ce qui est réalisé par \( k \coloneqq \nint{\mu_{i,j}}\). On en déduit donc l'opération 
\begin{equation}
    \mathbf{b}_2 \coloneqq  \mathbf{b}_2 - \nint{\mu_{2,1}} \mathbf{b}_1 \in \LL
\end{equation}

On peut donc étendre ce procédé par récurrence pour construire la nouvelle famille \( (\bb_i)_{1 \leq i \leq n} \).

En refaisant les calculs on peut montrer que la base de Gram-Schmidt associée est inchangée, mais que les nouveaux coefficients \( |\mu_{i,j}|<\frac12\) d'après la proposition 2.1.

\begin{figure}[H]
      \centering
      \includegraphics[width=0.23\textwidth]{images/lattice_1_9.png}
      \caption{ Base de \( \LL \) proche de la base Gram-Schmidt . }
      \label{fig:GS}
\end{figure}

\begin{definition}
	Soit \( (\bb_i)_{1 \leq i \leq n} \) une base d'un réseau et \( U \) la matrice triangulaire supérieure telle que \( B = UB^* \) .
    est dite \textbf{propre} \footnote{Une base propre est aussi connue sous le nom de base size-réduite dans la littérature.} si 
	
	\begin{equation}
		\displaystyle\max_{1 \leq i < j \leq n} |\mu_{i,j}| \leq \frac{1}{2}.
	\end{equation}
\end{definition}

\begin{counterexample}
    Bien que la proprification impose une certaine contrainte sur les coefficients de projection, elle ne garantit pas à elle seule que les vecteurs de la base soient presque orthogonaux.
    \begin{figure}[H]
        \centering
        \includegraphics[width=0.23\textwidth]{images/lattice_0_10.png}
        \caption{ Base propre mais peu orthogonale }
        \label{fig:nosufficient}
    \end{figure}
\end{counterexample}

Idéalement, nous souhaiterions trouver une base \( (\bb_i)_{1 \leq i \leq n} \) du réseau \( \LL \) telle que :

\[
\| \bb_1 \| = \lambda_1(\LL), \quad
\| \bb_2 \| = \lambda_2(\LL), \quad \ldots, \quad
\| \bb_n \| = \lambda_n(\LL)
\]
Ceci implique \( \| \bb_1 \| \leq \cdots \leq \| \bb_n \|\), mais cela est trop difficile de trouver une telle base car cela reviendrait à résoudre SIVP.

\begin{definition}
    Une base \( (\bb_i)_{1 \leq i \leq m} \) satisfait la \textbf{condition de Lovász} \footnote{Une condition plus générale : \( ( \delta - \mu_{i+1,i}^2 ) \, \|\bb_i^*\|^2 \leq \|\bb_{i+1}^*\|^2 \text{ pour } 1 \leq i \leq n, \text{ où } \delta \in \left] \frac{1}{4}, 1 \right]  \) } si:
    \[
    \|\bb_i^*\|^2 \leq 2\|\bb_{i+1}^*\|^2 \quad \text{ pour tout } 1 \leq i < n
    \]
\end{definition}

\begin{remark}
    On peut interpréter cette condition comme une forme de quasi-croissance des normes \( \|\bb_i^*\| \) : elle n'exige pas que celles-ci soient strictement croissantes, mais impose que toute éventuelle décroissance soit contrôlée, autrement dit, qu'elles ne décroissent pas trop rapidement.
\end{remark}

Dès lors, il est naturel de se demander pourquoi ne pas échanger les vecteurs lorsque cette condition de Lovász n’est pas satisfaite.


\begin{example}
    Si l’on applique cette idée à l’exemple précédent, alors après permutation des vecteurs concernés et une nouvelle phase de réduction, on obtient à nouveau une base améliorée :
    \begin{figure}[H]
        \centering
        \includegraphics[width=0.23\textwidth]{images/lattice_1_10.png}
        \caption{ Nouvelle base size-réduite}
        \label{fig:sufficient}
    \end{figure}
\end{example}


\begin{definition}
	Une base \( \mathcal{B} \) d'un réseau \( \LL (\mathcal{B}) \)  est dite \textbf{LLL-réduite} si
	\begin{itemize}
		\item[$\bullet$] $\mathcal{B}$ est propre.
		\item[$\bullet$] $\mathcal{B}$ satisfait la condition de Lovàsz.
	\end{itemize}
\end{definition}

\begin{remark}
	Chaque vecteur de la base réduite a une norme au moins égale à la moitié de celle du précédent, garantissant ainsi une décroissance modérée.
\end{remark}

\begin{theoreme}
	Soit \( \mathcal{B} \) une base réduite du réseau \( \LL \subseteq \R^n \) et soit \( v \in \LL \setminus \{0\} \). Alors
     \[ \|\bb_1\| \leq 2^{(n-1)/2} \cdot \|\bv\| \]
\end{theoreme}

En particulier, ce résultat s’applique à un vecteur \( \bv \in \LL \) de plus petite norme non nulle, c’est-à-dire un vecteur atteignant \( \lambda_1(\LL) \). On en déduit donc :
\[
\|\bb_1\| \leq 2^{(n-1)/2} \cdot \lambda_1(\LL),
\]
ce qui montre que \( \mathrm{LLL} \) fournit en temps polynomial un vecteur de norme à un facteur \( 2^{(n-1)/2} \) près du plus court vecteur du réseau.
Autrement dit, l’algorithme \( \mathrm{LLL} \) résout approximativement le problème du plus court vecteur (\( \mathrm{SVP} \)) avec un facteur d’approximation \( \gamma = 2^{(n-1)/2} \).
Par extension, en renvoyant les vecteurs de la base réduite, \( \mathrm{LLL} \) permet également de résoudre le problème \( \mathrm{SIVP} \) (\textit{Shortest Independent Vectors Problem}) avec le même facteur d’approximation.

\section{Fonctionnement et exemple}

Nous présentons à présent l’algorithme de Lenstra–Lenstra–Lovász (\( \mathrm{LLL} \)), dans sa forme classique. Originellement \( \mathrm{LLL} \) est apparu dans l'article de 1982 et servait à factoriser des polynômes à coefficients rationnels.

\begin{smallalgo}{LLL}{algo:LLL_MCA}
    \LinesNumbered 
    \DontPrintSemicolon
    \KwIn{Une base \( B = (\bb_1, \ldots, \bb_n) \)}
    \KwOut{Une base réduite \( G = (\bg_1, \ldots, \bg_n) \) de \( B \)}
    
    \For{\( i = 1 \) \KwTo \( n \)}{
        \( \bg_i \leftarrow \bb_i \)\;
    }
    
    \( (B^*, U) \leftarrow \) \textsc{Gram-Schmidt} \( (B) \)\;
    
    \While{\( i \leq n \)}
    {
        \For{\( j = i-1, i-2, \ldots, 1 \)}{
            \( \bg_i \leftarrow \bg_i - \nint{\mu_{i,j}} \, \bg_j \)\;
            Mettre à jour \( B^*, U  \)\;
        }
        
        \If{$i > 1$ \textbf{et} $\|\bg_{i-1}^*\|^2 > 2 \|\bg_{i}^*\|^2$}{
            Échanger \( \bg_{i-1} \) et \( \bg_i \)\;
            Mettre à jour \( B^*, U  \)\;
            \( i \leftarrow i - 1 \)\;
        }
        \Else{
            \( i \leftarrow i + 1 \)\;
        }
    }
    
    \KwRet{\( G = (\bg_1, \ldots, \bg_n) \)}
\end{smallalgo}

\vspace{1cm}
L’algorithme débute par le calcul de la base orthogonalisée de Gram–Schmidt (\textbf{ligne 3}), qui sert de support aux opérations de réduction.  
Le principe général repose sur l’application répétée de deux types d’étapes : des \emph{réductions de taille} (\textbf{lignes 5 à 7}) visant à raccourcir les vecteurs sans sortir du réseau, et des \emph{permutations} de vecteurs (\textbf{ligne 8}) effectuées lorsque la condition de Lovász, qui contrôle la décroissance des normes, n’est pas satisfaite.

L’ensemble de ces opérations est imbriqué dans une boucle \texttt{while} qui se répète tant qu’une condition de progression n’est pas remplie.  
Nous verrons que cette condition de Lovász garantit non seulement une amélioration à chaque étape, mais également la terminaison de l’algorithme en un nombre fini d’itérations.

\begin{example}
	Soit 
    \[
    B =
    \begin{pmatrix}
        1 & 1 & 1 \\
        -1 & 0 & 2 \\
        3 & 5 & 6
    \end{pmatrix}
    \in GL_3(\Z).
    \]
    On a $\displaystyle |\LL (B)| = 9, \quad A = \max_{1 \leq i \leq 3} \| \bb_i \| = 70$
    
    On va essayer d'estimer un plus court vecteur
    L'algorithme \hyperref[algo:LLL_MCA]{\emph{LLL}} commence par calculer la décomposition de Gram-Schmidt de \( B \), pour plus de détails sur le calcul de cette décomposition, ce calcul est effectué dans l'annexe \hyperref[appendix:linear_algebra]{\emph{Rappels d'algèbres linéaire }}.
    
    On obtient la décomposition
    
    \[
    \begin{aligned}
        U &= 
        \begin{pmatrix}
            1 & 0 & 0 \\
            \frac{1}{3} & 1 & 0 \\
            \frac{14}{3} & \frac{13}{14} & 1
        \end{pmatrix}, \quad
        B^* =
        \begin{pmatrix}
            1 & 1 & 1 \\
            -\frac{4}{3} & - \frac{1}{3} & \frac{5}{3} \\
            -\frac{3}{7} & \frac{9}{14} & - \frac{3}{14}
        \end{pmatrix}.
    \end{aligned}
    \] 
    
    Voici un tableau récapitulant les principales étapes de l'algorithme. Le tableau est volontairement détaillé et fourni, le lecteur pourra revenir sur cet exemple pour comprendre ce que sont $d_1$, $d_2$, $D$ ou la signification de $\Gram(G)$.
    
    \setlength{\tabcolsep}{0pt}
    
    \begin{center}
        \begin{tabular}{|c|c|c|c|c|c|} \hline % Premiere ligne
            \( G \)
            &
            \( U \)
            &
            \( G^* \)
            &
            \parbox[c][1cm][c]{1cm}
            {
                \centering
                \shortstack{\( d_1, d_2 \) \\ \( D \)}
            }
            & 
            \(
            \begin{pmatrix}
                \|\bg_1^*\|^2 \\
                \|\bg_2^*\|^2 \\
                \|\bg_3^*\|^2
            \end{pmatrix}
            \)
            &
            \( \Gram(G) \)
            \\
            \hline \tikzmark{l1} % Deuxieme ligne
            \parbox[c][2.2cm][c]{3cm}
            {%
                \[
                \begin{pmatrix}
                    1\quad 1\quad 1 \\
                    -1\quad 0\quad 2\\
                    3\quad 5\quad 6
                \end{pmatrix}
                \] 
            }
            & 
            \(
            \begin{pmatrix}
                1            & 0             & 0 \\
                \frac{1}{3}  & 1             & 0 \\
                \frac{14}{3} & \frac{13}{14} & 1
            \end{pmatrix}
            \) 
            & 
            \( 
            \begin{pmatrix}
                1            & 1            & 1 \\
                -\frac{4}{3} & -\frac{1}{3} & \frac{5}{3} \\
                -\frac{3}{7} & \frac{9}{14} & - \frac{3}{14}
            \end{pmatrix}
            \) 
            & 
            \parbox[c][0.9cm][c]{0.9cm}
            {
                \centering
                \shortstack{\( 3, 14 \) \\ \( 42 \)}
            }
            &
            \( 
            \begin{pmatrix}
                3 \\
                \frac{14}{3} \\
                \frac{9}{14}
            \end{pmatrix}
            \)
            &
            \(
            \begin{pmatrix}
                3 & 1 & 14 \\
                1 & 1 & 9 \\
                14 & 9 & 70
            \end{pmatrix}
            \)
            \\
            \hline \tikzmark{l2} % troisieme ligne
            \parbox[c][2.2cm][c]{3cm}
            {%
            \[
            \begin{pmatrix}
                1\quad 1\quad 1 \\
                -1\quad 0\quad 2\\
                0\quad 1\quad 0
            \end{pmatrix}
            \]
            }
            & 
            \(
            \begin{pmatrix}
                1           & 0             & 0 \\
                \frac{1}{3} & 1             & 0 \\
                \frac{1}{3} & \frac{-1}{14} & 1
            \end{pmatrix}
            \)
            & 
            \( 
            \begin{pmatrix}
                1 & 1 & 1 \\
                -\frac{4}{3} & -\frac{1}{3} & \frac{5}{3} \\
                -\frac{3}{7} & \frac{9}{14} & - \frac{3}{14}
            \end{pmatrix}
            \)
            &
            \parbox[c][0.9cm][c]{0.9cm}
            {
                \centering
                \shortstack{\( 3, 14 \) \\ \( 42 \)}
            }
            &
            \( 
            \begin{pmatrix}
                3 \\
                \frac{14}{3} \\
                \frac{9}{14}
            \end{pmatrix}
            \)
            &
              \(
            \begin{pmatrix}
                3 & 1 & 1 \\
                1 & 5 & 0 \\
                1 & 0 & 1
            \end{pmatrix}
            \)
            \\ \hline \tikzmark{l3}
            
            \parbox[c][1.6cm][c]{2.5cm}
            {%
                \centering
                \[
                \begin{pmatrix}
                    1& 1& 1 \\
                    0& 1& 0 \\
                    -1& 0& 2
                \end{pmatrix}
                \]
            }
            &
            \(
            \begin{pmatrix}
                1           & 0            & 0 \\
                \frac{1}{3} & 1            & 0 \\
                \frac{1}{3} & \frac{-1}{2} & 1
            \end{pmatrix}
            \) 
            & 
            \parbox[c][2.2cm][c]{3cm}
            {%
                
            \(
            \begin{pmatrix}
                1            & 1           & 1 \\
                -\frac{1}{3} & \frac{2}{3} & -\frac{1}{3} \\
                -\frac{3}{2} & 0           & \frac{3}{2}
            \end{pmatrix}
            \)
        }
            &
            \parbox[c][0.9cm][c]{0.9cm}
            {
                \centering
                \shortstack{\( 3, 2 \) \\ \( 6 \)}
            }
            &
            \( 
            \begin{pmatrix}
                3 \\
                \frac{2}{3} \\
                \frac{9}{2}
            \end{pmatrix}
            \)
            &
              \(
            \begin{pmatrix}
                1 & 0 & 0 \\
                0 & 1 & 0 \\
                \frac{1}{3} & \frac{-1}{2} & 1
            \end{pmatrix}
            \)
            \\
            \hline  \tikzmark{l4}
            \parbox[c][2.2cm][c]{3cm}
            {%
                \[
                \begin{pmatrix}
                    0& 1& 0 \\
                    1& 1& 1 \\
                    -1& 0& 2
                \end{pmatrix}
                \]
            }
            & 
            \( 
            \begin{pmatrix}
                1 & 0 & 0 \\
                1 & 1 & 0 \\
                \frac{1}{3} & \frac{-1}{2} & 1
            \end{pmatrix}
            \)
            & 
            \(
            \begin{pmatrix}
                0 & 1 & 0 \\
                1 & 0 & 1 \\
                -\frac{3}{2} & 0 & \frac{3}{2}
            \end{pmatrix}
            \)  
            &
            \parbox[c][0.9cm][c]{0.9cm}
            {
                \centering
                \shortstack{\( 1, 2 \) \\ \( 2 \)}
            }
            &
            \( 
            \begin{pmatrix}
                1 \\
                2 \\
                \frac{9}{2}
            \end{pmatrix}
            \)
            &
              \(
            \begin{pmatrix}
                1 & 0 & 0 \\
                0 & 1 & 0 \\
                \frac{1}{3} & \frac{-1}{2} & 1
            \end{pmatrix}
            \)
            \\
            \hline \tikzmark{l5}
            \parbox[c][2.2cm][c]{3cm}
            {%     
                \[
                \begin{pmatrix}
                    0 & 1 & 0 \\
                    1 & 0 & 1 \\
                    -1& 0 & 2
                \end{pmatrix}
                \]
            }
            & 
            \(
            \begin{pmatrix}
                1 & 0 & 0 \\
                0 & 1 & 0 \\
                \frac{1}{3} & \frac{-1}{2} & 1
            \end{pmatrix}
            \) 
            & 
            \(
            \begin{pmatrix}
                0 & 1 & 0 \\
                1 & 0 & 1 \\
                -\frac{3}{2} & 0 & \frac{3}{2}
            \end{pmatrix}
            \) 
            &
            \parbox[c][0.9cm][c]{0.9cm}
            {
                \centering
                \shortstack{\( 1, 2 \) \\ \( 2 \)}
            }
            &
            \( 
            \begin{pmatrix}
                1 \\
                2 \\
                \frac{9}{2}
            \end{pmatrix}
            \)
            &
            \(
            \begin{pmatrix}
                1 & 0 & 0 \\
                0 & 2 & 1 \\
                0 & 1 & 5
            \end{pmatrix}
            \)
            \\ \hline 
        \end{tabular}
    \end{center}

\vspace{1cm}

    On obtient la base  LLL réduite :

    \[
    \bg_{reduced}=
    \begin{pmatrix}
        0 & 1 & 0 \\
        1 & 0 & 1 \\
        -1 & 0 & 2
    \end{pmatrix}
    \in M_3(\Z) 
    \]



\end{example}


\begin{tikzpicture}[remember picture, overlay]
    \draw[->, thick, bend right=80]
    ([xshift=0em]pic cs:l1) to
    node[midway, left]{\small \shortstack{proprification\\ $\bg_3$}}
    ([xshift=0em, yshift=1em]pic cs:l2);
    
    \draw[->, thick, bend right=80]
    ([xshift=0em]pic cs:l2) to
    node[midway, left]{\small \shortstack{Lovasz\\ $\bg_2 \leftrightarrow \bg_3$}}
    ([xshift=0em, yshift=1em]pic cs:l3);
    
    \draw[->, thick, bend right=80]
    ([xshift=0em]pic cs:l3) to
    node[midway, left]{\small \shortstack{Lovasz\\ $\bg_1 \leftrightarrow \bg_2$}}
    ([xshift=0em, yshift=1em]pic cs:l4);
    
    \draw[->, thick, bend right=80]
    ([xshift=0em]pic cs:l4) to
    node[midway, left]{\small \shortstack{proprification\\ $\bg_2$}}
    ([xshift=0em, yshift=1em]pic cs:l5);
\end{tikzpicture}

\begin{remark}
    \leavevmode\vspace{0.5\baselineskip}
    \begin{itemize}
        \item[$\bullet$] La valeur $d_3$ ne nous intéresse pas car il s'agit d'un invariant, 
        \item[$\bullet$] \(Gram(G)\) se rapproche petit a petit d'une matrice diagonale.     
    \end{itemize}
\end{remark}

\chapter{Correction, terminaison et complexité de LLL}

Le lecteur pourra choisir de passer ce chapitre en première lecture. Les preuves et les résultats secondaires (lemmes et démonstrations annexes) sont regroupés en annexe afin d'alléger la lecture. Dans tous les cas, il suffira de garder à l'esprit que la terminaison de l'algorithme LLL repose sur la décroissance d'une quantité notée $D$, dont l'étude sera essentielle dans la suite. Le lecteur pourra également, s'il le souhaite, consulter directement les résultats relatifs à la complexité algorithmique de LLL.


\section{Correction}
\begin{theoreme}[Correction]
    L'algorithme \hyperref[algo:LLL_MCA]{\emph{LLL }} calcule une base réduite de \( \LL \).
\end{theoreme}

\vspace{0.2cm}
\begin{smallalgo}{Proprification de \( \bg_{i} \)}{step:P}
    \For{$j = i{-}1, i{-}2, \dots, 1$}{
        \hyperref[step:PP]{\emph{Proprification partielle de \( \bg_i \)}}\;
    }
\end{smallalgo}

\begin{lemma}
    \hyperref[step:P]{\emph{Proprification de $\bg_i$}} ne change pas $G^*$ et à la fin on a 
    \[
    |\mu_{i,l}| \leq \frac{1}{2} \text{ pour } 1 \leq l < i. 
    \]
\end{lemma}

On étudie maintenant l'étape de réduction, l'idée consiste à réorganiser les vecteurs afin de garantir une progression quantifiable, qui assurera la terminaison de LLL.

\vspace{0.2cm}
\begin{smallalgo}{Réduction de \( \bg_{i-1}, \bg_i \)}{step:R}
    \If{\( i > 1 \) \textbf{et} \( \norm{\bg_{i-1}^*}^2 > 2 \norm{\bg_{i}^*}^2 \)}
    {
        Échanger \( \bg_{i-1} \) et \( \bg_i \)\;
        
        Mettre à jour \( (B^*, U)  \)\;
        
        \( i \leftarrow i - 1 \)\;
    }
    \Else
    {
        \( i \leftarrow i + 1 \)\;
    }
\end{smallalgo}

\begin{lemma}
    Supposons que \( \bg_{i-1} \) et \( \bg_i \) sont échangés à l'étape \hyperref[step:R]{\emph{Réduction de $\bg_{i-1}, \bg_{i}$}}. 
    
    On note \( h_k \) les vecteurs après échange et \( h_k^* \) leur base orthogonale de Gram-Schmidt.
    
    Alors
    
    \begin{enumerate}
        \item \( \mathbf{h}_k^* = \bg_k^* \quad \text{pour tout } k \in \{1, \dots, n\} \setminus \{i-1, i\} \),
        \item \( \| \mathbf{h}_{i-1}^* \|^2 < \dfrac{3}{4} \| \bg_{i-1}^* \|^2 \),
        \item \( \| \mathbf{h}_i^* \| \leq \| \bg_{i-1}^* \| \).
    \end{enumerate}
\end{lemma}

\vspace{0.2cm}
\begin{smallalgo}{LLL}{step:LLL}
    \While{\( i \leq n \)}
    {
        \textit{Proprification de \( \bg_i \)}\;
        \textit{Réduction de \( \bg_{i-1}, \bg_i \)}\;
    }
\end{smallalgo}

\begin{lemma}
    Au début de chaque itération de la boucle à l'étape \hyperref[step:LLL]{\emph{LLL}}, les invariants suivants sont vérifiés :
    \[
    |\mu_{k,l}| \leq \frac{1}{2} \quad \text{pour } 1 \leq l < k < i, \qquad \|\bg_{k-1}^*\|^2 \leq 2 \|\bg_k^*\|^2 \quad \text{pour } 1 < k < i.
    \]
\end{lemma}

\section{Terminaison et complexité}
\begin{theoreme}[Terminaison et complexité]
    On pose \( \displaystyle A = \max_{1 \leq i \leq n} \| \bg_i \| \). L'algorithme \hyperref[algo:LLL_MCA]{\emph{LLL}} termine et utilise \( \OO(n^4 \log A) \) opérations arithmétiques sur des entiers.
\end{theoreme}

La difficulté est de montrer que la boucle Tant que ne va pas s'exécuter indéfiniment.

\begin{lemma}
    \leavevmode\vspace{0.3\baselineskip}
    \begin{enumerate}
        \item Orthogonalisation de Gram-Schmidt nécessite \( \OO(n^3) \) opérations dans \( \Z \).
        
        \item \hyperref[step:P]{\emph{Proprification de \( \bg_i \)}} nécessite \( \OO(n^2) \) opérations dans \( \Z \).
        
        \item \hyperref[step:R]{\emph{Réduction de \( \bg_{i-1}, \bg_{i} \)}} nécessite \( \OO(n) \) opérations dans \( \Z \).
    \end{enumerate}
\end{lemma}

Il reste à borner le nombre d'itérations de la boucle Tant que à l'étape \hyperref[step:LLL]{\emph{LLL}}. 

Pour tout \( 1 \leq k \leq n \), on pose
\[
\bg_k = 
\begin{pmatrix}
    \bg_1 \\
    \vdots \\
    \bg_k
\end{pmatrix}
\in \Z^{k \x n}
, \quad d_0=1, \quad d_k = \det(\bg_k \cdot \bg_k^T) \in \Z.
\]

\begin{lemma}
    Pour tout \( 1 \leq k \leq n \), on a :
    \[
    d_k = \prod_{1 \leq l \leq k} \| \bg_l^* \|^2 > 0.
    \]
\end{lemma}

\begin{lemma}
    \leavevmode\vspace{0.5\baselineskip}
    \begin{enumerate}
        \item \hyperref[step:P]{\emph{Proprification de \( \bg_i \)}} ne change pas \( d_k \)  pour tout \( 1 \leq k \leq n \).
        
        \item Si \( \bg_{i-1} \) et \( \bg_i \) sont échangés à l’étape \hyperref[step:P]{\emph{Réduction de \( \bg_{i-1}, \bg_{i} \)}}, et si \( d_k^* \) désigne la nouvelle valeur de \( d_k \), alors :
        \[
        d_k^* = d_k \quad \text{pour tout } k \neq i - 1, \quad \text{et} \quad d_{i-1}^* \leq \frac{3}{4} d_{i-1}.
        \]
    \end{enumerate}
\end{lemma}

\begin{proof}
    \begin{enumerate}
        \item D'après le lemme 2.2 \hyperref[step:P]{\emph{Proprification de \( \bg_i \)}} ne modifie pas \( \bg_k^* \) et donc ne modifie pas \( d_k \).
        
        \item Pour \( k \neq i-1\), une exécution de \hyperref[step:P]{\emph{Réduction de \( \bg_{i-1}, \bg_{i} \)}} multiplie \( \bg_k \) par une matrice de permutation, donc \( d_k^* = d_k \)
        
        De plus, on a
        \[
        d_{i-1} \eqjust{2.6} \prod_{1 \leq l \leq i-1} \| \bg_l^* \|^2 \leqjust{2.3} \frac{3}{4} \prod_{1 \leq l \leq i-1} \| \mathbf{h}_l^* \|^2 \eqjust{2.6} \frac{3}{4} d_{i-1}^* 
        \] 
    \end{enumerate}
\end{proof}
On pose
\[
D = \prod_{1 \leq k < n} d_k, \quad \displaystyle A = \max_{1 \leq i \leq n} \| \bg_i \|
\]

On désigne \( D_0 \) désigne la valeur de \( D \) au début de l'algorithme, on a \( 1 \leq D \in \Z \) et 

\[
\begin{aligned}
    D_0 &= \|\bg^*_1\|^{2(n-1)} \|\bg^*_2\|^{2(n-2)} \cdots \|\bg^*_{n-1}\|^2 \\
    &\leq \|\bg_1\|^{2(n-1)} \|\bg_2\|^{2(n-2)} \cdots \|\bg_{n-1}\|^2\\
    &\leq A^{n(n-1)}
\end{aligned} 
\] 


Puisque \( g^*_i \) est une projection de \( g_i \) pour tout \( i \). 

\begin{lemma}
    \leavevmode\vspace{0.5\baselineskip}
    \begin{enumerate}
        \item \hyperref[step:P]{\emph{Proprification de \( \bg_i \)}} ne modifie pas \( D \).
        \item \( D \) diminue d’au moins un facteur \( 3/4 \) si un échange a lieu dans \hyperref[step:R]{\emph{Réduction de \( \bg_{i-1}, \bg_{i} \)}}
    \end{enumerate}
\end{lemma}

\begin{proof}
    \begin{enumerate}
        \item D'après le lemme 2.7 \hyperref[step:P]{\emph{Proprification de \( \bg_i \)}} ne modifie pas \( d_k \) et donc ne modifie pas \( D \).
        
        \item Si \( \bg_{i-1} \) et \( \bg_i \) sont échangés lors de l'exécution de \hyperref[step:R]{\emph{Réduction de \( \bg_{i-1}, \bg_{i} \)}}, en notant \( D^* \) la nouvelle valeur de \( D \), alors d'après le lemme 2.7
        
        \[
        d_k^* = d_k, \quad d_{i-1}^* \leq \frac{3}{4} d_{i-1} \text{ donc } D^* \leq \frac{3}{4} D.
        \]
    \end{enumerate}
\end{proof}

À tout moment de l’algorithme, soit \( e \in \N \) le nombre d’échanges effectués jusqu’à présent, et \( e^* \) le nombre de fois où la branche alternative (le \textit{else}) dans \hyperref[step:R]{\emph{Réduction de \( \bg_{i-1}, \bg_{i} \)}} a été prise.

\begin{lemma}
    On a
    \[
    e \leq \log_{4/3} D_0 \in \OO(n^2 \log A)
    \]
\end{lemma}

\begin{proof}
    Soit \( D_e \) la valeur de \( D \) après \( e \) échanges.
    
    On doit avoir
    \[
    1 \;\le\; D_e \; \le\;\left(\frac34\right)^{e} D_0 \le\; \left(\frac34\right)^{e}A^{\,n(n-1)}.
    \]
    
    En appliquant \( lo\bg_{3/4} \), aux extrémités de l'inégalité.
    
    \[
    0=\log_{3/4}(1) \;\ge\; e + \log_{3/4}(A^{\,n(n-1)})=e + n(n-1)\frac{\log A}{\log(3/4)}.
    \]
    
    On en déduit que \( e \; \leq n(n-1)\frac{\log A}{-\log(3/4)}\) et donc \( e \in \OO(n^2 \log A)\)
    
\end{proof}

\begin{proof}[Preuve de la terminaison et la complexité]
    
    Comme \( i \) est décrémenté de \( 1 \) lors d’un échange et incrémenté de \( 1 \) sinon l'entier \( i + e - e^* \) est constant tout au long de \hyperref[step:LLL]{\emph{LLL}}.
    
    Initialement \( i + e - e^* = 2 \) et à la fin de \hyperref[step:LLL]{\emph{LLL}} on a \( n + 1 + e - e^* = 2 \).
    On en déduit donc que \( e + e^* = 2e + n - 1 \in \OO(n^2 \log A) \).
    et donc d'après le lemme 2.5 le coût total de \hyperref[step:LLL]{\emph{LLL}} est \( \OO(n^2 \x n^2 \log A) \) opérations dans \( \Z \). Ce qui acheve la preuve.
\end{proof}

\chapter*{État de l'art de la réduction de réseaux euclidiens}
\addcontentsline{toc}{chapter}{État de l'art de la réduction de réseaux euclidiens et l'objectif de mon stage}

Voici une rétrospective structurée des avancées majeures dans la réduction de réseaux euclidiens.

\begin{itemize}
    \item[\textbf{1982}] \textbf{LLL} (Lenstra, Lenstra, Lovász) \parencite{Lenstra1982} introduisent le premier algorithme de réduction polynomial, basé sur une combinaison de \emph{size reduction} et d'une condition de Lovász. Il garantit :
    \[
    \|\bb_1\| \leq (4/3)^{(n-1)/2} \cdot \lambda_1(\LL), \quad \text{avec complexité binaire } \mathcal{O}(n^5 \beta^2)
    \]
    
    \item[\textbf{1991}] \textbf{BKZ} \parencite{Schnorr1994} introduit une approche par blocs. L’algorithme applique un solveur SVP de petite dimension \(\beta\) à des sous-blocs de la base :
    \[
    \|\bb_1\| \leq \gamma_\beta^{(n-1)/(\beta - 1)} \cdot \lambda_1(\LL)
    \]
    Le coût est exponentiel en \(\beta\) mais reste efficace pour \(\beta \leq 40\) en pratique.
    
    \item[\textbf{2009}] \textbf{L2} \parencite{Nguyen2009} améliore LLL sur le plan de la stabilité numérique, sans gain théorique majeur sur la qualité de la base. Complexité similaire à LLL, mais plus efficace pour des entrées en flottants.
    
    \item[\textbf{2011}] \textbf{$\tilde{L}_1$} \parencite{Novocin2011} propose une version rapide de LLL inspirée du GCD rapide de Knuth–Schönhage. Il introduit une stratégie récursive appelée \emph{Lift-Reduction} et atteint une complexité quasi-linéaire :
    \[
    \mathcal{O}\left(d^{5+\varepsilon} \beta + d^{\omega+1+\varepsilon} \beta^{1+\varepsilon} \right)
    \]
    avec une qualité comparable à LLL : \( \|\bb_1\| \leq 2^{\alpha n} \cdot |\LL|^{1/n} \).
    
    \item[\textbf{2011}] \textbf{Terminating BKZ}  \parencite{cryptoeprint:2011/198} propose une modélisation dynamique affine de BKZ. Ils montrent que même interrompu prématurément, BKZ garantit :
    \[
    \|\bb_1\| \leq 2^{\frac{\gamma_\beta(n-1)}{2(\beta - 1)} + \frac{3}{2}} \cdot |\LL|^{1/n}
    \]
    après seulement \( \mathcal{O}(n^3/\beta^2 \cdot \log \|B\|) \) appels à un solveur SVP.
    
    \item[\textbf{2019}] \textbf{KEF} \parencite{Kirchner2021}propose un algorithme heuristique récursif exploitant la \emph{Geometric Series Assumption} (GSA) pour guider la réduction. Il utilise des techniques de FFT et obtient une complexité heuristique :
    \[
    \widetilde{\mathcal{O}}(n^\omega \cdot \log \kappa(B))
    \]
    avec une qualité empirique équivalente à BKZ en grande dimension (\( n > 2000 \)).
    
    \item[\textbf{2023}] \textbf{Iterated Compression} \parencite{Ryan2023} présente un algorithme récursif fondé sur des opérations de compression stables, une métrique de \emph{drop}, et des profils dynamiques :
    \[
    \|\bb_1\| \leq 2^{\alpha n} \cdot |\LL|^{1/n}, \quad \|\bb_n^*\| \geq 2^{-\alpha n} \cdot |\LL|^{1/n}
    \]
    avec complexité heuristique :
    \[
    \mathcal{O}(n^\omega(C + n)^{1+\varepsilon}), \quad C = \log(\|B\| \cdot \|B^{-1}\|)
    \]
    
\end{itemize}
\begin{comment}
    \subsection*{Mesures de qualité}
    La qualité d’une base \( B = (\bb_1, \dots, \bb_n) \) se mesure par :
    \begin{itemize}
        \item La norme \( \|\bb_1\| \), en comparaison avec \( \lambda_1(L) \),
        \item Le \emph{facteur d’Hermite} : \( \frac{\|\bb_1\|}{\det(L)^{1/n}} \),
        \item La décroissance des \( \|\bb_i^*\| \) (orthogonalité), souvent modélisée par la GSA : \( \|\bb_i^*\| \approx a \cdot r^i \).
    \end{itemize}
    
    \subsection*{Applications}
    La réduction de réseau est centrale en :
    \begin{itemize}
        \item \textbf{Cryptanalyse} de RSA (Coppersmith), NTRU, LWE, FHE,
        \item \textbf{Algèbre effective}, systèmes diophantiens,
        \item \textbf{Optimisation discrète}.
    \end{itemize}
    
    \subsection*{Conclusion}
    L'évolution de la réduction de réseaux est marquée par un passage progressif :
    \begin{itemize}
        \item de méthodes \textbf{théoriques lentes mais garanties} (HKZ, LLL),
        \item vers des approches \textbf{rapides, heuristiques et massivement parallélisables} (KEF, Iterated Compression),
    \end{itemize}
    en maintenant un objectif constant : approcher au mieux \( \lambda_1(L) \) avec un coût algorithmique acceptable, même en très grande dimension.
    
    
\end{comment}
	                     % Réduction de réseaux euclidiens : introduction à LLL, bases réduites, fonctionnement et propriétés %
%                                                                                                                                                             %
                                                                                                                                   %
%%%%%%%%%%%%%%%%%%%%%%%%%%%%%%%%%%%%%%%%%%%%%%%%%%%%%%%%%%%%%%%%%%%%%%%%%%%%%%%%%%%%%%%%%%%%%%%%%%%%%%%%%%%%%%%%%%%%%%%%%%%%%%%%%%%%%%%%%%%%%%%%%%%%%%%%%%%%%%%
%                                                                                                                                                             %
%                                             ▗▄▄▖▗▄▄▄▖ ▗▄▄▖▗▄▄▄▖▗▄▄▄▖ ▗▄▖ ▗▖  ▗▖    ▗▄▄▄▖        ▗▄▄▄▖▗▄▄▄▖                                                  %
%                                            ▐▌   ▐▌   ▐▌     █    █  ▐▌ ▐▌▐▛▚▖▐▌      █            █    █                                                    %
%                                             ▝▀▚▖▐▛▀▀▘▐▌     █    █  ▐▌ ▐▌▐▌ ▝▜▌      █            █    █                                                    %
%                                            ▗▄▄▞▘▐▙▄▄▖▝▚▄▄▖  █  ▗▄█▄▖▝▚▄▞▘▐▌  ▐▌    ▗▄█▄▖        ▗▄█▄▖▗▄█▄▖                                                  %
%                                                                                                                                                             %
%                ▗▄▄▖  ▗▄▖ ▗▖ ▗▖  ▗▖▗▖  ▗▖ ▗▄▖ ▗▖  ▗▖▗▄▄▄▖ ▗▄▖ ▗▖       ▗▖    ▗▄▖▗▄▄▄▖▗▄▄▄▖▗▄▄▄▖ ▗▄▄▖▗▄▄▄▖    ▗▄▄▖  ▗▄▖  ▗▄▄▖▗▄▄▄▖ ▗▄▄▖ ▗▄▄▖                  %
%                ▐▌ ▐▌▐▌ ▐▌▐▌  ▝▚▞▘ ▐▛▚▖▐▌▐▌ ▐▌▐▛▚▞▜▌  █  ▐▌ ▐▌▐▌       ▐▌   ▐▌ ▐▌ █    █    █  ▐▌   ▐▌       ▐▌ ▐▌▐▌ ▐▌▐▌     █  ▐▌   ▐▌                     %
%                ▐▛▀▘ ▐▌ ▐▌▐▌   ▐▌  ▐▌ ▝▜▌▐▌ ▐▌▐▌  ▐▌  █  ▐▛▀▜▌▐▌       ▐▌   ▐▛▀▜▌ █    █    █  ▐▌   ▐▛▀▀▘    ▐▛▀▚▖▐▛▀▜▌ ▝▀▚▖  █  ▐▌    ▝▀▚▖                  %
%                ▐▌   ▝▚▄▞▘▐▙▄▄▖▐▌  ▐▌  ▐▌▝▚▄▞▘▐▌  ▐▌▗▄█▄▖▐▌ ▐▌▐▙▄▄▖    ▐▙▄▄▖▐▌ ▐▌ █    █  ▗▄█▄▖▝▚▄▄▖▐▙▄▄▖    ▐▙▄▞▘▐▌ ▐▌▗▄▄▞▘▗▄█▄▖▝▚▄▄▖▗▄▄▞▘                  %
%                                                                                                                                                             %
%%%%%%%%%%%%%%%%%%%%%%%%%%%%%%%%%%%%%%%%%%%%%%%%%%%%%%%%%%%%%%%%%%%%%%%%%%%%%%%%%%%%%%%%%%%%%%%%%%%%%%%%%%%%%%%%%%%%%%%%%%%%%%%%%%%%%%%%%%%%%%%%%%%%%%%%%%%%%%%
%                                                                                                                                                             %
% Pas de sous-fichiers                                                                                                                                        %
%                                                                                                                                                             %
% Le fichier parent est : lattice_basics.tex                                                                                                                  %
%                                                                                                                                                             %
%%%%%%%%%%%%%%%%%%%%%%%%%%%%%%%%%%%%%%%%%%%%%%%%%%%%%%%%%%%%%%%%%%%%%%%%%%%%%%%%%%%%%%%%%%%%%%%%%%%%%%%%%%%%%%%%%%%%%%%%%%%%%%%%%%%%%%%%%%%%%%%%%%%%%%%%%%%%%%%
%                                                                                                                                                             %
% Il s'agit de la deuxième partie du chapitre 1 qui introduit les réseaux polyonomiaux.                                                                       %
%                                                                                                                                                             %
%%%%%%%%%%%%%%%%%%%%%%%%%%%%%%%%%%%%%%%%%%%%%%%%%%%%%%%%%%%%%%%%%%%%%%%%%%%%%%%%%%%%%%%%%%%%%%%%%%%%%%%%%%%%%%%%%%%%%%%%%%%%%%%%%%%%%%%%%%%%%%%%%%%%%%%%%%%%%%%

\chapter{Réseaux polynomiaux}      

\lettrine{L}{orsque} les coefficients des matrices appartiennent à un corps \( \K \), les opérations classiques telles que la multiplication, l'inversion, le calcul du déterminant ou la résolution de systèmes linéaires possèdent des complexités comparables. En revanche, lorsqu'on considère des matrices à coefficients dans l'anneau \( \K[x] \), des différences apparaissent : si le calcul du déterminant conserve la même complexité que celle du produit matriciel, l'inversion est plus coûteuse. Cette complexité découle de la structure de l'anneau \( \K[x] \) : bien qu'il s'agisse d'un anneau principal (à la différence de \( \K[x,y] \)), il ne s'agit pas d'un corps. Ainsi, certaines opérations, comme l’inversion, ne sont plus systématiquement réalisables. Notamment, dans \( \K[x] \), seuls les polynômes constants non nuls sont inversibles. Cette restriction impose de repenser et redéfinir rigoureusement plusieurs notions de l'algèbre linéaire. Les matrices à coefficients dans \( \K[x] \) sont essentielles dans de nombreuses applications. \footnote{Par exemple dans l’interpolation bivariée, une étape centrale du décodage des codes de Reed-Solomon}.  Ce chapitre vise à explorer les réseaux polynomiaux  et leurs propriétés spécifiques. 
Ce mémoire avait pour ambition initiale de motiver rigoureusement l’introduction des bases réduites dans le cadre des matrices polynomiales. Toutefois, afin de ne pas aborder un sujet trop éloigné des objectifs du stage, et par souci de concision, nous ne détaillerons pas ici les aspects liés aux mesures de complexité. Ce que j'avais rédigé initialement se retrouve en annexe. Notons simplement qu’il est essentiel d’analyser finement le comportement des matrices polynomiales vis-à-vis des opérations algébriques usuelles, en particulier la multiplication. Cela justifie l’introduction de la notion de degré de ligne, qui jouera un rôle central dans le chapitre suivant.

\section{Définitions et exemples}

On commence par introduire la notion de réseau polynomial. Des rappels détaillés sur les anneaux et sur les modules sont dans l’annexe correspondante.

\begin{definition}
	Un \textbf{réseau polynomial} $\LL$ est un \( \K[x] \)-module libre de type fini.
\end{definition}

On peut montrer qu'il existe une famille $\K[x]$-libre maximale $(\bb_i)_{1 \leq i \leq m}$ dans $\LL$ telle que

$$\LL = \bigoplus\limits_{1 \leq i \leq m} \K[x] \bb_i:=\{a_1\bb_1 + \cdots a_m\bb_m : a_i \in \K[x]\}$$


Cette famille est appelée \textbf{base} de $\LL$, si on note \(B \in \K[x]^{m \x n} \) la matrice de la famille $(\bb_i)_{1 \leq i \leq m}$ on notera $\LL(B)$ le réseau de base $B$, donc \textbf{engendré} par la famille $(\bb_i)_{1 \leq i \leq m}$. L'entier $m$ est commun à toutes les bases de $\LL$ et on l'appelle \textbf{rang} de $\LL$. Lorsque $n=m$, on dit que le réseau est de \textbf{rang plein}. Un élément de \( \K[x]^{m \x n} \) est appelé matrice polynomiale.

\begin{example}
	\[B=
	\begin{pmatrix}
		3x + 4  & x^9 \\
		5 & x^2 + 1
	\end{pmatrix}
	\in \R[x]^{2 \x 2}
	\]
	est une matrice polynomiale qui représente le réseau $\LL (B)$.
\end{example}

\begin{proposition}
    Soient \( P \) et \( Q \) deux bases de lignes d’un même \( \F[x] \)-module libre . Alors, il existe une matrice unimodulaire \( U \) telle que
    \[
    P \;=\; U \, Q.
    \]
\end{proposition}
On observe une analogie structurelle entre les matrices et les modules : de même que les matrices à coefficients dans \( \K \) sont naturellement liées aux \( \K \)-espaces vectoriels, les matrices à coefficients dans \( \K[x] \) interviennent dans l'étude des \( \K[x] \)-modules libres.

\begin{comment}
\section{Deux points de vue sur les matrices polynomiales}

\begin{theoreme}
    On dispose d'un \textbf{isomorphisme structurel} au sens des modules:
    \[
    \K[x]^{m \x n} \;\cong\; \K^{m\x n}\bigr[x]
    \]
\end{theoreme}

\begin{example}
    \[
    \begin{pmatrix}
        3x + 4  & x^9 \\
        5 & x^2 + 1
    \end{pmatrix}
    =
    \begin{pmatrix}
        0 & 1 \\
        0 & 0 
    \end{pmatrix}
    x^9
    +
    \begin{pmatrix}
        0 & 0 \\
        0 & 1 
    \end{pmatrix}
    x^2
    +
    \begin{pmatrix}
        3 & 0 \\
        0 & 0  
    \end{pmatrix}
    x
    +
    \begin{pmatrix}
        4 & 0 \\
        5 & 1 
    \end{pmatrix} 
    \]	



\end{example}

On peut donc interpréter une matrice polynomiale soit comme un polynôme à coefficients dans matriciel, soit comme une matrice à coefficients polynomiaux. Le choix de l’approche dépend alors principalement du type de calculs et d’estimations de complexité qu’on souhaite mener. Deux approches principales sont alors envisageables pour effectuer des calculs algébriques (résolution de systèmes, inversion, déterminant, etc.) :

\begin{enumerate}
    \item \textbf{Appliquer les algorithmes d'algèbre linéaire classique sur} \(\K[x]^{m \x n}\).
    
    On traite la matrice comme un objet usuel, en considérant simplement que les coefficients se trouvent dans l’anneau principal \( \K[x] \).
    \begin{itemize}
        \item[$\bullet$] \textbf{Avantage} : cette méthode tire parti de la robustesse théorique et de l'expérience accumulée avec les algorithmes classiques.
        
        \item[$\bullet$] \textbf{Limite} : on peut rapidement générer des calculs complexes, par exemple des fractions polynomiales de grand degré, et obtenir des bornes de complexité peu réaliste.
    \end{itemize}
    
    \vspace{0.5cm}
    \item \textbf{Appliquer des algorithmes d'arithmétique polynomiale sur }\( \K^{m\x n}[x] \)
    
        \begin{itemize}
        \item[$\bullet$] \textbf{Avantage} : on bénéficie de techniques optimisées pour les polynômes (accélération de la multiplication via FFT, etc.), on a un meilleur contrôle du degré.
        
        \item[$\bullet$] \textbf{Limite} : cela peut parfois s'avérer restrictif\footnote{Dans la division avec reste, on suppose souvent que \( B \) possède un coefficient dominant inversible (\( \mathrm{lc}(B)\neq 0 \)). On pourrait toutefois assouplir cette hypothèse, avec la notion de base "réduite" qui sera définie dans les prochains chapitres}
        ou inefficace
        \footnote{Si l'on travaille avec une matrice de degré \( d \) dont de nombreuses entrées ont un degré bien inférieur à \( d \), les algorithmes basés uniquement sur le degré maximal risquent de fournir des performances dégradées.}
        , en particulier si les degrés des entrées diffèrent sensiblement d'une ligne ou d'une colonne à l'autre.
    \end{itemize}
\end{enumerate}

\end{comment}

\section{Notion de degré en ligne}

\begin{comment}
    \begin{definition}
        Soit \( M \in \K[x]^{m \x n} \), la \textbf{taille} de \( M \), notée \( \size(M) \), est le nombre de coefficients distincts de \( \K \) nécessaires pour sa représentation dense.
        
        Et on a la relation
        \[
        \size(M) = \sum_{i,j} \size(a_{i,j}) = \sum_{i,j}( 1 + \max(0, \deg(M_{i,j}))).
        \]
    \end{definition}
    En général, la taille n'est pas compatible avec le produit matriciel, mais cela peut être le cas dans certains cas particuliers.
\end{comment}

\begin{notation}
    On note \( [d] \) un polynôme de degré \( d \). Par exemple \( x^2 + 1 \) sera noté \( [2] \), \(  x^9 \) sera noté \( [9] \).
\end{notation}

\begin{example}
	La multiplication ne se passe pas forcément bien. On voit que cela donne des bornes non pertinentes et il serait utile de rajoute des critères de mesure du degré. Considérons les matrices de degrés :
	\[
	\begin{pmatrix} 
		[100] & [1]  \\ 
		[100] & [1]  
	\end{pmatrix}
	\begin{pmatrix}
		[1] & [1]  \\ 
		[1] & [1] 
	\end{pmatrix}
	=
	\begin{pmatrix} 
		[101] & [101] \\ 
		[101] & [101] 
	\end{pmatrix}
	\]
\end{example}

\begin{definition}
   \leavevmode\vspace{0.5\baselineskip}
   
         $\bullet \quad$ Pour \( \mathbf{m} = (m_1, \cdots ,m_n) \in \K[x]^{1 \x n} \), on définit son \textbf{degré en ligne} par :
        \[
        \rdeg(\mathbf{m}) = \max_{1 \leq i \leq n} \deg(m_i)\in \Z
        \]
         $\bullet \quad$ Pour \( M = 
         \begin{pmatrix}
             \cdots \mathbf{M_1} \cdots \\
             \vdots \\
             \cdots \mathbf{M_n} \cdots
         \end{pmatrix}, 
         \mathbf{M_i} \in \K[x]^{1 \x n}  ~~ \forall 1 \leq i \leq n \), on définit son \textbf{degré en ligne} par :
        \[
        \rdeg(M) = (\rdeg(\mathbf{M_i}))_{1 \leq i \leq n} \in \Z^n
        \]
\end{definition}

\begin{example}
    Soit
    \(
    M = 
    \begin{pmatrix}
        3x + 4  & x^9 \\
        5 & x^2 + 1
    \end{pmatrix}
    \in \F_2[x].
    \) 
    Alors
    \(
    \rdeg(M)=
    \begin{pmatrix}
        \rdeg(~3x+4~,~x^9~) \\
        \rdeg(~5~,~x^2+1~)
    \end{pmatrix}
= 
\begin{pmatrix}
    9 \\
    2
\end{pmatrix}
    \)
\end{example}

Cette définition du degré de ligne présente une limite : si \( c=bA \), on a bien en général \( \rdeg(c)\leq \rdeg(b)+\rdeg(A) \), mais cette majoration est souvent trop lâche pour nos besoins. Ce qui nous intéresse est de pouvoir caractériser plus finement le degré de \( c \), voire d’obtenir une égalité. Cela motive l’introduction de la définition de degré décalé.

\begin{definition}
    Soit \( \vec{\mathbf{s}} =(s_1, \cdots, s_n)\in \Z^n \). On appelle \( \vec{\mathbf{s}} \) le \textbf{vecteur de décalage}.
    
    $\bullet \quad$ Pour \( \mathbf{m} = (m_1, \cdots ,m_n) \in \K[x]^{1 \x n} \), on définit son \textbf{degré en ligne \( \vec{\mathbf{s}} \)-décalé} par :
        \[
        \rdeg_{\vec{\mathbf{s}}}(\mathbf{m}) = \max_{1 \leq i \leq n} (\deg(m_i) + s_i)
        \]
        
        $\bullet \quad$ Pour \( M = 
        \begin{pmatrix}
            \cdots \mathbf{M_1} \cdots \\
            \vdots \\
            \cdots \mathbf{M_n} \cdots
        \end{pmatrix}, 
        \mathbf{M_i} \in \K[x]^{1 \x n}  ~~ \forall 1 \leq i \leq n \), on définit son \textbf{degré en ligne décalé} par :
        \[
        \rdeg_{\vec{\mathbf{s}}}(M) = \big( \rdeg_{\vec{\mathbf{s}}}(\mathbf{M_i}) \big)_{1 \leq i \leq m} \in \Z^m
        \]
\end{definition}

\begin{notation}
    Soit $\vec{\mathbf{s}}=(s_1, \cdots, s_n) \in \Z^n$. On note $x^{\vec{\mathbf{s}}}$ la matrice diagonale 
    \(
    \begin{pmatrix}	
        x^{s_1} & & \\
        & \ddots & \\
        &  & x^{s_n}
    \end{pmatrix} \in M_n(\K [x])
    \).
\end{notation}

\begin{proposition}
    Soit \( A \in \K[x]^{m \x n} \), et \( \vec{\mathbf{s}} \in \Z^n \). Alors \( \rdeg_{\vec{\mathbf{s}}}(A) = \rdeg (A x^{\vec{\mathbf{s}}}) \).
\end{proposition}

\begin{example}
    Soit
    \(
    M =
    \begin{pmatrix}
        3x + 4  & x^9 \\
        5 & x^2 + 1
    \end{pmatrix}
    \in \R_2[x]
    \quad \text{et} \quad \vec{\mathbf{s}} =(8, 0)
    \)
    
    Alors
    \[
    \rdeg_{\vec{\mathbf{s}}} (M) = \rdeg (M\cdot x^{\vec{\mathbf{s}}}) =
    \rdeg
    \begin{pmatrix}
        3x^9 + 4 x^8 & x^9\\
        5x^8 & x^2+1 
    \end{pmatrix} 
    =(9, 8)
    \]
\end{example}

\begin{proposition}
    Soit \( A \in \K[x]^{m \x n} \) et \( \vec{\mathbf{s}} \in \Z^n \). Alors, on a les propriétés suivantes :
    
        $\bullet ~$ \( \rdeg_{\vec{\mathbf{s}}}(A) = \vec{\mathbf{v}} \) si et seulement si \( \rdeg(x^{-\vec{\mathbf{v}}} A x^{\vec{\mathbf{s}}}) = 0 \).  
        
        $\bullet ~$ \( \rdeg_{\vec{\mathbf{s}}}(A) \leq \vec{\mathbf{v}} \) si et seulement si \( \rdeg(x^{-\vec{\mathbf{v}}} A x^{\vec{\mathbf{s}}}) \leq 0 \).  
    
\end{proposition}

\begin{example}
    Soit 
    \[
    F = 
    \begin{pmatrix}
        1 & 0         & 1 \\
        x & 1         & x + 1 \\
        1 & x^3 + x^2 & x
    \end{pmatrix},
    \quad \vec{\mathbf{u}} = (1, 0, 0, 1).
    \]
    
    Alors 
    \[
    \vec{\mathbf{v}} = \rdeg_{\vec{\mathbf{u}}}(F) = (1,2,3,4)
    \quad \text{et} \quad
    x^{-\vec{\mathbf{v}}} A x^{\vec{\mathbf{s}}} =
    \begin{pmatrix}
        1 & 0 & x^{-1} \\
        1 & x^{-2} & x^{-2} + x^{-1} \\
        x^{-2} & x^{-1} + 1 & x^{-2}
    \end{pmatrix}.
    \]
\end{example}

On va définir un ordre sur les degrés de ligne, bien que non total.

\begin{definition}
    Soit \( m \in \N^* \) et soient \( \mathbf{u} = (u_1, \dots, u_m) \), \( \mathbf{v} = (v_1, \dots, v_m) \in \Z^m \) deux vecteurs de degrés de ligne, triés par valeur croissante.
    On définit une relation d'ordre partiel, notée \( \leq_{ob} \) (ordre obtenu par composantes), par :
    \[
    \mathbf{u} \leq_{ob} \mathbf{v} \quad \text{si et seulement si} \quad u_i \leq v_i \quad \text{pour tout } i \in \{1, \dots, m\}.
    \]
\end{definition}

\begin{proposition}
    Soit \( A \in \K[x]^{m \x n} \), \( \vec{\mathbf{b}} \in \K[x]^{1 \x m} \) et \( \vec{\mathbf{c}}=\vec{\mathbf{b}}A \). Soit \( \vec{\mathbf{v}} = \rdeg_{\vec{\mathbf{u}}}(A) \) et \( w = \rdeg_{\vec{\mathbf{v}}}(b) \). 
    
    Alors
    \[
    \rdeg_{\vec{\mathbf{u}}}(c) \leq_{ob} w.
    \]
\end{proposition}

Les notions de degrés introduites précédemment, ainsi que les techniques qui en découlent, permettent d'accélérer certains algorithmes dans des situations spécifiques. Cependant, elles présentent des limites structurelles importantes : notamment, les degrés de lignes et de colonnes ne possèdent pas de bonnes propriétés vis-à-vis de la multiplication matricielle. De plus, plusieurs problèmes restent ouverts quant à la possibilité d'obtenir des algorithmes plus efficaces pour le calcul du déterminant ou de l'inverse de matrices polynomiales. Ces obstacles mettent en évidence l'intérêt crucial de la réduction des matrices polynomiales, outil indispensable pour contrôler la croissance des degrés et améliorer ainsi les performances des algorithmes associés.                       % Réseaux polynomiaux : définition en tant que $\K[x]$-modules libres de type fini                    %
%                                                                                                                                                             %
%%%%%%%%%%%%%%%%%%%%%%%%%%%%%%%%%%%%%%%%%%%%%%%%%%%%%%%%%%%%%%%%%%%%%%%%%%%%%%%%%%%%%%%%%%%%%%%%%%%%%%%%%%%%%%%%%%%%%%%%%%%%%%%%%%%%%%%%%%%%%%%%%%%%%%%%%%%%%%%                                   % Chapitre 1 : Bases sur les réseaux euclidiens et polynomiaux (rappels, définitions fondamentales)  %

%                                                                                                                                                             %
%%%%%%%%%%%%%%%%%%%%%%%%%%%%%%%%%%%%%%%%%%%%%%%%%%%%%%%%%%%%%%%%%%%%%%%%%%%%%%%%%%%%%%%%%%%%%%%%%%%%%%%%%%%%%%%%%%%%%%%%%%%%%%%%%%%%%%%%%%%%%%%%%%%%%%%%%%%%%%%
%                                                                                                                                                             %
%                                                   ▗▄▄▖▗▖ ▗▖ ▗▄▖ ▗▄▄▖▗▄▄▄▖▗▄▄▄▖▗▄▄▖     ▗▄▄▄▖▗▄▄▄▖                                                           %
%                                                  ▐▌   ▐▌ ▐▌▐▌ ▐▌▐▌ ▐▌ █  ▐▌   ▐▌ ▐▌      █    █                                                             %
%                                                  ▐▌   ▐▛▀▜▌▐▛▀▜▌▐▛▀▘  █  ▐▛▀▀▘▐▛▀▚▖      █    █                                                             %
%                                                  ▝▚▄▄▖▐▌ ▐▌▐▌ ▐▌▐▌    █  ▐▙▄▄▖▐▌ ▐▌    ▗▄█▄▖▗▄█▄▖                                                           %
%                                                                                                                                                             %
%                              ▗▖    ▗▄▖▗▄▄▄▖▗▄▄▄▖▗▄▄▄▖ ▗▄▄▖▗▄▄▄▖    ▗▄▄▖ ▗▄▄▄▖▗▄▄▄ ▗▖ ▗▖ ▗▄▄▖▗▄▄▄▖▗▄▄▄▖ ▗▄▖ ▗▖  ▗▖                                           %
%                              ▐▌   ▐▌ ▐▌ █    █    █  ▐▌   ▐▌       ▐▌ ▐▌▐▌   ▐▌  █▐▌ ▐▌▐▌     █    █  ▐▌ ▐▌▐▛▚▖▐▌                                           %
%                              ▐▌   ▐▛▀▜▌ █    █    █  ▐▌   ▐▛▀▀▘    ▐▛▀▚▖▐▛▀▀▘▐▌  █▐▌ ▐▌▐▌     █    █  ▐▌ ▐▌▐▌ ▝▜▌                                           %
%                              ▐▙▄▄▖▐▌ ▐▌ █    █  ▗▄█▄▖▝▚▄▄▖▐▙▄▄▖    ▐▌ ▐▌▐▙▄▄▖▐▙▄▄▀▝▚▄▞▘▝▚▄▄▖  █  ▗▄█▄▖▝▚▄▞▘▐▌  ▐▌                                           %
%                                                                                                                                                             %
%%%%%%%%%%%%%%%%%%%%%%%%%%%%%%%%%%%%%%%%%%%%%%%%%%%%%%%%%%%%%%%%%%%%%%%%%%%%%%%%%%%%%%%%%%%%%%%%%%%%%%%%%%%%%%%%%%%%%%%%%%%%%%%%%%%%%%%%%%%%%%%%%%%%%%%%%%%%%%%
%                                                                                                                                                             %
% Les sous-fichiers sont :                                                                                                                                    %
%                                                                                                                                                             %
% polynomial_lattice_reduction.tex     euclidean_lattice_reduction.tex                                                                                        %
%                                                                                                                                                             %
% Le fichier parent est : report.tex                                                                                                                          %
%                                                                                                                                                             %
%%%%%%%%%%%%%%%%%%%%%%%%%%%%%%%%%%%%%%%%%%%%%%%%%%%%%%%%%%%%%%%%%%%%%%%%%%%%%%%%%%%%%%%%%%%%%%%%%%%%%%%%%%%%%%%%%%%%%%%%%%%%%%%%%%%%%%%%%%%%%%%%%%%%%%%%%%%%%%%

\chapter{Réduction de réseaux polynomiaux}

%
%%%%%%%%%%%%%%%%%%%%%%%%%%%%%%%%%%%%%%%%%%%%%%%%%%%%%%%%%%%%%%%%%%%%%%%%%%%%%%%%%%%%%%%%%%%%%%%%%%%%%%%%%%%%%%%%%%%%%%%%%%%%%%%%%%%%%%%%%%%%%%%%%%%%%%%%%%%%%%%
%                                                                                                                                                             %
%                                             ▗▄▄▖▗▄▄▄▖ ▗▄▄▖▗▄▄▄▖▗▄▄▄▖ ▗▄▖ ▗▖  ▗▖    ▗▄▄▄▖▗▄▄▄▖        ▗▄▄▄▖                                                  %
%                                            ▐▌   ▐▌   ▐▌     █    █  ▐▌ ▐▌▐▛▚▖▐▌      █    █            █                                                    %
%                                             ▝▀▚▖▐▛▀▀▘▐▌     █    █  ▐▌ ▐▌▐▌ ▝▜▌      █    █            █                                                    %
%                                            ▗▄▄▞▘▐▙▄▄▖▝▚▄▄▖  █  ▗▄█▄▖▝▚▄▞▘▐▌  ▐▌    ▗▄█▄▖▗▄█▄▖        ▗▄█▄▖                                                  %
%                                                                                                                                                             %
%        ▗▄▄▖  ▗▄▖ ▗▖ ▗▖  ▗▖▗▖  ▗▖ ▗▄▖ ▗▖  ▗▖▗▄▄▄▖ ▗▄▖ ▗▖       ▗▖    ▗▄▖▗▄▄▄▖▗▄▄▄▖▗▄▄▄▖ ▗▄▄▖▗▄▄▄▖    ▗▄▄▖ ▗▄▄▄▖▗▄▄▄ ▗▖ ▗▖ ▗▄▄▖▗▄▄▄▖▗▄▄▄▖ ▗▄▖ ▗▖  ▗▖          %
%        ▐▌ ▐▌▐▌ ▐▌▐▌  ▝▚▞▘ ▐▛▚▖▐▌▐▌ ▐▌▐▛▚▞▜▌  █  ▐▌ ▐▌▐▌       ▐▌   ▐▌ ▐▌ █    █    █  ▐▌   ▐▌       ▐▌ ▐▌▐▌   ▐▌  █▐▌ ▐▌▐▌     █    █  ▐▌ ▐▌▐▛▚▖▐▌          %
%        ▐▛▀▘ ▐▌ ▐▌▐▌   ▐▌  ▐▌ ▝▜▌▐▌ ▐▌▐▌  ▐▌  █  ▐▛▀▜▌▐▌       ▐▌   ▐▛▀▜▌ █    █    █  ▐▌   ▐▛▀▀▘    ▐▛▀▚▖▐▛▀▀▘▐▌  █▐▌ ▐▌▐▌     █    █  ▐▌ ▐▌▐▌ ▝▜▌          %
%        ▐▌   ▝▚▄▞▘▐▙▄▄▖▐▌  ▐▌  ▐▌▝▚▄▞▘▐▌  ▐▌▗▄█▄▖▐▌ ▐▌▐▙▄▄▖    ▐▙▄▄▖▐▌ ▐▌ █    █  ▗▄█▄▖▝▚▄▄▖▐▙▄▄▖    ▐▌ ▐▌▐▙▄▄▖▐▙▄▄▀▝▚▄▞▘▝▚▄▄▖  █  ▗▄█▄▖▝▚▄▞▘▐▌  ▐▌          %
%                                                                                                                                                             %
%%%%%%%%%%%%%%%%%%%%%%%%%%%%%%%%%%%%%%%%%%%%%%%%%%%%%%%%%%%%%%%%%%%%%%%%%%%%%%%%%%%%%%%%%%%%%%%%%%%%%%%%%%%%%%%%%%%%%%%%%%%%%%%%%%%%%%%%%%%%%%%%%%%%%%%%%%%%%%%
%                                                                                                                                                             %
% Pas de sous-fichiers                                                                                                                                        %
%                                                                                                                                                             %
% Le fichier parent est : lattice_reduction.tex                                                                                                               %
%                                                                                                                                                             %
%%%%%%%%%%%%%%%%%%%%%%%%%%%%%%%%%%%%%%%%%%%%%%%%%%%%%%%%%%%%%%%%%%%%%%%%%%%%%%%%%%%%%%%%%%%%%%%%%%%%%%%%%%%%%%%%%%%%%%%%%%%%%%%%%%%%%%%%%%%%%%%%%%%%%%%%%%%%%%%

\lettrine{L}{a} réduction des réseaux polynomiaux est une étape essentielle dans plusieurs applications algorithmiques, en particulier dans le décodage efficace des codes de Reed-Solomon généralisés. Cette opération vise à transformer une base quelconque d'un réseau polynomial en une base simplifiée. Dans cette section, nous détaillerons les principaux concepts, outils et algorithmes permettant de réaliser cette réduction en temps polynomial. On notera \( \K \) un corps quelconque.

\section{Généralités et notion de base d'ordre}
Soit \( F \in \F[x]^{m \x n},\) et un \textbf{degré de précision} \( \sigma \in \N \). On définit \[ (F, \sigma):= \{v \in \F[x]^{1 \x m}  ~|~  vF = 0 \mod x^\sigma\} \]

\begin{proposition}
	\( (F, \sigma) \) est un réseau polynomial de dimension \( m \).
\end{proposition}

On va s'intérésser à étudier les bases de ce réseau et définir une notion de base réduite.

\begin{definition}
	Une \( (F, \sigma) \)-\textbf{base d'ordre} est une base \footnote{au sens des \( \F[x] \)-modules} de \( (F, \sigma) \) de degré minimale.
\end{definition}

Quelle est la définition du degré ? Que signifie minimale dans ce contexte ?

\begin{definition}
	Soit \( F \in \F[x]^{m \x n} \). On dit que \( F \) est \textbf{réduite par ligne} si, pour tout \( U \in \F[x]^{m \x m} \) unimodulaire, on a : 
    \[ \rdeg(F) \leq_{ob} \rdeg(UF). \]
\end{definition}

Pour parler d'un minimum, il faut un ordre total. On voit enfin la nouvelle définition de base réduite.

\begin{definition}
	Une \textbf{base d'ordre} est une base de \( (F, \sigma) \) qui est réduite par ligne.
\end{definition}

On peut définir une notion de base d'ordre en incluant la notion de décalage des degrés, on munit \( (F, \sigma) \) d'un vecteur de décalage \( s \), qu'on notera \( (F, \sigma, s) \), on peut alors dire qu'une base d'ordre de cet ensemble est une base d'ordre de \( (F x^s, \sigma) \), on peut l'interpréter intuitivement comme une base réduite pour un décalage, qui sera utile dans le chapitre 6.

L'existence d'une base réduite est garantie, mais elle n'est pas nécessairement unique. Pour assurer l'unicité, une condition supplémentaire est requise : la forme de Popov.

\begin{proof}[Preuve naïve (incorrecte)]
	Considérons le minimum de tous les \(\)$\rdeg(P U)$ triés, pour toutes les matrices unimodulaires \( U \in \F[x]^{m \x m} \).  
	
	Toute base \( PU \) ayant un degré minimal est une base d'ordre. Attention : l'ordre \( \leq_{ob} \) n'est pas un ordre total. En effet, il est possible d'avoir deux bases dont les degrés en ligne sont respectivement \( (1,2,3) \) et \( (1,1,4) \). On ne peut pas encore garantir l'existence d'un minimum !
\end{proof}

%Est-ce qu'on pourrait faire un lien avec le dev de Taylor

\begin{definition}
	Soit \( A \in \F[x]^{m \x n} \) et soit \( v = \rdeg_u(A) \). On définit la \textbf{matrice des coefficients dominants} \footnote{La matrice des coefficients dominants est accessible dans \texttt{SageMath} via la méthode \texttt{leading\_matrix()}.}
     de \( A \), notée \( \lcoeff(A) \in \F^{m \x n} \), comme étant la matrice obtenue en extrayant la partie constante de \( x^{-v} A\), c'est-à-dire :
	\[
	\lcoeff(A) = \lim_{x \to \infty} x^{-v} A.
	\]
\end{definition}

\begin{example}
	Soit 
	\[
	F = 
	\begin{pmatrix}
		1 & 0 & 1 \\
		x & 1 & 1 + x \\
		1 & x^2 + x^3 & x
	\end{pmatrix}
    , \quad
    \vec{v} := \rdeg(F) = (0,1,3)
	\]
	
	Alors
	\[\displaystyle
	x^{-\vec{v}} \cdot F =
	\begin{pmatrix}
		1 & 0 & 1\\
		1 & x^{-1} & x^{-1} + 1 \\
		x^{-3} & x^{-1} + 1 & x^{-2}\\
	\end{pmatrix}
	= 
	\begin{pmatrix}
		1 & 0 & 1 \\
		1 & 0 & 1 \\
		0 & 1 & 0 
	\end{pmatrix}
	+ 
	\OO_{x \to \infty}(x^{-1})
	\]
\end{example}

\begin{proposition}[Transitivité, revisitée]
	Soient \( c := b \cdot A \), \( v = \rdeg_u(A) \) et \( w = \rdeg_v(b) \).
	
	Si \( \lcoeff(A) \) est \textbf{injective à gauche}, alors \( \rdeg_u(c) = w \).
\end{proposition}


\begin{remark}
    On a donc l'égalité par rapport à la proposition 4.4.
\end{remark}
\begin{proposition}
	Soit \( A \) une matrice. Si \( \lcoeff(A) \) est injective à gauche, alors \( A \) est réduite par ligne.
\end{proposition}

Maintenant que l'on connaît la définition d'une base réduite et un critère pour la définir comment la calculer efficacement ?

\begin{definition}
    On définit le \textbf{pivot} d'une ligne comme l'élément non nul de degré maximal le plus à droite dans cette ligne.
\end{definition}

\begin{definition}
	Soit \( W \in \F[x]^{m \x n} \). La matrice \( W \) est dite en \textbf{forme Popov faible} si chaque ligne de \( W \) possède un pivot, et si les indices de colonnes de ces pivots sont distincts deux à deux.
\end{definition}

\begin{example}
	Soit $ W_1, W_2 \in M_2(\Z_7 [x])$ tel que
	\[
    W_1=
	\begin{pmatrix}
        3x + 4 & \fcolorbox{white}{red!20}{$x^9$} \\
        5 & \fcolorbox{white}{red!20}{$x^2+1$}
    \end{pmatrix}
    ,~W_2= 
    \begin{pmatrix}
        \fcolorbox{white}{red!20}{$2x^7 + 5x^5 + 3x + 4$} & x^5 \\
        5 & \fcolorbox{white}{red!20}{$x^2+1$}
    \end{pmatrix}
	\]
	On note les pivots en rouge. $W_1$ n'est pas en forme de Popov faible car les pivots sont sur la même colonne, et $W_2$ est en forme Popov faible , car les pivots, ont des indices de colonnes distincts.
\end{example}	

Pour transformer une matrice en forme Popov faible, on peut utiliser l'algorithme proposé dans \parencite{Mulders2003}, qui fournit une méthode systématique pour y parvenir en appliquant des transformations unimodulaires par lignes.

\begin{definition}
    On appelle \textbf{transformation simple} de la ligne \(k\) sur la ligne \(l\) l'opération consistant à soustraire \(c x^e\) fois la ligne \(k\) à la ligne \(l\), où \(c \in \F\) et \(e \in \mathbb{N}\). On dit qu'elle est du \textbf{premier type} si les pivots de la ligne \( k \) et de la ligne \( l \) ont les mêmes indices et du \textbf{deuxième type} sinon.
\end{definition}

%Une suite décroissante de row degree doit être stable, lien avec base de grobner
\footnote{L'algorithme suivant est disponible dans \texttt{SageMath} via la méthode \texttt{weak\_popov\_form()}.}

\begin{smallalgo}{WeakPopovForm \parencite{Mulders2003} }{algo:WeakPopovForm}
    \KwIn{\( \mathcal{M} \in \F[x]^{n \times m} \)}
    \KwOut{\( \mathcal{N} \) en forme de Popov faible, obtenue par des transformations simples de premier type appliquées à \( \mathcal{M} \)}
    
    \( A \gets \text{copie}(\mathcal{M}) \)\;
    
    \While{\( A \) n’est pas en forme de Popov faible}{
       Appliquer une transformation simple du premier type à \( A \)\;
    }
    
    \( \mathcal{N} \gets \text{copie}(A) \)\;
    \KwRet{\( \mathcal{N} \)}
\end{smallalgo}

\begin{theoreme}[Correction et complexité]
    \leavevmode\vspace{0.5\baselineskip}
    
    L'algorithme \textsc{WeakPopovForm} est correct. Sa complexité est bornée par $\OO(n m r d^2)$ opérations dans le corps de base, où $r$ désigne le rang de la matrice $M$, et $d$ une borne supérieure sur le degré des coefficients de $\mathcal{M}$.
\end{theoreme}

\begin{example}
	\[
    \begin{pmatrix}
        3x + 4 & \fcolorbox{white}{red!20}{$x^9$} \\
        5 & \fcolorbox{white}{red!20}{$x^2+1$}
    \end{pmatrix}
	\xrightarrow{(1)}
	\begin{pmatrix}
        2x^7+3x + 4 & \fcolorbox{white}{red!20}{$6x^7$} \\
        5 & \fcolorbox{white}{red!20}{$x^2+1$}
    \end{pmatrix}\xrightarrow{(2)}
	\begin{pmatrix}
        \fcolorbox{white}{red!20}{$2x^7 + 5x^5 + 3x + 4$} & x^5 \\
        5 & \fcolorbox{white}{red!20}{$x^2+1$}
    \end{pmatrix}
	\]
	
	\begin{enumerate}
		\item Ajouter \( 6 x^7 \) fois la deuxième ligne à la première ligne.
		\item Ajouter \( x^5 \) fois la deuxième ligne à la première ligne.
	\end{enumerate}
	
	La matrice finale est en forme Popov faible, les pivots ont des indices de colonnes distincts.
	
\end{example}
\begin{theoreme}
    Toute matrice en \textit{forme Popov faible} est réduite par ligne.
\end{theoreme}


\section{Algorithmes de calculs de base d'ordre}

Cette section va montrer qu'il existe des algorithmes plus rapide que \parencite{Mulders2003} pour calculer des bases d'ordres.

\subsection{Cas initial quand \( \sigma = 1 \)}

Si \( \sigma = 1 \), alors 

$$
\begin{aligned}
    (F, \sigma) & = \{ v \in \F[x]^{1 \x m}  ~|~  vF = 0 \mod x^1 \} \\
    &= \{ v \in \F^{1 \x m}  ~|~  vF = 0 \}
\end{aligned}
$$
Soit \( F \in \F^{m \x n} \), nous cherchons une base de \( (F, 1) \).

On remarque que si 
\[
\begin{pmatrix} 
	S \\
	K 
\end{pmatrix} 
F =
\begin{pmatrix} 
	R \\
	0 
\end{pmatrix}
\text{avec } 
 R 
 \text{ de rang maximal}
\]

alors
\[
\begin{pmatrix}
	xS \\
	K 
\end{pmatrix} 
F = 
\begin{pmatrix} 
	xR \\ 
	0 
\end{pmatrix} 
= 0 \mod x
\]
ce qui implique que 
\(
\begin{pmatrix}
	xS \\
	K 
\end{pmatrix}
\text{ est une base de }
(F, 1)
\)
Il existe des candidats naturels pour \( S \) et \( K \) : \( K \)  le noyau de $F$ et \( S \) le supplémentaire de $K$. On pourra choisir le noyau \( K \) utilisant les lignes de \( F \) de plus petit degré, une façon de les calculer consiste alors à obtenir la forme échelonnée par lignes de \( F \), avec une bonne permutation de façon à choisir les lignes de plus petit degrés, ceci aura son importance par la suite.

Il faut donc calculer une décomposition:
\[
\underbrace{
    \begin{pmatrix}
        L_r & 0 \\
        G   & I_{m-r} 
    \end{pmatrix}}_{L} 
 \cdot 
\begin{bmatrix}
    e_{\tau(1)} \\
    \vdots \\
    e_{\tau(n)}
\end{bmatrix} \cdot F =  
\underbrace{
    \begin{pmatrix}
        E' \\
        0 
    \end{pmatrix}
}_{E}
\]
où \( E \) est échelonnée en ligne, \( L \) est triangulaire inférieure, \( r \)est le rang de \( E \) et \( \tau \) est une permutation.

\vspace{0.2cm}

\begin{smallalgo}{Basis}{algo:Basis}
    \LinesNumbered 
    
    \KwIn{\( F \in (\F[x]_{\leq 0})^{m \times n} \), un vecteur de décalage \( s \).}
    \KwOut{Une \( (F, 1, s) \)-base d’ordre et son degré de ligne \( s \)-décalé.}
    
    On suppose que \( s \) est croissant.\;
    
    Calculer une forme ligne échelonnée \( F = \tau \cdot L \cdot E \) avec :\;
    \Indp
    \Indm  
    \KwRet{
        \(
        \begin{pmatrix} 
            xL_r & 0 \\ 
            G    & I_{m-r} 
        \end{pmatrix},
        \quad
        \tau^{-1}s + [1_r, 0_{n-r}]
        \)
    }
\end{smallalgo}

\vspace{0.2cm}

L'algorithme Basis calcul correctement une base d'ordre de \( (F, 1) \) d'après la construction ci-dessus, l'hypothèse que \( s \) est croissant est primordiale pour obtenir une base d'ordre. On ne cherchera pas à redémontrer ce résultat.

\subsection{Algorithmes pour le cas général}

On va devoir découper le problème pour trouver une base d'ordre pour \( \sigma > 1 \). 

\begin{theoreme}
    \leavevmode\vspace{0.5\baselineskip}
    
    Soit \( P_1\), \( P_2 \) des bases d'ordre de \( (F, \sigma_1)\) et \((F, \sigma_2)  \) respectivement, \( M \in \F [x]^{m \x n} \) tel que \( P_1F = x^{\sigma_1} M\). 
    
    Alors: \( P_2 P_1 \) est une base d'ordre de \( (F, \sigma_1 + \sigma_2 ) \).
\end{theoreme}

On peut découper par pas de \( 1 \), on se doute qu'il y aura plus efficace juste après ces lignes, on présente un algorithme itératif \textbf{quadratique}.
\[ 
(F, 1) \rightarrow (F, 2) \rightarrow (F, 3) \rightarrow \cdots (F, \sigma-1) \rightarrow  (F, \sigma) 
\]

\begin{smallalgo}{M-Basis}{algo:M-Basis}
    
    \KwIn{\( F \in (\mathbb{F}[x]_{<\sigma})^{m \times n} \), un vecteur de décalage \( \vec{s} \), \( \sigma \in \mathbb{N} \)}
    \KwOut{Une \( (F, \sigma, \vec{s}) \)-base d’ordre et son degré de ligne \( \vec{s} \)}
    
    
    \( P_0 \gets \hyperref[algo:Basis]{\emph{Basis}}(F \bmod x) \)\;
    
    \For{\( k \) de \( 1 \) à \( \sigma - 1 \)}{
        \( F' \gets x^{-k} P_{k-1} F \)\tcp*[r]{Décalage du problème}
        \( M_k \gets \hyperref[algo:Basis]{\emph{Basis}}(F' \bmod x) \)\;
        \( P_k \gets M_k P_{k-1} \)\;
    }
    
    \KwRet{\( P_{\sigma -1} \)}
\end{smallalgo}

\begin{theoreme}
    La complexité de l'algorithme \hyperref[algo:Basis]{\emph{Basis}} est \( \OO(m^\omega \sigma^2) \) opérations dans le corps de base.
\end{theoreme}

\begin{remark}
    \leavevmode\vspace{0.5\baselineskip}
    
    \begin{itemize}
        \item On peut réaliser la ligne 4 en \( \OO (m^\omega \sigma)\), la ligne 5 est en \( \OO (m^\omega) \) car les coefficients sont entiers.
        \item Il faut voir la ligne 3 comme un décalage.
        \item On se sert du théorème 5.3 pour prouver la correction de l'algorithme.
    \end{itemize}
\end{remark}

On présente maintenant un algorithme sur le principe diviser-pour-régner \textbf{quasi-linéaire} qui découpe la précision en \(2\).
\[ 
(F, 1) \rightarrow (F, 2) \rightarrow (F, 4) \rightarrow \cdots \rightarrow \left(F, \frac{\sigma}{2}\right) \rightarrow  (F, \sigma) 
\]

\begin{smallalgo}{PM-Basis}{algo:PM-Basis}
    
    \LinesNumbered 
    \DontPrintSemicolon
    
    \KwIn{\( F \in (\mathbb{F}[x]_{<\sigma})^{m \times n} \), un vecteur de décalage \( \vec{s} \), \( \sigma \in \mathbb{N} \)}
    \KwOut{Une \( (F, \sigma, \vec{s}) \)-base d’ordre et son degré de ligne \( \vec{s} \)}
    
    \eIf{\( \sigma = 1 \)}{
        \KwRet{\hyperref[algo:Basis]{\emph{Basis}}\( (F \bmod x) \)}\;
    }{
        \( P_{\text{low}} \gets \textsc{PM-Basis}(F, \lfloor \sigma/2 \rfloor) \)\tcp*[r]{Premier sous-problème}
        Soit \( F' \) tel que \( P_{\text{low}} F = x^{k} F' \)\tcp*[r]{Décalage du problème}
        \( P_{\text{high}} \gets \textsc{PM-Basis}(F', \lfloor \sigma/2 \rfloor) \)\tcp*[r]{Deuxième sous-problème}
        \KwRet{\( P_{\text{high}} \cdot P_{\text{low}} \)}\tcp*[r]{Résoudre le problème original}
    }
\end{smallalgo}



\begin{theoreme}
	La complexité de l'algorithme \hyperref[algo:PM-Basis]{\emph{PM-Basis}} est \( \OO(\MM(m, \sigma) \log(\sigma)) \) opérations dans le corps de base.
\end{theoreme}
    
\begin{remark}
    \leavevmode\vspace{0.5\baselineskip}
    
    \begin{itemize}
        \item Il faut voir la ligne 5 comme un décalage.
        \item La complexité peut se calculer en écrivant l'arbre binaire associé.
        \item On se sert du théorème 5.3 pour prouver la correction de l'algorithme.
    \end{itemize}
\end{remark}

Les algorithmes que nous avons vus sont des algorithmes pour calculer des bases d'ordre et non réduire des matrices par lignes.                     % Réduction de réseaux polynomiaux                                                                   %
%                    


%%%%%%%%%%%%%%%%%%%%%%%%%%%%%%%%%%%%%%%%%%%%%%%%%%%%%%%%%%%%%%%%%%%%%%%%%%%%%%%%%%%%%%%%%%%%%%%%%%%%%%%%%%%%%%%%%%%%%%%%%%%%%%%%%%%%%%%%%%%%%%%%%%%%%%%%%%%%%%%                                % Chapitre 2 : Réduction de réseaux euclidiens et polynomiaux                                        %
%                                                                                                                                                             %
%%%%%%%%%%%%%%%%%%%%%%%%%%%%%%%%%%%%%%%%%%%%%%%%%%%%%%%%%%%%%%%%%%%%%%%%%%%%%%%%%%%%%%%%%%%%%%%%%%%%%%%%%%%%%%%%%%%%%%%%%%%%%%%%%%%%%%%%%%%%%%%%%%%%%%%%%%%%%%%
%                                                                                                                                                             %
%                                                ▗▄▄▖▗▖ ▗▖ ▗▄▖ ▗▄▄▖▗▄▄▄▖▗▄▄▄▖▗▄▄▖     ▗▄▄▄▖▗▄▄▄▖▗▄▄▄▖                                                         %
%                                               ▐▌   ▐▌ ▐▌▐▌ ▐▌▐▌ ▐▌ █  ▐▌   ▐▌ ▐▌      █    █    █                                                           %
%                                               ▐▌   ▐▛▀▜▌▐▛▀▜▌▐▛▀▘  █  ▐▛▀▀▘▐▛▀▚▖      █    █    █                                                           %
%                                               ▝▚▄▄▖▐▌ ▐▌▐▌ ▐▌▐▌    █  ▐▙▄▄▖▐▌ ▐▌    ▗▄█▄▖▗▄█▄▖▗▄█▄▖                                                         %
%                                                                                                                                                             %
%         ▗▄▄▖  ▗▄▖ ▗▖ ▗▖  ▗▖▗▖  ▗▖ ▗▄▖ ▗▖  ▗▖▗▄▄▄▖ ▗▄▖ ▗▖       ▗▖    ▗▄▖▗▄▄▄▖▗▄▄▄▖▗▄▄▄▖ ▗▄▄▖▗▄▄▄▖    ▗▄▄▖ ▗▄▄▄▖▗▄▄▄ ▗▖ ▗▖ ▗▄▄▖▗▄▄▄▖▗▄▄▄▖ ▗▄▖ ▗▖  ▗▖         %
%         ▐▌ ▐▌▐▌ ▐▌▐▌  ▝▚▞▘ ▐▛▚▖▐▌▐▌ ▐▌▐▛▚▞▜▌  █  ▐▌ ▐▌▐▌       ▐▌   ▐▌ ▐▌ █    █    █  ▐▌   ▐▌       ▐▌ ▐▌▐▌   ▐▌  █▐▌ ▐▌▐▌     █    █  ▐▌ ▐▌▐▛▚▖▐▌         %
%         ▐▛▀▘ ▐▌ ▐▌▐▌   ▐▌  ▐▌ ▝▜▌▐▌ ▐▌▐▌  ▐▌  █  ▐▛▀▜▌▐▌       ▐▌   ▐▛▀▜▌ █    █    █  ▐▌   ▐▛▀▀▘    ▐▛▀▚▖▐▛▀▀▘▐▌  █▐▌ ▐▌▐▌     █    █  ▐▌ ▐▌▐▌ ▝▜▌         %
%         ▐▌   ▝▚▄▞▘▐▙▄▄▖▐▌  ▐▌  ▐▌▝▚▄▞▘▐▌  ▐▌▗▄█▄▖▐▌ ▐▌▐▙▄▄▖    ▐▙▄▄▖▐▌ ▐▌ █    █  ▗▄█▄▖▝▚▄▄▖▐▙▄▄▖    ▐▌ ▐▌▐▙▄▄▖▐▙▄▄▀▝▚▄▞▘▝▚▄▄▖  █  ▗▄█▄▖▝▚▄▞▘▐▌  ▐▌         %
%                                                                                                                                                             %
%                                            ▗▄▄▄▖▗▖  ▗▖▗▄▄▄▖▗▄▄▄▖ ▗▄▄▖▗▄▄▄▖▗▄▄▖      ▗▄▄▖ ▗▄▖  ▗▄▄▖▗▄▄▄▖                                                     %
%                                              █  ▐▛▚▖▐▌  █  ▐▌   ▐▌   ▐▌   ▐▌ ▐▌    ▐▌   ▐▌ ▐▌▐▌   ▐▌                                                        %
%                                              █  ▐▌ ▝▜▌  █  ▐▛▀▀▘▐▌▝▜▌▐▛▀▀▘▐▛▀▚▖    ▐▌   ▐▛▀▜▌ ▝▀▚▖▐▛▀▀▘                                                     %
%                                            ▗▄█▄▖▐▌  ▐▌  █  ▐▙▄▄▖▝▚▄▞▘▐▙▄▄▖▐▌ ▐▌    ▝▚▄▄▖▐▌ ▐▌▗▄▄▞▘▐▙▄▄▖                                                     %
%                                                                                                                                                             %
%%%%%%%%%%%%%%%%%%%%%%%%%%%%%%%%%%%%%%%%%%%%%%%%%%%%%%%%%%%%%%%%%%%%%%%%%%%%%%%%%%%%%%%%%%%%%%%%%%%%%%%%%%%%%%%%%%%%%%%%%%%%%%%%%%%%%%%%%%%%%%%%%%%%%%%%%%%%%%%
%                                                                                                                                                             %
% Pas de sous-fichiers                                                                                                                                        %
%                                                                                                                                                             %
% Le fichier parent est : report.tex                                                                                                                          %
%                                                                                                                                                             %
%%%%%%%%%%%%%%%%%%%%%%%%%%%%%%%%%%%%%%%%%%%%%%%%%%%%%%%%%%%%%%%%%%%%%%%%%%%%%%%%%%%%%%%%%%%%%%%%%%%%%%%%%%%%%%%%%%%%%%%%%%%%%%%%%%%%%%%%%%%%%%%%%%%%%%%%%%%%%%%

\chapter{Adaptation de la réduction de réseaux polynomiaux au cas des réseaux euclidiens}

Ce chapitre constitue le cœur de mon stage. Il vise à explorer l’adaptation des techniques exactes de réduction de réseaux polynomiaux, au cadre des réseaux euclidiens. Il s'agit d’esquisser des pistes, de formuler des questions pertinentes et de suggérer des idées nouvelles. L’ensemble de ce travail a été réalisé en collaboration avec mon encadrant, Romain Lebreton.


\section{Fonction potentielle et bornes}

\subsection{Les cas polynomial}

Dans l’algorithme \hyperref[algo:WeakPopovForm]{\emph{\parencite{Mulders2003}}}, chaque opération entraîne soit une diminution du degré total par lignes, soit un déplacement du pivot d’une ligne vers la gauche.

\parencite{nielsen2013gmssr} définit une fonction de \emph{valeur} pour les vecteurs :
\[
\psi : 
\begin{array}{rcl}
    \mathbb{F}[x]^{\ell+1} & \longrightarrow & \mathbb{N}_0 \\
    \mathbf{v} & \longmapsto & (\ell + 1) \cdot \rdeg \mathbf{v} + \mathrm{LP}(\mathbf{v})
\end{array}
\]

où \( \mathrm{LP}(\mathbf{v} \) désigne l'indice du pivot de \( \mathbf{v} \).

\begin{lemma}
    Soit \( \mathbf{v}_j' \) le vecteur qui remplace \( \mathbf{v}_j \) lors d’une réduction de ligne dans \hyperref[algo:WeakPopovForm]{\emph{\parencite{Mulders2003}}}.
    Alors :
    \[
    \psi(\mathbf{v}_j') < \psi(\mathbf{v}_j).
    \]
\end{lemma}

Ainsi, l’algorithme \hyperref[algo:WeakPopovForm]{\emph{\parencite{Mulders2003}}} repose sur une fonction potentielle qui décroît strictement à chaque opération. Cette fonction atteint nécessairement une valeur minimale, correspondant à une situation où aucune opération supplémentaire n’est possible, c’est-à-dire lorsque la matrice est réduite en ligne. Cette même idée se traduit dans l'algorithme.

\subsection{Le cas entier}

Le lecteur pourra utilement se référer à la démonstration de la terminaison de l’algorithme \( \mathrm{LLL} \). Dans cette preuve, une quantité \( D \) a été introduite, qui décroît d’un facteur \( \frac{3}{4}\) à chaque échange de vecteurs. Cette décroissance strictement contrôlée constitue l’invariant principal garantissant la terminaison de l’algorithme.

\begin{definition}[Rappel]
    On définit \( \displaystyle D \coloneqq \prod_{1 \leq k < n} d_k = \prod_{1 \leq k < n} \| \bb_k^*\|^{2(n-k)}\)
\end{definition}

En pratique, la valeur de \( D \) peut devenir très grande, il est donc plus pertinent de regarder l’ordre de grandeur en considérant son logarithme \( \log(D) \), c’est-à-dire son nombre de bits, puisque c'est comme ça qu'il est considérer dans la démonstration.

\begin{proposition}
    \( \displaystyle \log(D) = \sum_{k=1}^{n-1} 2(n-k) \log (\| \bb_k^*\|)\)
\end{proposition}

On impose dans la preuve de terminaison que \( D \geq 1 \). Toutefois, \( D = 1 \) si et seulement si le réseau est isomorphe à \( \mathbb{Z}^n \). Un raisonnement similaire montre que très peu de réseaux satisfont \( D = 2 \). En pratique, la valeur de \( D \) reste largement supérieure à \( 1 \), et l’algorithme retourne une base \( \mathrm{LLL} \)-réduite avec \( D \gg 1 \). Ainsi, bien que le critère de terminaison repose sur une décroissance stricte de \( D \), cela ne reflète pas toujours le fait que l’algorithme a effectivement atteint un état suffisant de réduction.

On peut essayer d'améliorer la borne inférieure, on se place dans l'hypothèse que la base est \( \mathrm{LLL} \)-réduite et donc satisfait la condition de Lovàsz.

\[
\begin{aligned}
    D & = \prod_{1 \leq k < n} \| \bb_k^*\|^{2(n-k)} \\
      & = \|\bb_1^*\|^{2(n-1)} \x \|\bb_2^*\|^{2(n-2)} \x \cdots \x \|b_n^*\|^2 \\
      & \geq \|\bb_1^*\|^{2(n-1)} \x \left(\frac{\|\bb_1^*\|}{2}\right)^{2(n-2)} \x \cdots \x \left(\frac{\|\bb_1^*\|}{2^{n-2}}\right)^2 \\
      & \geq \left( \frac{\|b_1^*\|}{2^{\frac{4}{3} (n-2)}} \right)^{(n-1)n}
\end{aligned}
\]

Après la remise de ce rapport, mon travail consistera à explorer la portée pratique de cette borne. Contrairement au cas polynomial, le comportement de \( D \) dans le cas entier reste difficile à cerner, et l’on ne sait pas précisément jusqu’où cette quantité peut décroître.

De manière similaire au raisonnement précédent, une borne supérieure sur \( D \) est également disponible.  
Dans \parencite{MCA}, on trouve :
\[
D \leq \left( \max_{1 \leq i \leq n} \| \bg_i \| \right)^{n(n-1)}.
\]
Une question naturelle est alors de savoir si cette borne peut être améliorée. On peut voir que l'exposant \( n(n-1) \) peut devenir \( (n-1)n \). Plutôt que d'utiliser \( \max_{1 \leq i \leq n} \| \bg_i \| \), on pourrait utiliser un invariant lié au réseau comme \( |\LL| \). 

\section{Vers une réduction LLL adaptée aux réseaux d’approximation}


On va donner une définition équivalente du réseau d'approximation \( (F, \sigma) \) mais adaptée pour les réseaux euclidiens.

\begin{definition}
    Soit \( F \in M_n (\Z) \), un degré de précision \( \sigma \in \N\), et \(p \in \N\). On définit
    \[
    F_{p^\sigma} \coloneqq  \{ v \in \Z^n | vF = 0 \mod p^\sigma \}
    \]
\end{definition}

Contrairement au cas polynomial, la situation est ici fondamentalement différente. Nous verrons par la suite s’il est nécessaire d’imposer des restrictions sur \( p \) et \( \sigma \), et le cas échéant, lesquelles.

\begin{proposition}
    \( F_{p^\sigma} \) est un réseau euclidien de dimension \( n \).
\end{proposition}

\begin{remark}
    \( F_{p^\sigma} \) est un réseau \( p^\sigma\)-aire.
\end{remark}

Comment peut-on calculer une base de \( F_{p^\sigma} \)~? Est-il possible d’en extraire une base \( \mathrm{LLL} \)-réduite, et ce, de manière efficace~? Comme dans le cas polynomial traité pour le réseau \( (F, \sigma) \), il est d'abord nécessaire de calculer une décomposition \footnote{Il s’agit d’une généralisation de la décomposition \( P L F = U \), où \( U \) était une matrice triangulaire supérieure. Ici, on relâche cette contrainte en ne demandant que \( U \) soit échelonnée par lignes.} de la forme

\begin{equation}
\underbrace{
    \begin{bmatrix}
        e_{\tau(1)} \\
        \vdots \\
        e_{\tau(n)}
    \end{bmatrix}
}_{P}
\cdot
\underbrace{
    \begin{bmatrix}
        L_r & 0 \\
        G & I_{m-r}
    \end{bmatrix}
}_{L}
\cdot
F
=
\underbrace{
    \begin{bmatrix}
        E'\\
        0
    \end{bmatrix}
}_{E}
\end{equation}

où \(r \coloneqq \rang(F) \), \( P \) est une matrice de permutation, \( L \) est une matrice triangulaire inférieure et \( E \) est échelonnée en ligne.


\begin{smallalgo}{\textsc{PLE}$(A)$}{algo:ple}
    \KwIn{$A \in \mathbb{K}^{n \times m}$}
    \KwOut{Matrices \( P \), \( L \), \( E \) telles que \( 6.1 \) est satisfaite.}
    
    $n \gets \text{nrows}(A)$,\quad $m \gets \text{ncols}(A)$\;
    $P \gets I_n$, \quad $L \gets I_n$, \quad $E \gets A$\;
    
    \Pour{$i \gets 0$ \KwTo $m{-}1$}{
        $(\text{pivot}, i_{\text{pivot}}) \gets \textsc{Pivot}(E_{*,i}, \{i, \dots, n{-}1\})$\;
        
        \Si{$\text{pivot} = \texttt{None}$}{
            \textbf{continuer}
        }
        
        \Si{$i_{\text{pivot}} \neq i$}{
            Échanger les lignes $i$ et $i_{\text{pivot}}$ dans $P$ et $E$\;
            
            \Pour{$k \gets 0$ \KwTo $i{-}1$}{
                Échanger $L[i,k]$ et $L[i_{\text{pivot}},k]$\;
            }
        }
        
        \Pour{$j \gets i{+}1$ \KwTo $n{-}1$}{
            $s \gets E[j,i] / \text{pivot}$\;
            \Si{$s \neq 0$}{
                $E \gets E$ avec ligne $j$ \textbf{moins} $s$ fois ligne $i$\;
                $L \gets L$ avec ligne $j$ \textbf{moins} $s$ fois ligne $i$\;
            }
        }
    }
    
    \KwRet{$(P, L, E)$}
\end{smallalgo}

\begin{remark}

Comme c'est vrai pour toute permutations, il serait intéressant de choisir certaines permutations particulières, notamment qui ordonne les vecteurs par norme croissante.

\end{remark}

\begin{remark}
    J'ai mis trop d'investissement dans cet algorithme, j'en ai fais plusieurs versions, allant d'une version naïve où je calculais tous les pivots pour ordonner les lignes de façon à avoir les mineurs principaux inversibles, à une stratégie plus subtile d'échange quand nécessaire, grâce aux conseils avisés de mon encadrant. Cet algorithme fonctionne sur \( \Z_p \) pour \( p \) premier, car les pivots sont inversibles, je réfléchissais à une stratégie pour rendre cet algorithme fonctionnel  sur \( \Z_n \).
\end{remark}

\begin{theoreme}
    L'algorithme \textsc{PLE} calcule correctement une décomposition qui satisfait 6.1.
\end{theoreme}

\begin{counterexample}[Limite de l’algorithme \textsc{PLE}]
    L’algorithme \textsc{PLE} ne s’applique pas dans tous les cas. 
    
    Soit
    \[
    A =
    \begin{bmatrix}
        3 & 4 \\
        4 & 3
    \end{bmatrix}
    \in M_2(\mathbb{Z}/12\mathbb{Z}).
    \]
    Dans cet anneau, les coefficients \( 3 \) et \( 4 \) ne sont pas inversibles, ce qui empêche de procéder aux opérations de pivot nécessaires.\\
    
    Ce contre-exemple montre que la validité de l’algorithme repose sur une hypothèse cruciale : \emph{les pivots doivent être inversibles}.
\end{counterexample}



\begin{example}
    \[ \displaystyle
    \underbrace{\begin{pmatrix}
            1 & 0 & 0 & 0 \\
            -\frac{3}{4} & 1 & 0 & 0 \\
            -\frac{1}{2} & 0 & 1 & 0 \\
            0 & -\frac{1}{3} & 0 & 1
    \end{pmatrix}}_{L}
    \x
    \underbrace{
        \begin{pmatrix}
            1 & 0 & 0 & 0 \\
            0 & 0 & 1 & 0 \\
            0 & 1 & 0 & 0 \\
            0 & 0 & 0 & 1
        \end{pmatrix}
    }_{P}
    \underbrace{\begin{pmatrix}
            4 & 2 & 4 & 2 \\
            2 & 1 & 2 & 1 \\
            3 & 3 & 3 & 3 \\
            1 & 1 & 1 & 1
    \end{pmatrix}}_{F}
    =
    \underbrace{\begin{pmatrix}
            4 & 2 & 4 & 2 \\
            0 & \frac{3}{2} & 0 & \frac{3}{2} \\
            0 & 0 & 0 & 0 \\
            0 & 0 & 0 & 0
    \end{pmatrix}}_{E}
    \]
\end{example}


On en déduit donc un algorithme pour calculer une base.


De 6.1 on déduit que 

\[
\begin{bmatrix}
    e_{\tau(1)} \\
    \vdots \\
    e_{\tau(n)}
\end{bmatrix}
\cdot
\begin{bmatrix}
    L_r \cdot p^\sigma & 0 \\
    G & I_{m-r}
\end{bmatrix}
\cdot
F
=
\begin{bmatrix}
    E' \cdot p^\sigma\\
    0
\end{bmatrix}
=
\begin{bmatrix}
    0\\
    0
\end{bmatrix}
\mod p^\sigma
\]

\begin{theoreme}
    Les matrices
    \[
    \begin{bmatrix}
        L_r \cdot p^\sigma & 0 \\
        G & I_{m-r}
    \end{bmatrix}
    \cdot
    \begin{bmatrix}
        e_{\tau(1)} \\
        \vdots \\
        e_{\tau(n)}
    \end{bmatrix}
    , \quad
    \begin{bmatrix}
        I_r \cdot p^\sigma & 0 \\
        G & I_{m-r}
    \end{bmatrix}
    \cdot
    \begin{bmatrix}
        e_{\tau(1)} \\
        \vdots \\
        e_{\tau(n)}
    \end{bmatrix}
    \]
    sont deux bases de \( F_{p^\sigma}\)
\end{theoreme}

\begin{remark}
    Il vaut mieux privilégier la seconde base car les \(r\) premières lignes sont orthonormées, ce qui nous permet de déduire intuitivement que la base sera de meilleure qualité.
\end{remark}
%CA MARCHE QUE QUAND SIGMA VAUT 1
\begin{smallalgo}{\textsc{Basis}$(F, p, \texttt{mode})$}{algo:basis_modes}
    \KwIn{Une matrice \( F \in \mathbb{K}[x]^{m \times n} \), un scalaire \( p \in \mathbb{K} \), et une chaîne \texttt{mode} égale à \texttt{v1} ou \texttt{v2}}
    \KwOut{Une base transformée selon le mode choisi}
    
    $G \gets \text{copie}(F)$\;
    $(P, L, E) \gets \textsc{PLE}(G)$\;
    $r \gets \text{rang}(F)$\;
    
    \Si{\texttt{mode} = \texttt{v1}}{
        \Pour{$i \gets 0$ \KwTo $r{-}1$}{
            Multiplier la ligne $i$ de $L$ par $p$\;
        }
    }
    
    \Si{\texttt{mode} = \texttt{v2}}{
        \Pour{$i \gets 0$ \KwTo $r{-}1$}{
            Remplacer la ligne $i$ de $L$ par $p \cdot e_i$\;
            \tcp*{$e_i$ : $i$-ème vecteur de la base canonique}
        }
    }
    
    \KwRet{$L \cdot P$}
\end{smallalgo}


\begin{example}
    
    En partant de la matrice 
    \(
    F = 
    \begin{pmatrix}
        4 & 2 & 4 & 2 \\
        2 & 1 & 2 & 1 \\
        3 & 3 & 3 & 3 \\
        1 & 1 & 1 & 1
    \end{pmatrix}
    \)
    
    On peut calculer une base de \( F_{5^4}\)
    
    \[
        \begin{pmatrix}
            625 & 0 & 0 & 0 \\
            -\frac{1}{2} & 1 & 0 & 0 \\
            0 & 0 & 625 & 0 \\
            0 & 0 & -\frac{1}{3} & 1
        \end{pmatrix}
    \]
\end{example}

On peut donc maintenant définir un algorithme sur un principe diviser pour régner.  
\begin{smallalgo}{\textsc{LLL-DAC-Padique}$(F, p, \sigma)$}{algo:lll_dac_padic}
    \KwIn{$F \in \mathbb{K}^{m \times n}$, un entier premier $p$, un entier $\sigma \geq 1$}
    \KwOut{Une base "LLL-réduite?" en précision $p^\sigma$}
    
    \Si{$\sigma = 1$}{
        \KwRet{$\textsc{LLL}(\textsc{ApproximantBasis}(F, p))$}
    }
    
    $\tau \gets \left\lfloor \dfrac{\sigma + 1}{2} \right\rfloor$\;
    
    $V_1 \gets \textsc{LLL-DAC-Padique}(F, p, \tau)$ \tcp*[r]{Appel récursif sur demi-précision}
    
    $F_{\text{low}} \gets \dfrac{V_1 \cdot F}{p^\tau}$ \tcp*[r]{Mise à jour du problème}
    
    $V_2 \gets \textsc{LLL-DAC-Padique}(F_{\text{low}}, p, \sigma - \tau)$ \tcp*[r]{Appel récursif décalé}
    
    \KwRet{$V_2 \cdot V_1$}
\end{smallalgo}

Similairement à la preuve du chapitre précédent, \(V_2 V_1\) est une base de \( F_{p^\sigma} \).

\begin{problem}[\textbf{Question}]
    Dans quelle mesure le produit de matrices LLL-réduites reste-t-il lui-même LLL-réduit ? Peut-on quantifier cette propriété ?
\end{problem}

Si \( V_1 \) est une matrice orthogonale, c'est-à-dire dont les lignes sont orthonormées, alors \( V_2 V_1\) est \( \mathrm{LLL}\)-réduite.

On rappelle que \( V_1 = U_1 V_1^* \) d'après le procédé de Gram-Schmidt. En écrivant \( V_1 V_2 = V_1 V_2^* ((V_2^*)^{-1} U_2 V_2^*) \), peut-être pourront nous mieux contrôler le résultat du produit.

\section{Comprendre le rôle du shift dans le cas euclidien}

On rappel que calculer le degré de ligne décalé d'une matrice revient à calculer le degré de ligne de cette matrice multipliée à gauche par une matrice diagonale.

On pourrait ici définir une notion de décalage avec une matrice ligne-orthogonale. Cela a donné lieu à l'algorithme suivant que j'ai écris et dont j'ai prouvé la correction par moi-même.

\begin{smallalgo}{\textsc{shiftLLL}}{algo:shiftLLL}
    \KwIn{Une base \( G \) de \( \LL \), \( S^* \) une matrice ligne-orthogonale}
    \KwOut{Une base \( B \) de \( \LL \) tel que \( BS^* \) soit \( \mathrm{LLL} \)-réduite}
    
    \KwRet{ \( \mathrm{LLL}(G S^*) \cdot (S^*)^{-1} \) }
\end{smallalgo}


\begin{theoreme}
    L'algorithme \hyperref[algo:shiftLLL]{\emph{shiftLLL}} calcule correctement une base \( B \) du réseau \( \LL \) telle que \( BS^* \) soit \( \mathrm{LLL} \)-réduite.
\end{theoreme}

\begin{proof}
    On a le diagramme commutatif suivant en voyant \( S^* \) comme la matrice de passage de la base canonique \( \mathcal{C} \) à la base \( \mathcal{C'} \coloneqq S^* \). L'écriture \( \LL_{\mathcal{C}} \) où \( \mathcal{C} \) est une base représente \( \LL \) exprimée dans la base \( \mathcal{C} \).
     
    \[\begin{tikzcd}
        {\LL_{\mathcal{C}}} && {\LL_{\mathcal{C}}} \\
        \\
        && {\LL_{\mathcal{C}'}} 
        \arrow["G", from=1-1, to=1-3]
        \arrow["{GS^*}"', from=1-1, to=3-3]
        \arrow["{S^*}", from=1-3, to=3-3]
    \end{tikzcd}\]

    Comme \( LLL(GS^*) \) est une base de \( \LL_{\mathcal{C}'}\), on en déduit le diagramme commutatif suivant et donc que \( LLL(GS^*) (S^*)^{-1} \) est une base de \( \LL_{\mathcal{C}} \).
     
    \[\begin{tikzcd}
        {\LL_{\mathcal{C}}} && {\LL_{\mathcal{C}}} \\
        \\
        && {\LL_{\mathcal{C}'}}
        \arrow["{\mathrm{LLL}(GS^*)(S^*)^{-1}}", from=1-1, to=1-3]
        \arrow["{\mathrm{LLL}(GS^*)}"', from=1-1, to=3-3]
        \arrow["{S^*}", from=1-3, to=3-3]
    \end{tikzcd}\]

    On a par construction de \( \mathrm{LLL} \) que \( LLL(GS^*) (S^*)^{-1} S^*\) est \( \mathrm{LLL}\)-réduite.
\end{proof}

L'algorithme termine car \( \mathrm{LLL}\) termine et sa complexité est de l'ordre de la complexité de \( \mathrm{LLL}\).

En revenant à la décomposition précédente, en calculant une base telle que \(V_1 V_2^*\) est \( \mathrm{LLL}\)-réduite, c'est-à-dire \(V_1 \) est \( \mathrm{LLL}\)-réduite pour le décalage \(V_2^*\). On peut se poser les questions suivantes qui sont des pistes à explorer :

Est-ce que \( (V_2^*)^{-1} U_2 V_2 \) est "quasi" \( \mathrm{LLL}\)-réduite ?

\begin{problem}[Hypothèse]
    \( (V_2^*)^{-1} U_2 V_2 \) est une base propre, c'est-à-dire qu'en écrivant sa décomposition de Gram-Schmidt, les coefficients correspondant ne sont pas trop grand.
\end{problem}

La condition de Lovász peut être interprétée comme une forme de \(2\)-quasi-croissance.  
Une question naturelle serait alors de se demander si, par analogie, le produit calculé par LLL pourrait satisfaire une propriété de \(4\)-quasi-croissance.        % Chapitre 3 : Adaptation de la réduction de réseaux polynomiaux au cas entier                       %
%                                                                                                                                                             %
%%%%%%%%%%%%%%%%%%%%%%%%%%%%%%%%%%%%%%%%%%%%%%%%%%%%%%%%%%%%%%%%%%%%%%%%%%%%%%%%%%%%%%%%%%%%%%%%%%%%%%%%%%%%%%%%%%%%%%%%%%%%%%%%%%%%%%%%%%%%%%%%%%%%%%%%%%%%%%%
%                                                                                                                                                             %
%                                                   ▗▄▄▖▗▖ ▗▖ ▗▄▖ ▗▄▄▖▗▄▄▄▖▗▄▄▄▖▗▄▄▖     ▗▄▄▄▖▗▖  ▗▖                                                          %
%                                                  ▐▌   ▐▌ ▐▌▐▌ ▐▌▐▌ ▐▌ █  ▐▌   ▐▌ ▐▌      █  ▐▌  ▐▌                                                          %
%                                                  ▐▌   ▐▛▀▜▌▐▛▀▜▌▐▛▀▘  █  ▐▛▀▀▘▐▛▀▚▖      █  ▐▌  ▐▌                                                          %
%                                                  ▝▚▄▄▖▐▌ ▐▌▐▌ ▐▌▐▌    █  ▐▙▄▄▖▐▌ ▐▌    ▗▄█▄▖ ▝▚▞▘                                                           %
%                                                                                                                                                             %
%                            ▗▖    ▗▄▖▗▄▄▄▖▗▄▄▄▖▗▄▄▄▖ ▗▄▄▖▗▄▄▄▖ ▗▄▄▖    ▗▄▄▖▗▖  ▗▖    ▗▄▄▖ ▗▄▄▄▖▗▖    ▗▄▖▗▄▄▄▖▗▄▄▄▖ ▗▄▖ ▗▖  ▗▖ ▗▄▄▖                           %
%                            ▐▌   ▐▌ ▐▌ █    █    █  ▐▌   ▐▌   ▐▌       ▐▌ ▐▌▝▚▞▘     ▐▌ ▐▌▐▌   ▐▌   ▐▌ ▐▌ █    █  ▐▌ ▐▌▐▛▚▖▐▌▐▌                              %
%                            ▐▌   ▐▛▀▜▌ █    █    █  ▐▌   ▐▛▀▀▘ ▝▀▚▖    ▐▛▀▚▖ ▐▌      ▐▛▀▚▖▐▛▀▀▘▐▌   ▐▛▀▜▌ █    █  ▐▌ ▐▌▐▌ ▝▜▌ ▝▀▚▖                           %
%                            ▐▙▄▄▖▐▌ ▐▌ █    █  ▗▄█▄▖▝▚▄▄▖▐▙▄▄▖▗▄▄▞▘    ▐▙▄▞▘ ▐▌      ▐▌ ▐▌▐▙▄▄▖▐▙▄▄▖▐▌ ▐▌ █  ▗▄█▄▖▝▚▄▞▘▐▌  ▐▌▗▄▄▞▘                           %
%                                                                                                                                                             %
%                                        ▗▖  ▗▖ ▗▄▖▗▄▄▄▖     ▗▄▄▖▗▄▄▄▖▗▖  ▗▖▗▄▄▄▖▗▄▄▖  ▗▄▖▗▄▄▄▖▗▄▖ ▗▄▄▖  ▗▄▄▖                                                 %
%                                        ▐▛▚▖▐▌▐▌ ▐▌ █      ▐▌   ▐▌   ▐▛▚▖▐▌▐▌   ▐▌ ▐▌▐▌ ▐▌ █ ▐▌ ▐▌▐▌ ▐▌▐▌                                                    %
%                                        ▐▌ ▝▜▌▐▌ ▐▌ █      ▐▌▝▜▌▐▛▀▀▘▐▌ ▝▜▌▐▛▀▀▘▐▛▀▚▖▐▛▀▜▌ █ ▐▌ ▐▌▐▛▀▚▖ ▝▀▚▖                                                 %
%                                        ▐▌  ▐▌▝▚▄▞▘ █      ▝▚▄▞▘▐▙▄▄▖▐▌  ▐▌▐▙▄▄▖▐▌ ▐▌▐▌ ▐▌ █ ▝▚▄▞▘▐▌ ▐▌▗▄▄▞▘                                                 %
%                                                                                                                                                             %
%%%%%%%%%%%%%%%%%%%%%%%%%%%%%%%%%%%%%%%%%%%%%%%%%%%%%%%%%%%%%%%%%%%%%%%%%%%%%%%%%%%%%%%%%%%%%%%%%%%%%%%%%%%%%%%%%%%%%%%%%%%%%%%%%%%%%%%%%%%%%%%%%%%%%%%%%%%%%%%
%                                                                                                                                                             %
% Pas de sous-fichiers                                                                                                                                        %
%                                                                                                                                                             %
% Le fichier parent est : report.tex                                                                                                                          %
%                                                                                                                                                             %
%%%%%%%%%%%%%%%%%%%%%%%%%%%%%%%%%%%%%%%%%%%%%%%%%%%%%%%%%%%%%%%%%%%%%%%%%%%%%%%%%%%%%%%%%%%%%%%%%%%%%%%%%%%%%%%%%%%%%%%%%%%%%%%%%%%%%%%%%%%%%%%%%%%%%%%%%%%%%%%


             % chapitre 4 : Réseaux définis par relations plutôt que par générateurs                              %
%                                                                                                                                                             %
%%%%%%%%%%%%%%%%%%%%%%%%%%%%%%%%%%%%%%%%%%%%%%%%%%%%%%%%%%%%%%%%%%%%%%%%%%%%%%%%%%%%%%%%%%%%%%%%%%%%%%%%%%%%%%%%%%%%%%%%%%%%%%%%%%%%%%%%%%%%%%%%%%%%%%%%%%%%%%%
%                                                                                                                                                             %
%            ▗▄▄▖ ▗▄▖ ▗▖  ▗▖ ▗▄▄▖▗▖   ▗▖ ▗▖ ▗▄▄▖▗▄▄▄▖ ▗▄▖ ▗▖  ▗▖    ▗▄▄▄▖▗▄▄▄▖    ▗▄▄▖ ▗▄▄▄▖▗▄▄▖  ▗▄▄▖▗▄▄▖ ▗▄▄▄▖ ▗▄▄▖▗▄▄▄▖▗▄▄▄▖▗▖  ▗▖▗▄▄▄▖ ▗▄▄▖               %
%           ▐▌   ▐▌ ▐▌▐▛▚▖▐▌▐▌   ▐▌   ▐▌ ▐▌▐▌     █  ▐▌ ▐▌▐▛▚▖▐▌    ▐▌     █      ▐▌ ▐▌▐▌   ▐▌ ▐▌▐▌   ▐▌ ▐▌▐▌   ▐▌     █    █  ▐▌  ▐▌▐▌   ▐▌                  %
%           ▐▌   ▐▌ ▐▌▐▌ ▝▜▌▐▌   ▐▌   ▐▌ ▐▌ ▝▀▚▖  █  ▐▌ ▐▌▐▌ ▝▜▌    ▐▛▀▀▘  █      ▐▛▀▘ ▐▛▀▀▘▐▛▀▚▖ ▝▀▚▖▐▛▀▘ ▐▛▀▀▘▐▌     █    █  ▐▌  ▐▌▐▛▀▀▘ ▝▀▚▖               %
%           ▝▚▄▄▖▝▚▄▞▘▐▌  ▐▌▝▚▄▄▖▐▙▄▄▖▝▚▄▞▘▗▄▄▞▘▗▄█▄▖▝▚▄▞▘▐▌  ▐▌    ▐▙▄▄▖  █      ▐▌   ▐▙▄▄▖▐▌ ▐▌▗▄▄▞▘▐▌   ▐▙▄▄▖▝▚▄▄▖  █  ▗▄█▄▖ ▝▚▞▘ ▐▙▄▄▖▗▄▄▞▘               %
%                                                                                                                                                             %
%%%%%%%%%%%%%%%%%%%%%%%%%%%%%%%%%%%%%%%%%%%%%%%%%%%%%%%%%%%%%%%%%%%%%%%%%%%%%%%%%%%%%%%%%%%%%%%%%%%%%%%%%%%%%%%%%%%%%%%%%%%%%%%%%%%%%%%%%%%%%%%%%%%%%%%%%%%%%%%
%                                                                                                                                                             %
% Pas de sous-fichiers                                                                                                                                        %
%                                                                                                                                                             %
% Le fichier parent est : report.tex                                                                                                                          %
%                                                                                                                                                             %
%%%%%%%%%%%%%%%%%%%%%%%%%%%%%%%%%%%%%%%%%%%%%%%%%%%%%%%%%%%%%%%%%%%%%%%%%%%%%%%%%%%%%%%%%%%%%%%%%%%%%%%%%%%%%%%%%%%%%%%%%%%%%%%%%%%%%%%%%%%%%%%%%%%%%%%%%%%%%%%

\chapter*{Conclusion et perspectives}

\addcontentsline{toc}{chapter}{Conclusion}

Ce stage a été l’occasion de m’immerger pleinement dans une thématique à la frontière de plusieurs domaines, en explorant les liens entre la réduction de bases dans les réseaux polynomiaux et euclidiens. J'ai appris beaucoup de choses sur la cryptographie. L’objectif initial, comprendre dans quelle mesure certaines techniques exactes issues du cas polynomial peuvent être transposées au cas entier, m’a conduit à une réflexion profonde sur les structures internes de ces deux cadres, leurs obstacles respectifs, et c'est difficile, j'ai perdu littéralement des cheveux.

Après une première phase dédiée à la compréhension fine de l’algorithme de Lenstra–Lenstra–Lovász (LLL), de sa preuve et de ses variantes, j’ai étudié les réseaux d’approximation dans le cas polynomial, en particulier à travers les algorithmes de type \textsc{Basis}, \textsc{M-Basis} et \textsc{PM-Basis}. Cela m’a permis de me familiariser avec des notions comme les fonctions potentielles, les décalages, et les stratégies de réduction modulaire.

Dans un second temps, j’ai tenté d’adapter certaines de ces idées au cadre euclidien. J’ai notamment proposé une version \(p\)-adique de réduction de type LLL, fondée sur une stratégie récursive par précision croissante, et prouvé la validité d’un algorithme que j’ai nommé \textsc{shiftLLL}, inspiré d’une vision géométrique du changement de base. Ces constructions, bien qu’exploratoires, constituent des pistes sérieuses pour une adaptation plus fine des techniques de réduction exactes aux contraintes du calcul numérique.

Ce stage m’a permis d’approfondir mes compétences en mathématiques théoriques et en algorithmique constructive, tout en me confrontant à des problématiques de mise en œuvre concrète, notamment via SageMath. J’ai également eu l’opportunité de présenter certains résultats à l’oral, d’échanger avec des chercheurs du domaine, et de développer une plus grande autonomie dans la conduite d’un projet de recherche. Ce travail ouvre naturellement la voie à de nombreuses questions, que ce soit du point de vue théorique (analyse fine de la stabilité des produits de bases réduites) ou algorithmique (optimisation, généralisation, complexité).

Plusieurs directions naturelles s’ouvrent à la suite de ce travail. Un premier axe concerne l’analyse fine du comportement de la fonction potentielle \( D \) dans le cadre euclidien : mieux comprendre ses bornes effectives, son lien avec la qualité de la base en sortie de LLL, et la possibilité d’adapter dynamiquement ses paramètres à des situations spécifiques. 

Un second axe réside dans l’étude approfondie des algorithmes de type \textsc{LLL-DAC-Padique}. Leur structure récursive offre un cadre intéressant pour l’optimisation par précision croissante, mais leur stabilité, leur efficacité réelle et la propagation des erreurs méritent une étude plus systématique, notamment en lien avec les bornes théoriques et le comportement empirique.

Enfin, une piste originale, initiée ici, concerne l’exploration de stratégies de réduction adaptées à un décalage donné. L’algorithme \textsc{shiftLLL} en constitue une première illustration. Il serait pertinent d’approfondir les conditions sous lesquelles un produit de matrices \( \mathrm{LLL} \)-réduites reste lui-même bien réduit, et de développer des critères mesurant le degré de "quasi-réduction" d’une base obtenue par composition. Ces réflexions pourraient notamment s’articuler autour de notions telles que la \( k \)-quasi-croissance ou la structure spectrale du produit de Gram-Schmidt associé.

Lors de la présentation orale du stage, j’envisage de compléter ce travail théorique par des simulations numériques illustrant concrètement les algorithmes développés.

Je tiens à remercier vivement mon encadrant Romain Lebreton pour ses conseils, sa bienveillance et la richesse des échanges tout au long de ce stage.                       % Conclusion et perspectives                                                                         %
%                                                                                                                                                             %
\include{appendix}                                       % Annexes — démonstrations, rappels supplémentaires                                                  %
%                                                                                                                                                             %
\printbibliography                                       % Bibliographie (gérée avec biblatex + biber)                                                        %
%                                                                                                                                                             %
\end{document}                                                                                                                                                %
%                                                                                                                                                             %
%%%%%%%%%%%%%%%%%%%%%%%%%%%%%%%%%%%%%%%%%%%%%%%%%%%%%%%%%%%%%%%%%%%%%%%%%%%%%%%%%%%%%%%%%%%%%%%%%%%%%%%%%%%%%%%%%%%%%%%%%%%%%%%%%%%%%%%%%%%%%%%%%%%%%%%%%%%%%%%