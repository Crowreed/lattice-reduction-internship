%%%%%%%%%%%%%%%%%%%%%%%%%%%%%%%%%%%%%%%%%%%%%%%%%%%%%%%%%%%%%%%%%%%%%%%%%%%%%%%%%%%%%%%%%%%%%%%%%%%%%%%%%%%%%%%%%%%%%%%%%%%%%%%%%%%%%%%%%%%%%%%%%%%%%%%%%%%%%%%
%                                                                                                                                                             %
%                                                ▗▄▄▖▗▄▄▄▖ ▗▄▄▖▗▄▄▄▖▗▄▄▄▖ ▗▄▖ ▗▖  ▗▖    ▗▄▄▄▖        ▗▄▄▄▖                                                    %
%                                               ▐▌   ▐▌   ▐▌     █    █  ▐▌ ▐▌▐▛▚▖▐▌      █            █                                                      %
%                                                ▝▀▚▖▐▛▀▀▘▐▌     █    █  ▐▌ ▐▌▐▌ ▝▜▌      █            █                                                      %
%                                               ▗▄▄▞▘▐▙▄▄▖▝▚▄▄▖  █  ▗▄█▄▖▝▚▄▞▘▐▌  ▐▌    ▗▄█▄▖        ▗▄█▄▖                                                    %
%                                                                                                                                                             %
%                ▗▄▄▄▖▗▖ ▗▖ ▗▄▄▖▗▖   ▗▄▄▄▖▗▄▄▄ ▗▄▄▄▖ ▗▄▖ ▗▖  ▗▖    ▗▖    ▗▄▖▗▄▄▄▖▗▄▄▄▖▗▄▄▄▖ ▗▄▄▖▗▄▄▄▖    ▗▄▄▖  ▗▄▖  ▗▄▄▖▗▄▄▄▖ ▗▄▄▖ ▗▄▄▖                       %
%                ▐▌   ▐▌ ▐▌▐▌   ▐▌     █  ▐▌  █▐▌   ▐▌ ▐▌▐▛▚▖▐▌    ▐▌   ▐▌ ▐▌ █    █    █  ▐▌   ▐▌       ▐▌ ▐▌▐▌ ▐▌▐▌     █  ▐▌   ▐▌                          %
%                ▐▛▀▀▘▐▌ ▐▌▐▌   ▐▌     █  ▐▌  █▐▛▀▀▘▐▛▀▜▌▐▌ ▝▜▌    ▐▌   ▐▛▀▜▌ █    █    █  ▐▌   ▐▛▀▀▘    ▐▛▀▚▖▐▛▀▜▌ ▝▀▚▖  █  ▐▌    ▝▀▚▖                       %
%                ▐▙▄▄▖▝▚▄▞▘▝▚▄▄▖▐▙▄▄▖▗▄█▄▖▐▙▄▄▀▐▙▄▄▖▐▌ ▐▌▐▌  ▐▌    ▐▙▄▄▖▐▌ ▐▌ █    █  ▗▄█▄▖▝▚▄▄▖▐▙▄▄▖    ▐▙▄▞▘▐▌ ▐▌▗▄▄▞▘▗▄█▄▖▝▚▄▄▖▗▄▄▞▘                       %
%                                                                                                                                                             %
%%%%%%%%%%%%%%%%%%%%%%%%%%%%%%%%%%%%%%%%%%%%%%%%%%%%%%%%%%%%%%%%%%%%%%%%%%%%%%%%%%%%%%%%%%%%%%%%%%%%%%%%%%%%%%%%%%%%%%%%%%%%%%%%%%%%%%%%%%%%%%%%%%%%%%%%%%%%%%%
%                                                                                                                                                             %
% Pas de sous-fichiers                                                                                                                                        %
%                                                                                                                                                             %
% Le fichier parent est : lattice_basics.tex                                                                                                                  %
%                                                                                                                                                             %
%%%%%%%%%%%%%%%%%%%%%%%%%%%%%%%%%%%%%%%%%%%%%%%%%%%%%%%%%%%%%%%%%%%%%%%%%%%%%%%%%%%%%%%%%%%%%%%%%%%%%%%%%%%%%%%%%%%%%%%%%%%%%%%%%%%%%%%%%%%%%%%%%%%%%%%%%%%%%%%
%                          



\chapter{Réseaux euclidiens}

\lettrine{\textbf{L}}{'étude} des réseaux euclidiens puise ses racines dans les travaux mathématiques du XVIII\textsuperscript{e} siècle, notamment ceux de Leonhard Euler sur l'organisation géométrique des points dans l'espace. C'est cependant en 1891 qu'Hermann Minkowski établit véritablement les fondements modernes avec l'introduction de la théorie géométrique des nombres, reliant explicitement les réseaux à divers problèmes d'optimisation et de minimisation. Ses résultats joueront ultérieurement un rôle déterminant dans le développement de la cryptographie moderne. Néanmoins, c'est au cours du XX\textsuperscript{e} siècle, et plus particulièrement à partir des années 1990, que les réseaux euclidiens connaissent une véritable intégration dans la cryptographie. Les travaux de chercheurs tels que Ajtai, Dwork ou Regev marquent alors un tournant décisif, en démontrant que les difficultés algorithmiques intrinsèques aux réseaux peuvent constituer une base solide pour la conception de nouveaux systèmes cryptographiques résistants aux attaques conventionnelles et quantiques. Ce chapitre vise précisément à introduire de manière approfondie les concepts fondamentaux liés aux réseaux euclidiens.
                                                                                                                                   %
\vspace{-0.1cm}                                                                                         %
\section{Définitions et exemples}
\vspace{-0.1cm}
Nous considérons un espace euclidien, c’est-à-dire un espace vectoriel réel de dimension finie muni d’un produit scalaire, noté $\langle \bf, \bg \rangle$. Dans ce chapitre, nous utiliserons le produit scalaire usuel défini par $\langle \bf, \bg \rangle := \bf \cdot \bg^t = \sum_{i=1}^{n} \bf_i \bg_i$, lequel induit la norme-$2$ donnée par $\|\bf\|_2 = \sqrt{\sum_{i=1}^{n} \bf_i^2}$. Pour alléger les notations, nous omettrons l’indice $2$. Ce chapitre se contentera d’exposer les résultats classiques, sans chercher à les redémontrer. Rappelons qu’au sein de $\R^n$, toutes les normes sont équivalentes, ce qui signifie qu’aucune ne change la nature intrinsèque des problèmes que nous aborderons.  Nous nous concentrerons néanmoins sur la norme-$2$ pour sa simplicité géométrique, car elle offre une mesure intuitive des longueurs et des angles, point essentiel pour étudier la structure des réseaux euclidiens. Afin de simplifier l’écriture, nous désignerons par $\mathcal{B} = (\bb_i)_{1 \leq i \leq n}$ une base de $\R^n$ et par $\mathcal{B}^* = (\bb^*_i)_{1 \leq i \leq n}$ sa base orthogonalisée par le procédé de Gram-Schmidt. Les matrices $B$ et $B^*$ (sans calligraphie) seront composées (en ligne) des vecteurs de la famille $(\bb_i)_{1 \leq i \leq n}$ et $(\bb^*_i)_{1 \leq i \leq n}$ respectivement.

\vspace{0.2cm}

On commence par introduire la notion de réseau euclidien. Pour un rappel sur les groupes ou les modules le lecteur est invité à lire les annexes correspondantes. Le point de vue de module nous sera utile dans la suite de ce manuscrit.

\vspace{-0.2cm}

\begin{definition}
	Soit $n \in \N^*$, $\bb_1, \ldots, \bb_n \in \R^n$. Les définitions suivantes sont équivalentes.\footnote{Plusieurs définitions équivalentes d’un réseau euclidien coexistent dans la littérature, selon que l’on se place ou non dans un espace euclidien $\R^n$, ou dans un espace $\R^n$ équipé explicitement d’une forme quadratique définie positive. Dans tous les cas, l’idée générale reste la même : un réseau euclidien est un sous-ensemble discret de $\R^n$ formé par toutes les combinaisons linéaires entières d’un ensemble de vecteurs générateurs.}
	\begin{itemize}
		\item[$\bullet$] Un \textbf{réseau euclidien} $\LL$ est un sous-groupe discret additif de $\R^n$.
        \begin{itemize}
            \item Sous-groupe additif : $\mathbf{0} \in \LL$, et pour tout $\bx, \by \in \LL$, $\bx + \by, -\bx \in \LL$
            \item Discret : $\forall \bx \in \LL, \exists \varepsilon > 0$ tel que $\mathcal{B}(\bx, \varepsilon) \cap \LL = \{\bx\}$ (où $\mathcal{B}(\bx, \varepsilon)$ désigne la boule ouverte de rayon $\varepsilon$ centrée en $\bx$).
        \end{itemize}
		\item[$\bullet$] Un \textbf{réseau euclidien} $\LL$ est un $\Z$-module libre de type fini de $\R^n$.
	\end{itemize}
\end{definition}

\begin{definition}
    Un \textbf{sous-réseau} $\LL'$ de $\LL$ est un sous groupe de $\LL$, on notera $\LL' \subseteq \LL$.
\end{definition}

\begin{example}
    Les entiers de Gauss, définit par $\Z[i]:= \Z \oplus i \Z$ forment un réseau de rang $2$ dans $\C$, c'est même un anneau.
\end{example}

\begin{example}
    Un exemple plus exotique, $\Z \oplus \sqrt{2} \cdot \Z$ est un réseau de rang $2$ dans $\R$.
\end{example}

\begin{counterexample}
    $\Q$ n'est pas un réseau euclidien , car $\Q$ est dense dans $\R$, ce qui brise la discrétude, bien que ce soit un sous-groupe de $\R$.
\end{counterexample}

En particulier on peut montrer qu'il existe une famille $\Z$-libre maximale $(\bb_i)_{1 \leq i \leq m}$ dans $\LL$ telle que 
$$\LL = \bigoplus\limits_{1 \leq i \leq m} \Z \bb_i:=\{a_1\bb_1 + \cdots a_m\bb_m : a_i \in \Z\}$$
Cette famille est appelée \textbf{base} de $\LL$, si on note $B$ la matrice de la famille $(\bb_i)_{1 \leq i \leq m}$ on notera $\LL(B)$ le réseau de base $B$, donc \textbf{engendré} par la famille $(\bb_i)_{1 \leq i \leq m}$. L'entier $m$ est commun à toutes les bases de $\LL$ et on l'appelle \textbf{rang} de $\LL$. Lorsque $n=m$, on dit que le réseau est de \textbf{rang plein}.

\begin{example}
     On a la suite d'inclusions $2\Z \subset \Z \subset \frac{1}{2}\Z$. \footnote{On rappelle que pour \( a \in \R \), on a \( a\Z = \{ an \in \R | n \in \Z \} \).} Bien que $\rang(2\Z) = \rang(\Z) = \rang(\frac{1}{2}\Z)$, ces ensembles sont distincts. Cela montre que, contrairement aux espaces vectoriels, avoir une relation d'inclusion et avoir le même rang ne suffit pas à garantir l'égalité des réseaux.
\end{example}

Existe-t-il une notion de \textbf{bonne} base ? Nous verrons qu'une base idéale est celle qui est en un sens la plus orthogonale possible. Il n'existe pas toujours de base strictement orthogonale, ce qui justifie la notion de quasi-orthogonalité. Nous allons rajouter des façons de mesurer la qualité d'une base dans le chapitre suivant et introduire la notion de réduction, qui consistera à trouver une telle base.

\begin{figure}[h]
    \centering
    \begin{subfigure}[b]{0.25\textwidth}
        \centering
        \includegraphics[width=\textwidth]{images/lattice_0_2.png}
        \caption{$B = \begin{pmatrix}
                3 & -1 \\ 
                2 & 0
            \end{pmatrix}$.}
    \end{subfigure}
    \hspace{0.05\textwidth}
    \begin{subfigure}[b]{0.25\textwidth}
        \centering
        \includegraphics[width=\textwidth]{images/lattice_1_2.png}
        \caption{$B' = \begin{pmatrix}
                1 & 1 \\ 
                1 & -1
            \end{pmatrix}$}
    \end{subfigure}
    \caption{Deux bases pour le même réseau de $\R^2$.}
    \label{fig:bases_lattice}
\end{figure}

\begin{definition}
    La \textbf{taille}\footnote{\textbf{taille} et \textbf{volume} sont synonymes, mais l'usage le plus courant dans la littérature anglaise est "déterminant du réseau" ou "volume".} d'un réseau $\LL(B)$ est $|\det(B)|$ et est noté $|\LL|$. La taille d'un réseau est indépendante de la base choisie.
\end{definition} 



\begin{remark}
    On peut définir de façon équivalente $|\LL| := \sqrt{ \det( \Gram(B))}$ où $\Gram(B) = B^t \cdot B$. On peut voir $|\LL|$ comme le \textbf{volume du domaine fondamental} de $\left\{ \sum\limits_{i=0}^{n} \lambda_i \bb_i \mid 0 \leq \lambda_i < 1 \right\}.$
\end{remark}
\begin{theoreme}[Inégalité d'Hadamard]
    Soit $B \in M_n(\K)$ et $\LL(B)$ un réseau.
    Alors 
    $$|\LL|=|\det(B)| \leq \prod_{i=1}^{n} \| \bb_i \|$$
    
    La borne est atteinte si, et seulement si $(\bb_i)_{1 \leq i \leq n}$ est une famille orthogonale.	
\end{theoreme}
Toutes les bases d'un réseau euclidien diffèrent d'une transformation de déterminant $\pm 1$. L'ensemble de ces transformations est connu sous le nom de groupe unimodulaire. Un ensemble de points non alignés dans un réseau ne constitue pas une base si son déterminant est différent de $\pm |\LL|$.

\begin{proposition}
    Soit $\LL$ et $\LL'$ deux réseaux de rang $n$ de base $B$ et $B'$. Alors $\LL$ = $\LL'$ si et seulement si il existe $ U \in \mathrm{GL}_n(\Z)$ tel que $B'=BU$. Où $\mathrm{GL}_n(\Z)= \{M \in \Z^{n \x n} | \det (M) = \pm 1\}$.
\end{proposition}

\begin{remark}
	On a l'action de groupe
	\begin{align*}
		\mathrm{GL}_n(\Z) \x \mathrm{GL}_n(\R) &\longrightarrow \mathrm{GL}_n(\R) \\
		(U, \mathbf{B}) &\longmapsto \mathbf{B}U
	\end{align*}	
	Un réseau est exactement une orbite de cette action. Nous verrons dans le chapitre suivant que la réduction de réseaux consiste à trouver un bon représentant pour chaque orbite.
\end{remark}

Nous présentons maintenant deux \textbf{invariants} fondamentaux d'un réseau : 

\begin{itemize}
	\item[$\bullet$] La \textbf{taille du vecteur minimal} du réseau, notée $\lambda_1(\LL)$.
	\item[$\bullet$] Le \textbf{volume du réseau}, aussi appelé la \textbf{taille du réseau} souvent désigné par $|\LL|$.
\end{itemize}



\begin{proposition}
	Soit $\LL$, $\LL'$  deux réseaux de $\R^n$ tel que $\LL' \subseteq \LL$ alors $\frac{|\LL'|}{|\LL|} \in \N$.
\end{proposition}

\begin{remark}
	Ce résultat est une conséquence directe du théorème de Lagrange en théorie des groupes.
\end{remark}


\begin{definition}
	On appelle \textbf{minimum d'un réseau} \footnote{on peut aussi le définir comme $\lambda_1(\LL) = \min\{r>0 : |\mathcal{B}(r)\cap\LL|>1\} \in \R_+$} $\LL$ la quantité	
	\[
	\lambda_1(\LL) = \min_{\substack{v \in \LL \\ v \neq 0}} \|v\|\]
\end{definition}

Plus généralement, pour $k \in \{1, \ldots, n\}$, on pose $\lambda_k(\LL)$ le plus petit réel $r$ tel qu'il existe $k$ vecteurs $\R$-linéairement indépendants dans $\LL$ de norme au plus $r$.

\begin{remark}
	$\lambda_1(\LL)$ correspond à la distance minimale entre deux points quelconques de $ \LL $.
\end{remark}

\begin{example}
	Soit $\LL$ le réseau de la figure ~\ref{fig:bases_lattice}. On a $|\LL| = 2$ et $\lambda_1( \LL) = \lambda_2( \LL) = \sqrt{2}$.
\end{example}

\begin{theoreme}[Premier théorème de Minkowski]
	Pour tout $n \in \N^*$, il existe une constante $C_n > 0$ telle que pour tout réseau $\LL$ de $\R^n$, on a :
	
	\[
	\lambda_1(\LL) \leq C_n |\LL|^{1/n}
	\]
	

\end{theoreme}
	On peut prendre cette constante égale à $C_n=(2/\sqrt{\pi}) \Gamma(n/2+1)^{1/n}$. \footnote{La fonction \(\Gamma\), appelée fonction gamma, généralise la notion de factorielle aux nombres réels (et complexes). Elle est définie, pour tout \(z \in \mathbb{C}\), par la formule suivante : \[
    \Gamma(z) = \int_0^{+\infty} t^{z-1} e^{-t} \, dt.
    \]}
On appelle \textbf{constante de Hermite-Minkowski} le carré de la constante optimale possible pour cette inégalité, noté $\gamma _n$, en particulier, on a $\gamma _n \leq C_n$.
En développant, on a 
\[
\gamma_n = \sup_{\dim(\LL) = n} \gamma (\LL) , \quad  \text{ où }\gamma (\LL) := \frac{\lambda _1 (\LL) ^2}{\det (\LL) ^{2/n}}
\]

\begin{proposition}
    \[
    \gamma_n = 4 \left(\frac{ \Delta _n}{V_n}\right)^{2/n} \quad \forall n \in \N^*
    \]
    où \( V_n = \frac{\pi ^{n/2}}{\Gamma(\frac{n}{2} + 1)} \) et \( \Delta_n \) est la densité d'un empilement compact \footnote{Un empilement compact d'une collection d'objets est un agencement de ces objets de telle sorte qu'ils occupent le moins d'espace possible.} de densité maximum d'hypersphères.
\end{proposition} % expliquer c'est quoi un empilement comapct

On ne connaît pas la valeur exacte de \( \gamma_n \) pour tout \( n \). Seuls les valeurs dans le tableau suivant sont connues de manières exactes.\footnote{
    \textbf{En dimension} $n = 1$, n'importe quel réseau de dimension 1 atteint cette borne.  
    \textbf{En dimension} $n = 2$, le réseau optimal est celui des entiers d’Eisenstein, également appelé réseau en nid d’abeille.  
    \textbf{En dimension} $n = 3$, le réseau optimal est présenté dans \parencite{SP3}, mais sa démonstration  nécessite environ 130 pages de minimisation de fonctions analytiques.  
    Des avancées majeures ont été obtenues pour les \textbf{dimensions} $n = 8$ et $n = 24$ grâce à la mathématicienne ukrainienne Maryna Viazovska, qui a démontré l’optimalité du réseau $E_8$ \parencite{SP8} et, en collaboration, celle du réseau de Leech \parencite{SP24}.
}

\begin{figure}[h]
    \centering
    \begin{tabular}{|c|c|c|c|c|c|c|c|c|c|}
        \hline
        $n$ & $1$ & $2$ & $3$ & $4$ & $5$ & $6$ & $7$ & $8$ & $24$ \\
        \hline
        $\gamma_n^n$ & $1$ & $\frac{4}{3}$ & $2$ & $4$ & $8$ & $\frac{64}{3}$ & $64$ & $256$ & $4^{24}$ \\
        \hline
    \end{tabular}

    \caption{Valeurs connues de $ \gamma _n$}
\end{figure}

On sait que \( ( \gamma _n)_n \) est une suite d'ordre de croissance linéaire mais on ne sait pas si elle est croissante.

\begin{example}
	Le \textbf{réseau d'Eisenstein}, ou \textbf{réseau en nid d'abeille}, est un réseau de \(\R^2\) de rang $2$, engendré par la base 
    \(
    \begin{pmatrix}
		1 & \frac{1 + \sqrt{3}}{2} \\
		1 & \frac{1 - \sqrt{3}}{2}
	\end{pmatrix} \). 
	On a \(|\LL| = \sqrt{3},  \lambda_1 = \sqrt{2}, \gamma(\LL) = \frac{2}{\sqrt{3}}\).
    En traçant des sphères de centre les points du réseaux et de rayon \(\sqrt{2}\), on obtient l'empilement compact le plus dense en dimension $2$.
	
	\begin{figure}[h]
		\centering
		\includegraphics[scale=0.4]{images/hexagonal_lattice.png}
		\caption{Empilement hexagonal dans \(\R^2\)}
	\end{figure}
\end{example}

\vspace{-1cm}

\begin{proposition}
	Soit $\LL \subset \R^n$ un réseau de base $(\bb_i)_{1 \leq i \leq n}$ et sa base de Gram-Schmidt associée $(\bb_i^*)_{1 \leq i \leq n}$.
	Alors pour tout $\bv \in \LL \backslash \{\mathbf{0}\}$, on a $$\lambda_1(\LL) \geq \|\bv\| \geq \min_{1 \le i \le n} \|\bb_i^*\|$$ 

\end{proposition}

Tous ces résultats et inégalités vont nous donner des critères de mesure de la qualité de réduction d'une base dans le chapitre sur la réduction de réseaux euclidiens.

\section{Réseaux définis par relations plutôt que par générateurs}

\begin{definition}
    Soit $\LL \subset \R^n$ un réseau de base $(\bb_i)_{1 \leq i \leq n}$. Le \textbf{dual} de \( \LL \) est défini par
    \[
    \LL^\vee:=\{ \bx \in \R^n ~|~ \forall \by \in \LL, \langle \bx, \by \rangle \in \Z\}.
    \]
\end{definition}

\begin{proposition}
    On a $\LL = \LL(\bb_1, \ldots, \bb_m)$ si et seulement si $\LL^\vee= \LL(\bb_1^\vee, \ldots, \bb_n^\vee)$,
    
    où \( \bb_i^\vee \) vérifie \( \left \langle \bb_i^\vee, \bb_i \right \rangle = \delta_{i,j}\). \footnote{ \( \delta_{i,j} \) est le symbole de Kronecker, il vaut \( 1 \) si \( i = j \) et \( 0 \) sinon.}
\end{proposition}

\begin{proposition}
    $\LL^\vee$ est un réseau de base : 
    
    \begin{itemize}
        \item [$\bullet$] $(B^t)^{-1}$ lorsque le réseau est de rang plein.
        \item[$\bullet$] $ B(B^t B)^{-1} $ lorsque le réseau n'est pas de rang plein.
    \end{itemize}
\end{proposition}

\begin{proposition} De la proposition précédente découle les propriétés suivantes :
    \begin{itemize}
        \item[$\bullet$] $\rang(\LL) = \rang(\LL^\vee)$.
        \item[$\bullet$] $|\LL^\vee| = |\LL|^{-1}$.
        \item[$\bullet$] \( (\LL ^\vee)^\vee = \LL \)
    \end{itemize}
\end{proposition}

\begin{definition}
    On dit que \( \LL \) est \textbf{auto-dual} si \( \LL = \LL^\vee \).
\end{definition}

\begin{proposition} On a les propriétés suivantes :
    \begin{itemize}
        \item[$\bullet$] $(a \LL)^\vee = \frac{1}{a} \LL^\vee$ pour tout $a \in \R^*$.
        \item[$\bullet$] $(\Z u)^\vee= \frac{1}{ \|u\|} \Z u$ pour tout $u \in \R^m \backslash \{ \mathbf{0} \}$.
    \end{itemize}
\end{proposition}

\begin{example}
    Le réseau dual de $\Z^n$ est $\Z^n$ et est donc auto-dual.
\end{example}

\begin{figure}[h]
    \centering
    \begin{subfigure}[b]{0.25\textwidth}
        \centering
        \includegraphics[width=\textwidth]{images/lattice_0_5.png}
        \caption{\( 2 \Z^2 \)}
    \end{subfigure}
    \hspace{0.05\textwidth}
    \begin{subfigure}[b]{0.25\textwidth}
        \centering
        \includegraphics[width=\textwidth]{images/lattice_1_5.png}
        \caption{\( \frac12 \Z^2  \)}
    \end{subfigure}
    \caption{Un réseau et son dual.}
    \label{fig:dual}
\end{figure}


\begin{proposition}
    Soit $\LL_1$, $\LL_2$ des réseaux, alors
    $$(\LL_1 \oplus \LL_2)^\vee = \LL_1^\vee \oplus \LL_2^\vee$$
\end{proposition}

\begin{lemma}
    Soit $\LL$ un réseau de dimension $n$. On a
    \begin{itemize}
        \item[$\bullet$] $\lambda_1(\LL) \cdot \lambda_1(\LL^\vee) \leq n$,
        \item[$\bullet$] $\lambda_1(\LL) \cdot \lambda_n(\LL^\vee) \geq 1$.
    \end{itemize}
\end{lemma}

\parencite{Banaszczyk1993} a démontré une relation encore plus forte entre les minima d'un réseau et ceux de son dual, connue sous le nom de théorème de transfert.

\begin{theoreme}[Théorème de transfert]
    Soit $\LL$ un réseau de dimension $n$. On a
    \[
    1 \leq \lambda_1(\LL) \cdot \lambda_n(\LL^\vee) \leq n.
    \]
\end{theoreme}

\begin{definition}
    On définit le réseau euclidien \( A_n \subset \R^{n+1} \) par :
    \[A_n = \left\{ (x_1, \dots, x_{n+1}) \in \Z^{n+1} \;\middle|\; \sum_{i=1}^{n+1} x_i = 0 \right\}\]
\end{definition}

\begin{example}
    On a :
    \begin{align*}
        A_0 &= \{ 0 \} \subset \R, \\
        A_1 &= \{ (x, -x) \in \Z^2 \}, \text{ donc } A_1 \text{ est de rang } 1\\
        A_2 &= \{ (x, y, z) \in \Z^3 \mid x + y + z = 0 \}, \\
        &= \langle a_1, a_2 \rangle \quad \text{où } 
        a_1 = (1, -1, 0), \quad a_2 = (0, 1, -1), \\
        &\text{donc } A_2 \text{ est de rang } 2.
    \end{align*}
\end{example}

\begin{proposition}
    Pour tout $n \in \N$, \( A_n \) a pour matrice génératrice :
    
    \[
    B_n := 
    \begin{pmatrix}
        1 & 0 & 0 & \cdots & 0 \\
        -1 & 1 & 0 & \cdots & 0 \\
        0 & -1 & 1 & \cdots & 0 \\
        \vdots & \vdots & \ddots & \ddots & \vdots \\
        0 & 0 & \cdots & -1 & 1 \\
    \end{pmatrix} 
    \in M_n(\Z)
    \]
\end{proposition}

\begin{proposition}
    \( A_n \) est un réseau euclidien de rang \( n \), pour tout \( n \in \N \).
\end{proposition}

Il existe une classe particulière de réseaux qui joue un rôle important en cryptographie.
\begin{definition}[Réseau $q$-aire]
    Un réseau $\LL$ est un réseau $q$-aire si 
    \[
    q\mathbb{Z}^n \subseteq \LL \subseteq \mathbb{Z}^n.
    \]
\end{definition}

 
Étant donnée une matrice $\mathbf{A} \in \Z_q^{m \times n}$ pour certains entiers $n, m, q \in \N$, on peut définir deux réseaux :

\[
\LL_q(\mathbf{A}) = \left\{ \by \in \Z^m : \by = \mathbf{A}\mathbf{s} \bmod q \text{ pour un certain } \mathbf{s} \in \Z^n \right\}
\]

\[
\LL_q^\perp(\mathbf{A}^T) = \left\{ \by \in \Z^m : \mathbf{A}^T \by = 0 \bmod q \right\}
\]

Les deux réseaux sont de dimension $m$. Le premier est engendré par les lignes de $\mathbf{A}$ et a pour déterminant $q^{m-n}$, tandis que le second contient tous les vecteurs orthogonaux aux lignes de $\mathbf{A}$ et a pour déterminant $q^n$.

De plus, ils sont liés par la dualité des réseaux, c’est-à-dire :
\[
\LL_q^\perp(\mathbf{A}^T) = q \cdot \LL_q(\mathbf{A})^\vee \quad \text{et} \quad \LL_q(\mathbf{A}) = q \cdot \LL_q^\perp(\mathbf{A}^T)^\vee.
\]

\section{Quelques problèmes algorithmiques liés aux réseaux euclidiens}

Cette section s’inspire largement des travaux de \textcite{Boudgoust2023}, auxquels le lecteur intéressé pourra se référer pour un traitement plus approfondi. On définira deux problèmes algorithmiques important sur les réseaux euclidiens. Il y en a beaucoup plus, \parencite{stephens-davidowitz_latticeproblems} donne un aperçu des réductions entre ces problèmes.

\subsection{Des problèmes faciles}

Certains problèmes liés aux réseaux euclidiens sont relativement simples à résoudre, notamment la vérification de l’appartenance d’un vecteur à un réseau donné, ou encore la décision de l’égalité de deux bases de réseaux. Le lecteur intéressé pourra s’exercer en tentant de résoudre ces problèmes.

\begin{problem}[\textbf{Adhésion}]
    Étant donné une base $B$ d’un réseau $\LL$, et $\bv \in \R^n$, décider si $\bv \in \LL$.
\end{problem}

\begin{problem}[\textbf{Équivalence}]
    Étant donné deux bases $B$ et $B'$, décider si $\LL (B) = \LL (B')$,
\end{problem}

\subsection{Le problème du vecteur le plus court}


Considérons le problème suivant, paramétré par la dimension $n$ du réseau :


\begin{problem}[\textbf{Shortest Vector Problem (SVP)}, \textbf{NP-complet}, \parencite{Ajtai1996}.]
    Étant donné une base $B$ d’un réseau $\LL$, trouver un vecteur $\bv \neq \mathbf{0}$ tel que $\|\bv\|_2 = \lambda_1(\LL)$.
\end{problem}

On ne connaît que des algorithmes demandant au moins un nombre exponentiel d'opérations pour résoudre ce problème, même en utilisant des algorithmes quantiques. Les algorithmes de type \textbf{énumération} \footnote{ils énumèrent tous les vecteurs du réseau qui sont dans une certaine boule bien choisie, en pratique ils sont utilisés jusqu'aux dimensions $n \approx 80$. On peut leur ajouter des optimisations et des heuristiques.} et les algorithmes de type \textbf{crible} \footnote{on génère deux listes d'éléments du réseau, puis on construit la liste de toutes les différences entre les éléments des deux listes. On espère obtenir des vecteurs plus court. On recommence le procédé. Le temps d'exécution est en $2^{\OO(n)}$.} se démarquent pour ce problème. Le calcul d'un plus court vecteur dans un réseau euclidien de $\R^n$ est en général un problème difficile qui sert de fondation à de nombreuses primitives cryptographiques. On s’intéresse souvent à la version approximative :


\begin{problem}[\textbf{SIVP$_\gamma$}, où $\gamma > 0$]
    Étant donné une base $B$ du réseau $\LL$, trouver des vecteur $\bv_1, \ldots \bv_n \neq \mathbf{0}$ linéairement indépendants tel que $\|\bv_i\|_2  \leq \gamma \cdot \lambda_i(\LL)$ pour tout \( i \).
\end{problem}

\begin{figure}[H]
    \centering
    \includegraphics[width=0.25\textwidth]{images/lattice_0_8.png}
    \caption{ Une instance de SIVP. }
    \label{fig:sivp}
\end{figure}

L’état des connaissances actuelles est le suivant :

\begin{itemize}
	\item[$\bullet$] Pour $\gamma = \OO(1)$, le problème est prouvé \textbf{NP-complet}, \parencite{Ajtai1996}. %pas d'algo exp ?
	\item[$\bullet$] Pour $\gamma = \text{poly}(n)$, il existe des algorithmes en \textbf{temps exponentiel}.
	\item[$\bullet$] Pour $\gamma = 2^{\OO(n)}$, l’algorithme \textbf{LLL} \parencite{Lenstra1982} permet de le résoudre en \textbf{temps polynomial}.
\end{itemize}

\begin{problem}[\textbf{GapSVP$_\gamma$}, où $\gamma > 0$]
    Étant donné une base $B$ du réseau $\LL$, et $r \in \R_+^*$. Décider si \( \lambda_1(\LL) \leq r \) (instance positive) ou \( \lambda_1(\LL) > \gamma \cdot r \) (instance négative).
\end{problem}

\begin{theoreme}[\cite{Banaszczyk1993},\cite{stephens-davidowitz_latticeproblems}].
    
    $\mathrm{SVP}_\gamma$ n'est pas plus simple que $\mathrm{GapSVP}_\gamma$.
\end{theoreme}

\begin{conjecture}
    Il n'existe aucun algorithme classique ou quantique en temps polynomial qui approxime les problèmes de réseaux 
    $\mathrm{SVP}_\gamma$, $\mathrm{GapSVP}_\gamma$ ou à un facteur polynomial près $\gamma$ (pour tous les réseaux d'entrée possibles).
\end{conjecture}

\subsection{Le problème du vecteur le plus proche}
Un autre problème important concerne la recherche de vecteurs proches d'une cible dans un réseau.

\begin{problem}[\textbf{Closest Vector Problem (CVP)}, \textbf{NP-complet}, \parencite{Ajtai1996}.]
    Étant donnés $t \in \R^n$, un réseau $\LL(B)$, trouver $\bv \in \LL$ tel que 
    $$\displaystyle \|t - \bv\|_2  = d(t, \LL) := \min_{ v \in \LL } \{ \|t - \bv\|_2  \}.$$
\end{problem}


Le problème $\mathrm{CVP}$ est en général difficile pour un réseau arbitraire. Cependant, pour certaines familles spécifiques de réseaux, comme $\Z^n$, des algorithmes en temps polynomial sont connus. La qualité de la base choisie joue un rôle crucial dans la résolution du problème. De même, on peut considérer une version approximative :


\begin{problem}[\textbf{CVP$_\gamma$}, $\gamma > 0$]
    Étant donnés $t \in \R^n$, un réseau $\LL(B)$, trouver $\bv \in \LL$ tel que 
    \(
    \|t - \bv\|_2  \leq \gamma \cdot d(t, \LL).
    \)
\end{problem}

\begin{problem}[\textbf{GapCVP$_\gamma$}, $\gamma > 0$]
    Étant donné \( r \in \R_+^* \), $t \in \R^n$, un réseau $\LL(B)$.
    Décider si il existe \( \bv \in \LL \) tel que \( \|t - \bv\|_2  \leq r \) (instance positive) ou \( \|t - \bv\|_2  > \gamma \cdot r \) (instance négative) 
\end{problem}

\begin{theoreme}[\cite{Goldreich1999}].
    
    $\mathrm{GapSVP}_\gamma$ se réduit à $\mathrm{GapCVP}_\gamma$ en temps polynomial.
\end{theoreme}


\begin{theoreme}
	Il existe un algorithme qui résout \textbf{CVP$_{\exp(n)}$} en temps polynomial via l'algorithme \textbf{LLL}.
\end{theoreme}

L'efficacité des algorithmes dépend grandement de la qualité de la base du réseau euclidien choisie. Le chapitre sur la réduction abordera des techniques pour améliorer la base, via l'algorithme \textbf{LLL}.\footnote{En dimension fixée, résoudre \textbf{exactement} le problème \textbf{SVP} pour la norme $\|\cdot\|_\infty$ fournit en fait une $\gamma$-approximation (avec $\gamma$ dépendant de la dimension) pour le problème \textbf{SVP} dans la norme $\|\cdot\|_2$. Il existe notamment une constante $C$ telle que $\|v\|_2 \leq C \|v\|_\infty$ pour tout $v \in \R^n$. Ainsi, un vecteur minimisant $\|v\|_\infty$ donne un vecteur $\sqrt{n}$-proche du vecteur réellement le plus court en norme euclidienne.}
