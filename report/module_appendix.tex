%%%%%%%%%%%%%%%%%%%%%%%%%%%%%%%%%%%%%%%%%%%%%%%%%%%%%%%%%%%%%%%%%%%%%%%%%%%%%%%%%%%%%%%%%%%%%%%%%%%%%%%%%%%%%%%%%%%%%%%%%%%%%%%%%%%%%%%%%%%%%%%%%%%%%%%%%%%%%%%
%                                                                                                                                                             %
%                                                    ▗▄▖ ▗▄▄▖ ▗▄▄▖ ▗▄▄▄▖▗▖  ▗▖▗▄▄▄ ▗▄▄▄▖ ▗▄▄▖▗▄▄▄▖                                                            %
%                                                   ▐▌ ▐▌▐▌ ▐▌▐▌ ▐▌▐▌   ▐▛▚▖▐▌▐▌  █  █  ▐▌   ▐▌                                                               %
%                                                   ▐▛▀▜▌▐▛▀▘ ▐▛▀▘ ▐▛▀▀▘▐▌ ▝▜▌▐▌  █  █  ▐▌   ▐▛▀▀▘                                                            %
%                                                   ▐▌ ▐▌▐▌   ▐▌   ▐▙▄▄▖▐▌  ▐▌▐▙▄▄▀▗▄█▄▖▝▚▄▄▖▐▙▄▄▖                                                            %
%                                                                                                                                                             %
%                                                          ▗▖  ▗▖ ▗▄▖ ▗▄▄▄ ▗▖ ▗▖▗▖   ▗▄▄▄▖                                                                    %
%                                                          ▐▛▚▞▜▌▐▌ ▐▌▐▌  █▐▌ ▐▌▐▌   ▐▌                                                                       %
%                                                          ▐▌  ▐▌▐▌ ▐▌▐▌  █▐▌ ▐▌▐▌   ▐▛▀▀▘                                                                    %
%                                                          ▐▌  ▐▌▝▚▄▞▘▐▙▄▄▀▝▚▄▞▘▐▙▄▄▖▐▙▄▄▖                                                                    %
%                                                                                                                                                             %
%%%%%%%%%%%%%%%%%%%%%%%%%%%%%%%%%%%%%%%%%%%%%%%%%%%%%%%%%%%%%%%%%%%%%%%%%%%%%%%%%%%%%%%%%%%%%%%%%%%%%%%%%%%%%%%%%%%%%%%%%%%%%%%%%%%%%%%%%%%%%%%%%%%%%%%%%%%%%%%
%                                                                                                                                                             %
% Pas de sous-fichiers                                                                                                                                        %
%                                                                                                                                                             %
% Le fichier parent est : general_algebra_review.tex                                                                                                          %
%                                                                                                                                                             %
%%%%%%%%%%%%%%%%%%%%%%%%%%%%%%%%%%%%%%%%%%%%%%%%%%%%%%%%%%%%%%%%%%%%%%%%%%%%%%%%%%%%%%%%%%%%%%%%%%%%%%%%%%%%%%%%%%%%%%%%%%%%%%%%%%%%%%%%%%%%%%%%%%%%%%%%%%%%%%%

\chapter{Rappels sur les modules}
\label{appendix:modules}

\section{Généralités}

\begin{definition}
	Soit $A$ un anneau commutatif unitaire. Un \textbf{$A$-module} $(M, +, \cdot)$ est un ensemble $M$ :
	\begin{itemize}
		\item muni d’une loi interne $+$ faisant de $(M,+)$ un groupe abélien,
		\item muni d’une loi externe $A \x M \to M,\ (a,m)\mapsto am$, satisfaisant pour tout
		$a,b \in A$ et $m,m' \in M$ :
		\begin{enumerate}
			\item Distributivité : $a(m + m') = a m + a m'$,
			\item Distributivité : $(a + b) m = a m + b m$,
			\item Associativité : $(a b) m = a(b m)$,
			\item Neutre : $1 \cdot m = m$ (où $1$ est l’élément neutre de $A$).
		\end{enumerate}
	\end{itemize}
\end{definition}

\begin{remark}
	Cette définition généralise la notion d’espace vectoriel, en remplaçant le \emph{corps} des scalaires par un \emph{anneau} commutatif. Dans le cas d’un corps, tout élément non nul de $A$ est inversible, tandis qu’ici on n’exige pas cette propriété.
\end{remark}
\begin{example}
	Soit $n \in \N^*$. L'ensemble $\Z^n$, muni de l'addition vectorielle et de la multiplication par un scalaire entier, est un $\Z$-module. En effet :
	\begin{itemize}
		\item $(\Z^n, +)$ est un groupe abélien,
		\item pour tout $a \in \Z$ et $x = (x_1, \dots, x_n) \in \Z^n$, on a $a \cdot x = (a x_1, \dots, a x_n) \in \Z^n$,
		\item toutes les axiomes d'un module sont satisfaits (associativité, distributivité, existence d'un neutre).
	\end{itemize}
	C'est un module libre de type fini, de base canonique $(e_1, \dots, e_n)$.
\end{example}

\begin{counterexample}
	Considérons l'ensemble $\R$ et l'anneau $\C$. On cherche à définir une structure de $\C$-module sur $\R$ via la multiplication usuelle :
	\[
	\forall \alpha \in \C,\ \forall r \in \R,\quad \alpha \cdot r := \alpha r.
	\]
	Cependant, cette loi externe n'est pas bien définie car elle n'est pas \textbf{fermée} :
	\[
	\text{par exemple, } \alpha = i \in \C,\ r = 1 \in \R \quad \Rightarrow \quad \alpha \cdot r = i \notin \R.
	\]
	Donc $\R$ n’est pas stable par multiplication par les scalaires complexes. Il ne peut pas être muni d’une structure de $\C$-module.
\end{counterexample}


\begin{table}[h]
	\centering
	\caption{Exemples et contre-exemples de modules}
	\begin{tabular}{|c|c|c|}
		\hline
		\textbf{Ensemble considéré} & \textbf{Sur quel anneau $A$} & \textbf{Module ?} \\
		\hline
		$\Z^n$ & $\Z$ & \checkmark \\
		$\Q^n$ & $\Q$ & \checkmark \\
		$\Z/n\Z$ & $\Z$ & \checkmark \\
		$C^0([0,1], \R)$ & $\R$ & \checkmark \\
		$\R[x]$ & $\R$ & \checkmark \\
		$\Q$ & $\Z$ & \xmark \\
		$\Z[\sqrt{2}]$ & $\Z[x]$ & \xmark \\
		$\R$ & $\C$ & \xmark \\
		$\Z^n$ & $\Q$ & \xmark \\
		\hline
	\end{tabular}
\end{table}

\subsection{Sous-modules, type fini et modules libres}

\begin{definition}
	Soit $M$ un $A$-module. Un \textbf{sous-module} $N \subseteq M$ est un sous-groupe additif de $(M,+)$ qui est stable par multiplication externe, c’est-à-dire pour tout $a\in A$ et tout $x\in N$, on a $ax \in N$.
\end{definition}

\begin{definition}
	Un $A$-module $M$ est de \textbf{type fini} s’il existe un ensemble fini $S \subset M$ tel que tout élément de $M$ s’écrive comme combinaison $A$-linéaire des éléments de $S$. On dit alors que $S$ engendre $M$.
\end{definition}

\begin{definition}
	Un $A$-module $M$ est \textbf{libre} s’il admet une famille $(x_i)_{i\in I}$ telle que tout $x \in M$ s’écrive de manière unique sous la forme
	\[
	x \;=\; \sum_{i\in I} \alpha_i\, x_i,
	\]
	avec $\alpha_i \in A$. Cette famille $(x_i)$ est appelée \textbf{base} de $M$.
\end{definition}

Si $M$ est libre et de type fini, il existe donc une base finie de $M$. La démonstration n'est pas triviale. Dans ce cas, toutes les bases de $M$ ont le même nombre d’éléments, que l’on appelle le \textbf{rang} de $M$.

\begin{example}.
	\begin{itemize}
		\item $\Z/n\Z$ est un \(\Z\)-module de type fini, mais il n’est pas libre pour $n \neq 0$, car aucun élément non nul de $\Z/n\Z$ n’est librement générateur.
		\item Au contraire, $\Z^n$ est libre de rang $n$ (les vecteurs de la base canonique en constituent une base).
	\end{itemize}
\end{example}

\begin{table}[h!]
	\centering
	\begin{tabular}{|c|c|c|}
		\hline
		\textbf{Libre} & \textbf{Type fini} & \textbf{Exemple} \\
		\hline
		Oui & Oui &  $\Z^n$ \\
		\hline
		Oui & Non &  $\bigoplus_{i \in \N} \Z$ \\
		\hline
		Non & Oui &  $\Z/n\Z$ pour $n \geq 2$ \\
		\hline
		Non & Non &  $\Q$ \\
		\hline
	\end{tabular}
	\caption{Exemples de $\Z$-modules selon leur liberté et leur type fini}
\end{table}

\section{Modules sur un anneau principal}

Dans la suite, on considère un anneau principal $A$, c’est-à-dire un anneau (commutatif unitaire) dans lequel \emph{tout idéal} est principal.

\begin{theoreme}
	Soit $M$ un module \textbf{libre} sur un anneau principal $A$. Alors tout sous-module de $M$ est également libre et son rang est inférieur ou égal à celui de $M$.
\end{theoreme}

\begin{example}
	Dans le $\Z$-module libre $\Z^2$, considérons le sous-ensemble
	\[
	N \;=\; \{(a,b)\in \Z^2 \mid a \equiv b \pmod{10}\}.
	\]
	On constate que $N$ est un \textbf{sous-module} de $\Z^2$, et qu’il est engendré par les vecteurs $(1,1)$ et $(0,10)$. Ceux-ci sont $A$-linéairement indépendants, donc $N$ est libre de rang $2$, tout comme $\Z^2$ lui-même.
\end{example}

\begin{table}[h!]
	\centering
	\renewcommand{\arraystretch}{1.2}
	\begin{tabular}{|c|c|c|c|c|c|}
		\hline
		\textbf{Module $M$} & \textbf{Anneau $A$} & \textbf{Libre} & \textbf{Type fini} & \textbf{Torsion} & \textbf{Réf.} \\
		\hline
		$\K[x]^n$ & $\K[x]$ & Oui & Oui & Non & (1) \\
		$\K[x]/(x^n)$ & $\K[x]$ & Non & Oui & Oui & (2) \\
		$\K[x]^\infty$ & $\K[x]$ & Oui & Non & Non & (3) \\
		$\K(x)$ & $\K[x]$ & Non & Non & Oui & (4) \\
		$(\K[x])^n / (x \cdot \K[x])^n$ & $\K[x]$ & Non & Oui & Oui & (5) \\
		\hline
	\end{tabular}
	\caption{Exemples de modules sur l’anneau $\K[x]$}
\end{table}

\vspace{1em}
\noindent
\textbf{Commentaires sur les exemples :}
\begin{enumerate}
	\item \(\K[x]^n\) est le module libre canonique : c’est un \(\K[x]\)-module libre de rang \(n\). Il sert de modèle aux réseaux polynomiaux.
	\item \(\K[x]/(x^n)\) est un module de torsion : tout élément est annulé par une puissance de \(x\). Il n’admet pas de base libre.
	\item \(\K[x]^\infty\) (somme directe infinie) est un module libre, mais non de type fini. Il possède une base infinie indexée par \(\N\).
	\item \(\K(x)\), le corps des fractions rationnelles, est un module divisible mais non libre. Il contient des éléments sans expression unique comme combinaison de base.
	\item $(\K[x])^n / (x \cdot \K[x])^n$ est un module quotient, utilisé dans les algorithmes de bases d’ordre modulo $x^\sigma$. C’est un module de torsion.
\end{enumerate}
