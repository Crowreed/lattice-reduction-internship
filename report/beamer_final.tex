\documentclass{beamer}
\usepackage[english]{babel}
\usetheme{Madrid}
\usecolortheme{seahorse}
\usepackage[T1]{fontenc}
\usepackage{lmodern}
\usepackage[utf8]{inputenc}
\usepackage[french,ruled,vlined]{algorithm2e}

\SetKwInput{KwIn}{\textbf{Entrée}}
\SetKwInput{KwOut}{\textbf{Sortie}}
\SetKw{KwRet}{\textbf{Retourner}}
\SetKw{KwIf}{\textbf{Si}}
\SetKw{KwElse}{\textbf{Sinon}}
\SetKw{KwElseIf}{\textbf{Sinon si}}
\SetKw{KwFor}{\textbf{Pour}}
\SetKw{KwWhile}{\textbf{Tant que}}
\SetKw{KwDo}{\textbf{faire}}
\SetKw{KwThen}{\textbf{alors}}
\SetKw{KwEnd}{\textbf{Fin}}
\SetKw{KwTo}{\textbf{à}}

\SetKwFor{While}{\textbf{Tant que}}{\textbf{faire}}{\textbf{Fin}}
\SetKwIF{If}{ElseIf}{Else}{\textbf{Si}}{\textbf{alors}}{\textbf{Sinon si}}{\textbf{Sinon}}{\textbf{Fin}}
\SetKwFor{For}{\textbf{Pour}}{\textbf{faire}}{\textbf{Fin}}

\usepackage{enumitem}
\setlist[itemize]{topsep=2pt, partopsep=0pt, parsep=0pt, itemsep=2pt}
\setlist[enumerate]{topsep=2pt, partopsep=0pt, parsep=0pt, itemsep=2pt}
\usepackage{caption}
\usepackage{tikz}
\usepackage{tikz-cd}
\usetikzlibrary{decorations.pathreplacing,arrows.meta}
\usetikzlibrary{fit}
\usetikzlibrary{tikzmark}


\usetikzlibrary{calc}
\usepackage{ragged2e}
\usepackage{parskip}
\usepackage{relsize}

\setlength{\parindent}{0pt}
\setlength{\parskip}{2pt}

\newenvironment{smallalgo}[2]{
    \begin{center}
        \begin{minipage}{0.7\linewidth}
            \begin{algorithm}[H]
                \captionsetup{labelformat=empty}
                \caption*{\textit{#1}}
                \label{#2}
                \vspace{4pt}
                \LinesNumbered
                \DontPrintSemicolon
            }{
            \end{algorithm}
        \end{minipage}
    \end{center}
}
\usepackage[backend=biber,style=authoryear]{biblatex}
\setlength{\bibitemsep}{1\baselineskip}                                            
\addbibresource{references/references.bib}
\usepackage[most]{tcolorbox}
\tcbuselibrary{skins, breakable}
\usepackage{mdframed}
\usepackage{amsmath, amsthm, amssymb}
\usepackage{mathtools}
\usepackage{mathrsfs}
\usepackage{bm}
\newcommand{\SVP}{\mathrm{SVP}}

\newcommand{\LLL}{\mathrm{LLL}}
\newcommand{\CVP}{\mathrm{CVP}}
\newcommand{\Z}{\mathbb{Z}}
\renewcommand{\u}{\mathbf{u}}
\renewcommand{\v}{\mathbf{v}}
\newcommand{\g}{\mathbf{g}}
\renewcommand{\b}{\mathbf{b}}

\newcommand{\N}{\mathbb{N}}
\newcommand{\Q}{\mathbb{Q}}
\newcommand{\C}{\mathbb{C}}
\newcommand{\F}{\mathbb{F}}
\newcommand{\R}{\mathbb{R}}
\newcommand{\K}{\mathbb{K}}
\newcommand{\LL}{\mathscr{L}}
\newcommand{\OO}{\mathcal{O}}
\DeclareMathOperator{\rdeg}{rdeg}
\DeclareMathOperator{\MM}{MM}
\DeclareMathOperator{\M}{M}
\DeclareMathOperator{\lcoeff}{lcoeff}
\DeclareMathOperator{\rang}{rang}
\DeclareMathOperator{\size}{size}
\DeclareMathOperator{\Gram}{Gram}
\providecommand{\x}{\times}
\newcommand{\eqjust}[1]{\overset{(#1)}{=}}
\newcommand{\leqjust}[1]{\overset{(#1)}{\leq}}
\newcommand{\nint}[1]{\left\lceil #1 \right\rfloor} 
\newcommand{\ps}[2]{\left\langle #1 , #2 \right\rangle}
\newcommand{\norm}[1]{\left\| #1 \right\|}
\newcommand{\gscoeff}[2]{\frac{\ps{#1}{#2}}{\norm{#2}^2} #2}
\newtheoremstyle{definitionstyle}
{10pt}
{10pt}
{\normalfont}
{}
{\bfseries}
{.}
{1em}
{}
\theoremstyle{definitionstyle}
\newtheoremstyle{examplestyle}
{10pt}
{10pt}
{\normalfont}
{}
{\itshape}
{.}
{1em}
{}
\theoremstyle{examplestyle}
\newtheorem*{counterexample}{Contre exemple}
\newtheorem*{notation}{Notation}
\newtheorem*{remark}{Remarque}
\newtheoremstyle{propositionstyle}
{10pt}
{10pt}
{\normalfont}
{}
{\bfseries}
{.}
{1em}
{}
\usepackage[absolute,overlay]{textpos}

\title[Réduction : adaptations du cas polynomial]{Réduction de réseaux: Adaptations d'idées provenant du cas polynomial au cas entier.}
\author{Lucas Petit \\ \textbf{Encadrant}: Romain Lebreton}
\date{3 juillet 2025}

\begin{document}
    
    %%%%%%%%%%%%%%%%%%%%%%%%%%%%%%%%%%%%%%%%%%%%%%%%%%%%%%%%%%%%%%%%%%%%%%%%%%%%%%%%%%%%%%%%%%%%%%%%%%%%%%%%%%%%%%%%%%%%%%%%%%%%%%%%%%%%%%%%%%%%%%%%%%%%%%%%%%%%%%%
    
    \begin{frame} 
        \titlepage
    \end{frame}
    
    %%%%%%%%%%%%%%%%%%%%%%%%%%%%%%%%%%%%%%%%%%%%%%%%%%%%%%%%%%%%%%%%%%%%%%%%%%%%%%%%%%%%%%%%%%%%%%%%%%%%%%%%%%%%%%%%%%%%%%%%%%%%%%%%%%%%%%%%%%%%%%%%%%%%%%%%%%%%%%%
        
    \begin{frame}{Réseaux euclidiens}
        \begin{textblock*}{\paperwidth}(0.5cm, 1.5cm)
            
            \uncover<2->
            {
                Un \textbf{réseau euclidien} $\LL$ est un sous-groupe discret additif de $\R^n$.
            }
            
            \uncover<3->
            {
                $\rightarrow$ \textbf{Sous-groupe additif:}  
                \vspace{-0.2cm}
                \[
                \mathbf{0} \in \LL\text{, } \mathbf{x} + \mathbf{y} \in \LL \text{, } -\mathbf{x} \in \LL \text{ pour tout } \mathbf{x}, \mathbf{y} \in \LL. 
                \]
            }
            \vspace{-0.5cm}
            \uncover<4->
            {
                $\rightarrow$  \textbf{Discret:} Pour tout $\mathbf{x} \in \LL$, il existe $\varepsilon > 0$ tel que
                \vspace{0.1cm}
                \[
                \footnotemark[1]\mathcal{B}(\mathbf{x}, \varepsilon) \cap \LL = \{\mathbf{x}\}
                \]
                
                \footnotetext[1]{{\footnotesize  où $\mathcal{B}(\mathbf{x}, \varepsilon)$ désigne la boule ouverte de rayon $\varepsilon$ centrée en $\mathbf{x}$.}}
            }
            
           
            
            
    \end{textblock*}
    
       
    \begin{textblock*}{\paperwidth}(4.5cm, 5cm)
        
        \uncover<5->
        {
            \includegraphics[width=0.3\textwidth]{images/final_slide_2.png}
        }
    \end{textblock*}
    
    \end{frame}
    
    %%%%%%%%%%%%%%%%%%%%%%%%%%%%%%%%%%%%%%%%%%%%%%%%%%%%%%%%%%%%%%%%%%%%%%%%%%%%%%%%%%%%%%%%%%%%%%%%%%%%%%%%%%%%%%%%%%%%%%%%%%%%%%%%%%%%%%%%%%%%%%%%%%%%%%%
    
    \begin{frame}{Base d'un réseau euclidien}
        
        \begin{textblock*}{\paperwidth}(0.5cm, 1.5cm)
            \uncover<2->
            {
                Tout réseau $\LL \subseteq \R^n$ admet une famille $\Z$-libre maximale $(\b_i)_{1 \leq i \leq m}$, avec \( m \leq n \) tel que:
                \[
                \LL = \bigoplus_{i=1}^{m} \Z \b_i = \left\{ a_1 \b_1 + \cdots + a_m \b_m \mid a_i \in \Z \right\}
                \]
                
                Cette famille est appelée une \textbf{base} du réseau $\LL$.
            }
        \end{textblock*}
        
        \begin{textblock*}{\paperwidth}(4.5cm, 5cm)
            \uncover<1-2>
            {
                \includegraphics[width=0.3\textwidth]{images/final_slide_2.png}
            }
        \end{textblock*}
    
        \begin{textblock*}{\paperwidth}(4.5cm, 5cm)
            \uncover<3->
            {
                \includegraphics[width=0.3\textwidth]{images/final_slide_3.png}
            }
        \end{textblock*}
    \end{frame}
    
    %%%%%%%%%%%%%%%%%%%%%%%%%%%%%%%%%%%%%%%%%%%%%%%%%%%%%%%%%%%%%%%%%%%%%%%%%%%%%%%%%%%%%%%%%%%%%%%%%%%%%%%%%%%%%%%%%%%%%%%%%%%%%%%%%%%%%%%%%%%%%%%%%%%%%%%%%%%%%%%
    
    \begin{frame}{Deux bases différentes du même réseau}
        
        \begin{textblock*}{\paperwidth}(4.5cm, 1.3cm)
            \uncover<3->
            {
                \includegraphics[width=0.3\textwidth]{images/final_slide_4.png}
            }
        \end{textblock*}
    
        \begin{textblock*}{\paperwidth}(9cm, 2.8cm)
            \uncover<4->
            {
                Vecteurs longs
                
                Peu orthogonaux
            }
        \end{textblock*}
    
        \begin{textblock*}{\paperwidth}(4.5cm, 5cm)
            \uncover<1->
            {
                \includegraphics[width=0.3\textwidth]{images/final_slide_3.png}
            }
        \end{textblock*}
    
        \begin{textblock*}{\paperwidth}(9cm, 6.2cm)
            \uncover<2->
            {
                Vecteurs courts
                
                "Orthogonaux"
            }
        \end{textblock*}
    \end{frame}
    
    
    
    \begin{frame}{Réduction utopique et cryptographie}
    
        \pause On appelle \textbf{minimums d'un réseau} $\LL$ :	
        \[
        \lambda_1(\LL) = \min_{\substack{v \in \LL \\ v \neq 0}} \|v\|, \quad \lambda_2(\LL)= \cdots
        \]
        
        \pause \textbf{Shortest Independant Vector Problem (\( _\gamma \mathrm{SIVP} \))} \\
        
        On veut une base \( (\b_i)_{1 \leq i \leq n} \) de \( \LL \) tel que:
        \[
        \| \b_1 \| = \gamma(n) \cdot \lambda_1(\LL), \quad
        \| \b_2 \| = \gamma(n) \cdot \lambda_2(\LL), \quad \ldots, \quad
        \| \b_n \| = \gamma(n) \cdot  \lambda_n(\LL)
        \]
        
        \pause \begin{tabular}{@{}ll}
            \(\gamma(n) = \OO(1)\) & \( \mathrm{SIVP} \) — \textbf{NP-difficile} (Ajtai, 1996) \\
            \pause \(\gamma(n) = \mathrm{poly}(n)\) & \textbf{Cryptographie} à base de réseaux (Conjecturé difficile)\\
            \pause \(\gamma(n) = 2^{\OO(n)}\) & \textbf{Temps polynomial} (Lenstra, Lenstra, Lovasz, 1982) \\
        \end{tabular}
    \end{frame}



%%%%%%%%%%%%%%%%%%%%%%%%%%%%%%%%%%%%%%%%%%%%%%%%%%%%%%%%%%%%%%%%%%%%%%%%%%%%%%%%%%%%%%%%%%

    \begin{frame}{Orthogonalisation dans les \( \R \) -- espaces vectoriels}
        
        
        \begin{textblock*}{\paperwidth}(4.5cm, 1.3cm)
            \uncover<2->
            {
                \includegraphics[width=0.3\textwidth]{images/final_slide_7.png}
            }
        \end{textblock*}
        
        \begin{textblock*}{\paperwidth}(4.5cm, 1.3cm)
            \uncover<3-4>
            {
                \includegraphics[width=0.3\textwidth]{images/final_slide_8.png}
            }
        \end{textblock*}
    
        \begin{textblock*}{\paperwidth}(4.5cm, 1.3cm)
            \uncover<5->
            {
                \includegraphics[width=0.3\textwidth]{images/final_slide_9.png}
            }
        \end{textblock*}
        
        \begin{textblock*}{\paperwidth}(0.5cm, 5.3cm)
            \uncover<3->
            {
                \(\R \rightarrow \) procédé d'orthogonalisation de Gram--Schmidt.
            }
        \end{textblock*}
      
      \begin{textblock*}{\paperwidth}(0cm, 6cm)
          \uncover<7->
          {
              \[
              \underbrace
              {
                  \begin{pmatrix}
                      2 & 1 \\
                      1 & 2
                  \end{pmatrix}
              }_{\mathcal{B}}
              =
              \underbrace
              {
                  \begin{pmatrix}
                      1  & 0 \\
                      \frac{4}{5} & 1
                  \end{pmatrix}
              }_{U}
              \times
              \underbrace
              {
                  \begin{pmatrix}
                      2 & 1 \\
                      \frac{-3}{5} & \frac{6}{5}
                  \end{pmatrix}
              }_{\mathcal{B}^*}
              \]
          }
      \end{textblock*}
  
        \begin{textblock*}{\paperwidth}(0.5cm, 8cm)
            \uncover<8->
            {
            Les coefficients \( \mu_{i,j} \) sont appelés coefficients de \textbf{Gram–Schmidt}.
            }
        \end{textblock*}
  
       
       
  
        \begin{textblock*}{\paperwidth}(8.5cm, 2.3cm)
            \uncover<4->
            {
                \( b_1^* \leftarrow b_1 \)
            }
        \end{textblock*}
     
        \begin{textblock*}{\paperwidth}(8.5cm, 3.3cm)
            \uncover<5>
            {
            \( b_2^* \leftarrow b_2 - \frac{\langle b_2, b_1^* \rangle}{\|b_1^*\|^2} b_1^*\)
            }
        \end{textblock*}
        \begin{textblock*}{\paperwidth}(8.5cm, 3.3cm)
            \uncover<6->
            {
                \( b_2^* \leftarrow b_2 - \frac{4}{5} b_1^*\)
            }
        \end{textblock*}
    \end{frame}
    
    
    
    %%%%%%%%%%%%%%%%%%%%%%%%%%%%%%%%%%%%%%%%%%%%%%%%%%%%%%%%%%%%%%%%%%%%%%%%%%%%%%%%%%%%%%%%%%%%%%%%%%%%%%%%%%%%%%%%%%%%%%%%%%%%%%%%%%%%%%%%%%%%%%%%%%%%%%%%%%%%%%%
    
    \begin{frame}{Orthogonalisation dans les \( \Z \) -- espaces vectoriels}
        
        \begin{textblock*}{\paperwidth}(4.5cm, 1.3cm)
            \uncover<1-3>
            {
                \includegraphics[width=0.3\textwidth]{images/final_slide_9.png}
            }
        \end{textblock*}
    
        \begin{textblock*}{\paperwidth}(4.5cm, 1.3cm)
            \uncover<4->
            {
                \includegraphics[width=0.3\textwidth]{images/final_slide_10.png}
            }
        \end{textblock*}
    
    
        \begin{textblock*}{\paperwidth}(0.5cm, 5.3cm)
            \uncover<2->
            {
                \textbf{Problème:} \( \mathcal{B}^* \) pas une base de \( \LL(\mathcal{B}) \).
            }
        \end{textblock*}
    
    
        \begin{textblock*}{\paperwidth}(0.5cm, 5.8cm)
            \uncover<3->
            {
                \( \Z \rightarrow \) Gram--Schmidt discret.
            }
        \end{textblock*}
    
    
    
        \begin{textblock*}{\paperwidth}(0cm, 6.2cm)
            \uncover<-4>
            {
                \[
                \underbrace
                {
                    \begin{pmatrix}
                        2 & 1 \\
                        1 & 2
                    \end{pmatrix}
                }_{\mathcal{B}}
                =
                \underbrace
                {
                    \begin{pmatrix}
                        1  & 0 \\
                        \frac{4}{5} & 1
                    \end{pmatrix}
                }_{U}
                \times
                \underbrace
                {
                    \begin{pmatrix}
                        2 & 1 \\
                        \frac{-3}{5} & \frac{6}{5}
                    \end{pmatrix}
                }_{\mathcal{B}^*}
                \]
            }
        \end{textblock*}
        \begin{textblock*}{\paperwidth}(0cm, 6.2cm)
            \uncover<5->
            {
            \[
            \underbrace
            {
                \begin{pmatrix}
                    2 & 1 \\
                    -1 & 1
                \end{pmatrix}
            }_{\mathcal{B}}
            =
            \underbrace
            {
                \begin{pmatrix}
                    1  & 0 \\
                    \frac{1}{5} & 1
                \end{pmatrix}
            }_{U}
            \times
            \underbrace
            {
                \begin{pmatrix}
                    2 & 1 \\
                    \frac{-3}{5} & \frac{6}{5}
                \end{pmatrix}
            }_{\mathcal{B}^*}
            \]
        }
    \end{textblock*}

        \begin{textblock*}{\paperwidth}(0.5cm, 8.1cm)
            \uncover<6->
            {
            \( \mathcal{B} \) est dite \textbf{size-réduite} si:
            \vspace{-0.3cm}
            \[
            \max_{1 \leq j < i \leq n} |\mu_{i,j}| \leq \frac{1}{2}
            \]
        }
        \end{textblock*}
      
    \end{frame}
    
    

    
    %%%%%%%%%%%%%%%%%%%%%%%%%%%%%%%%%%%%%%%%%%%%%%%%%%%%%%%%%%%%%%%%%%%%%%%%%%%%%%%%%%%%%%%%%%%%%%%%%%%%%%%%%%%%%%%%%%%%%%%%%%%%%%%%%%%%%%%%%%%%%%%%%%%%%%%%%%%%%%%
    
    \begin{frame}{Définition: condition de Lovász}
      
        
        \begin{textblock*}{\paperwidth}(0.5cm, 1.5cm)
            
            \uncover<2->
            {
            Gram--Schmidt projette, réduit la taille des vecteurs.
            }
            
            \uncover<3->
            {
            \textbf{Utopie pour Gram--Schmidt discret:}
            \vspace{-0.2cm}
            \[
            \|\b_i^*\|^2 \leq \|\b_{i+1}^*\|^2
            \]
            \vspace{-0cm}
            \( \rightarrow \) trop difficile, on relache la contrainte
            }
        
            \uncover<4->
            {
            \textbf{Condition de Lovàsz} (\( 2 \)--quasi croissance) :
            \vspace{-0.2cm}
            \[
            \|\b_i^*\|^2 \leq 2\|\b_{i+1}^*\|^2
            \]
            }
        \end{textblock*}
        
        \begin{textblock*}{\paperwidth}(1.5cm,5.5cm)
            \uncover<6>
            {
                \includegraphics[width=0.28\textwidth]{images/final_slide_14.png}
            }
        \end{textblock*}
    
        \begin{textblock*}{\paperwidth}(1.5cm,5.5cm)
            \uncover<7->
            {
                \includegraphics[width=0.28\textwidth]{images/final_slide_15.png}
            }
        \end{textblock*}
    
    \begin{textblock*}{\paperwidth}(7.5cm,5.5cm)
        \uncover<5->
        {
        \includegraphics[width=0.4\textwidth]{images/final_lovasz.png}
    }
    \end{textblock*}
        
    \end{frame}
    
    
%%%%%%%%%%%%%%%%%%%%%%%%%%%%%%%%%%%%%%%%%%%%%%%%%%%%%%%%%%%%%%%%%%%%%%%%%%%%%%%%%%%%%%%%%%%%%%%%%%%%%%%%%%%%%%%%%%%%%%%%%%%%%%%%%%%%%%%%%%%%%
    
    \begin{frame}{Algorithme LLL(Lenstra, Lenstra, Lovàsz, 1982)}
        
        \vspace{-0.5cm}
        \pause \textbf{Définition}: \( \mathcal{B} = (\mathbf{b_i})_{1 \leq i \leq n} \) est dite \textbf{\( \LLL \)--réduite} si :
        \begin{itemize}
            \item[$\bullet$] \( \mathcal{B} \) est\textbf{ size-réduite}.
            \item[$\bullet$] \( \mathcal{B} \) satisfait la \textbf{condition de Lovász}.
        \end{itemize}
        \vspace{0.2cm}
        \pause \begin{smallalgo}{LLL (vulgarisé)}{algo:LLL}
            \KwIn{Une base \( \mathcal{B} = (\mathbf{b_i})_{1 \leq i \leq n} \) de \( \LL \).}
            \KwOut{Une base de \( \LL \) \( \mathrm{LLL} \)-réduite.}
            
            \While{\( \mathcal{B} \) n’est pas \( \mathrm{LLL} \)--réduite }
            {
            \Indp
            \textbf{size-réduire}(\( \mathcal{B} \))\;
            \textbf{Lovász}\( (\mathcal{B}) \)\;
            \Indm
            }
            \KwRet{\( \mathcal{B} \)}
        \end{smallalgo}
        \vspace{0.2cm}
        \pause \textbf{Théorème:} LLL utilise \( \displaystyle \widetilde{\OO}\left( n^5 \log^2 \left(  \max_{1 \leq i \leq n} \|\b_i\| \right) \right) \) opérations binaires.
    \end{frame}

    %%%%%%%%%%%%%%%%%%%%%%%%%%%%%%%%%%%%%%%%%%%%%%%%%%%%%%%%%%%%%%%%%%%%%%%%%%%%%%%%%%%%%%%%%%%%%%%%%%%%%%%%%%%%%%%%%%%%%%%%%%%%%%%%%%%%%%%%%%%%%%%%%%%%%%%%%%%%%%%
 
    \begin{frame}{Appropriation du domaine}
        
        \begin{itemize}
            \item[\( \bullet \)] Théorie des réseaux euclidiens
            \item[\( \bullet \)] Implémentation efficace de LLL et Gram-Schmidt (SageMath)
            \item[\( \bullet \)] Groupe de travail(interne): cryptographie à base de réseaux
            \item[\( \bullet \)] Analyse et explication vulgarisée de LLL (en anglais)
        \end{itemize}
        
    \end{frame}
    
    \begin{frame}{Piste de recherche: qualité de réduction}
        
        \begin{textblock*}{\paperwidth}(0cm,2cm)
            \uncover<2->
            {
                \centering
                \includegraphics[width=0.9\textwidth]{images/final_slide.png}
            }
        \end{textblock*}
        
        \begin{textblock*}{\paperwidth}(1.7cm,1.5cm)
            \uncover<2->
            {
                \textbf{Réseaux polynomiaux}
            }
        \end{textblock*}
    
        \begin{textblock*}{\paperwidth}(1.4cm,7.7cm)
            \uncover<3->
            {
                \( \rightarrow \) unicité de la réduction.
            }
        \end{textblock*}
    
        \begin{textblock*}{\paperwidth}(8.2cm,1.5cm)
            \uncover<2->
            {
                \textbf{Réseaux euclidiens}
            }
        \end{textblock*}
        
        \begin{textblock*}{\paperwidth}(7cm,7.7cm)
            \uncover<4->
            {
                {\small \( \rightarrow \) pas d'unicité sur la LLL-réduction.}
                
                {\small \( \rightarrow \) utilisé pour terminaison de LLL.}
            }
        \end{textblock*}
        
        \begin{textblock*}{\paperwidth}(0.5cm,8.5cm)
            \uncover<5->
            {
                \textbf{Piste}: The lower the $D$, the better?
            }
        \end{textblock*}
    \end{frame}
 
    \begin{frame}{Réseau défini par une relation}
        
        
        \pause Soit \( F \in M_n(\Z) \), un degré de précision \( \sigma \in \N \), et \( p \in \N \). On définit :
        \[
        F_{p^\sigma} \coloneqq \{ v \in \Z^n \mid vF = 0 \mod p^\sigma \}
        \]
        
        \pause \( \rightarrow \) réseau euclidien, fondamentale en cryptographie
        
        \pause \textbf{Analogue polynomial} : classe \( \mathbf{P} \), réduction exacte, approche diviser-pour-régner
        
        
        \vspace{0.5em}
        \pause \( \rightarrow \) LLL utilise \( \displaystyle \widetilde{\OO} \left( n^5 \sigma^2 \log^2 p \right) \) opérations binaires.
                 
        \pause \textbf{Problème} : comment calculer une base LLL-réduite de ce réseau ?
       
        
    \end{frame}


    
    
    \begin{frame}{Extraire une base de \( F_{p^\sigma} \)}
        
        \begin{textblock*}{\paperwidth}(0.5cm, 1.5cm)
            \uncover<2-> { \( \rightarrow \) on s'inspire du monde polynomial. }
        \end{textblock*}
        
        \begin{textblock*}{\paperwidth}(0.5cm,2cm)
            \uncover<3-> { Soit \( F \in M_n(\Z) \) de rang \( r \). On peut écrire }
        \end{textblock*}
        
        \begin{textblock*}{\paperwidth}(0cm,3cm)
            \uncover<4->
            {
                \[
                \begin{pmatrix}
                    L_r & 0 \\ G & I_{m-r}
                \end{pmatrix}
                P
                F
                =
                \begin{pmatrix}
                    E' \\ 0
                \end{pmatrix}
                \mod p
                \]
                
                
            }
        \end{textblock*}
        \begin{textblock*}{\paperwidth}(0.5cm,4.5cm)
            \uncover<4->
            {
                \( L_r \) : {\small triangulaire inférieure}, \( P \) : {\small permutation}, \( E' \) : {\small échelonnée en ligne}.
                
                \( \rightarrow \) généralisation de la décomposition PLU
            }
        \end{textblock*}
        
        
        
            \begin{textblock*}{\paperwidth}(0.5cm,7.5cm)
                \uncover<9->
                {
                    \textbf{Piste:} Choix de P. 
            
                    \( \rightarrow \) qui minimise \( D \) ?    
            
                    \( \rightarrow \) qui contrôle de la taille des entiers
                }
            \end{textblock*}
        
        \begin{textblock*}{\paperwidth}(0cm,6cm)
            \uncover<5>
            {
            \[
            \begin{pmatrix}
                L_r & 0 \\ G & I_{m-r}
            \end{pmatrix}
            P
            F
            =
            \begin{pmatrix}
                E' \\ 0
            \end{pmatrix}
            \mod p
            \]
            }
        \end{textblock*}
        
        \begin{textblock*}{\paperwidth}(0cm,6cm)
            \uncover<6>
            {
                \[
                \begin{pmatrix}
                    p \cdot L_r & 0 \\ G & I_{m-r}
                \end{pmatrix}
                P
                F
                =
                \begin{pmatrix}
                    p \cdot E' \\ 0
                \end{pmatrix}
                \mod p
                \]
            }
        \end{textblock*}
    
        \begin{textblock*}{\paperwidth}(0cm,6cm)
            \uncover<7>
            {
                \[
                \begin{pmatrix}
                    p \cdot L_r & 0 \\ G & I_{m-r}
                \end{pmatrix}
                P
                F
                =
                \begin{pmatrix}
                    0 \\ 0
                \end{pmatrix}
                \mod p
                \]
            }
        \end{textblock*}
    
        \begin{textblock*}{\paperwidth}(0cm,6cm)
            \uncover<8->
            {
                \[
                \begin{pmatrix}
                    p \cdot I_r & 0 \\ G & I_{m-r}
                \end{pmatrix}
                P
                F
                =
                \begin{pmatrix}
                    0 \\ 0
                \end{pmatrix}
                \mod p
                \]
            }
        \end{textblock*}
    \end{frame}

\begin{frame}{Calcul de la décomposition : démarche scientifique}
    
    \( \rightarrow \)  Conception, implémentation, correction (\texttt{SageMath})
        
       \( \rightarrow \) Valide sur \(\mathbb{Z}_p\), \(p\) premier.
        
       \( \rightarrow \) Complexité binaire de manipulation d'entiers dans \( \mathbb{Z}_p\) : \( \tilde{\OO} (\log p) \)
       
       \( \rightarrow \) Complexité binaire du calcul du noyau dans \( M_n(\mathbb{Z}_p) \) : \( \tilde{\OO} (n^3 \log p) \)
        
\end{frame}


\begin{frame}{Une étape de diviser-pour-régner}
   
    \begin{textblock*}{\paperwidth}(0cm,1cm)
        \uncover<2>
        {
            \centering
            \includegraphics[width=1\textwidth]{images/slide_15_1.png}
        }
    \end{textblock*}

    \begin{textblock*}{\paperwidth}(0cm,1cm)
        \uncover<3>
        {
            \centering
            \includegraphics[width=1\textwidth]{images/slide_15_0.png}
        }
    \end{textblock*}

    \begin{textblock*}{\paperwidth}(0cm,1cm)
        \uncover<4->
        {
            \centering
            \includegraphics[width=1\textwidth]{images/slide_15_2.png}
        }
    \end{textblock*}
\begin{textblock*}{\paperwidth}(0.5cm,5.5cm)
    
    \uncover<5->
    {
    \textbf{Pour quel décalage ?}
        
        \textbf{Piste 1} : \textsc{ShiftLLL($B$, $S^*$)} qui calcule \( B \) tel que \( B S^* \) est LLL-réduite
        \begin{itemize}
            \item \textbf{Théorème} : \textsc{ShiftLLL} est correct
            \item \textbf{Problème} : complexité pas plus intéressante
        \end{itemize}
        
        \textbf{Piste 2 (prometteuse)} : appliquer LLL sur \( V_2 \) avec une norme perturbée.
    }
\end{textblock*}
    
\end{frame}


\begin{frame}{Conclusion et perspectives}
    
    \pause
    \textbf{Limite du stage de recherche}
    \begin{itemize}
        \item[\( \rightarrow \)] Pistes non encore testées expérimentalement ou théoriquement.
        
    \end{itemize}
    
    \pause
    \textbf{Pistes à explorer}
    \begin{itemize}
        \item[\( \rightarrow \)] Notion de décalage à mieux comprendre et exploiter.
        \item[\( \rightarrow \)] Estimer plus précisément \( D \).
        \item[\( \rightarrow \)] Gagner de la complexité sur la taille des entiers ?
        \item[\( \rightarrow \)] Réduire le nombre d’étapes de LLL final ?
    \end{itemize}
    
    \pause
    \textbf{Questions ouvertes}
    \begin{itemize}
        \item Conjecture : \( V_2 V_1 \) est \textit{size-réduit}.
        \item \( V_2 V_1 \) est 4-quasi croissante ?
        \item Décalage de norme prometteur.
    \end{itemize}
    
\end{frame}
 
    \begin{frame}[c]
        \centering
        \Huge
        \textbf{Merci pour votre attention!}
        
        \vspace{2em}
        \LARGE
        Questions?
    \end{frame}
\end{document}
