\documentclass[a4paper,12pt]{report}  % Déclare un document de type "report" (idéal pour les rapports longs)
                                       % a4paper : Définit le format du papier en A4 (210 × 297 mm)
                                       % 12pt : Définit la taille de police principale à 12 points (standard pour les rapports)

\usepackage[utf8]{inputenc} % Permet d'écrire directement des accents et caractères spéciaux avec l'encodage UTF-8

\usepackage[T1]{fontenc}  % Définit l'encodage des polices pour une meilleure gestion des caractères
                          % T1 : Active un encodage qui améliore l'affichage des accents (é, à, ç...) 
                          %      et la césure correcte des mots en français
                          % Recommandé pour les documents en français

\usepackage{amsmath, amsthm, amssymb}   % amsmath :  Pour des équations et des                                         notations mathématiques avancées.
                                        % amsthm : Pour formater des théorèmes, définitions, propositions, etc.
                                        % amssymb : Pour ajouter des symboles mathématiques supplémentaires

% Définition d'un style pour les définitions
\newtheoremstyle{definitionstyle}  % Nom du style
  {10pt}                           % Espace avant
  {10pt}                           % Espace après
  {\normalfont}                    % Police normale
  {}                               % Retrait
  {\bfseries}                      % Titre en gras
  {.}                              % Ponctuation après le titre
  {1em}                            % Espace après le titre (ajout d'un petit                                    espace)
  {}                               % Personnalisation du titre

\theoremstyle{definitionstyle}

\newtheorem{definition}{Définition}[chapter] % Numérotation par chapitres

% Définition d'un style pour les exemples
\newtheoremstyle{examplestyle}  % Nom du style
  {10pt}                        % Espace avant
  {10pt}                        % Espace après
  {\normalfont}                 % Police normale
  {}                            % Retrait
  {\itshape}                    % Titre en italique
  {.}                           % Ponctuation après le titre
  {1em}                         % Espace après le titre
  {}                            % Personnalisation du titre

\theoremstyle{examplestyle}
\newtheorem{example}{Exemple}[chapter] % Numérotation par section

\usepackage[a4paper, left=2.5cm, right=2.5cm, top=2.5cm, bottom=2.5cm]{geometry} % Définition des marges du document (A4 avec 2.5 cm sur chaque côté)

\usepackage[backend=biber,style=authoryear]{biblatex} % Gestion de la bibliographie
\addbibresource{references.bib} % On inclue le fichier BibTeX references

\usepackage{fancyhdr}     % En-têtes et pieds de page personnalisés

\pagestyle{fancy}         % modification flexible de l'en-tête et du pied de                             page  
\fancyhf{}                % vide l'en-tête et le pied de page
\fancyhead[L]{\leftmark}  % Section ou chapitre courant
\fancyfoot[C]{\thepage}   % Numéro de page au centre

\usepackage{hyperref}   % Ajouter des liens hypertextes dans le document                                 (liens internes, externes, etc.), mettre les liens                             en surbrillance

%%%%%%%%%%%%%%%%%%%%%%%%%%%%%%%%%%%%%%%%%%%%%%%%%%%%
%%%%%%%%%%%%%%%%%%%%%%%%%%%%%%%%%%%%%%%%%%%%%%%%%%%%
%%%%%%%%%%%%%%%%%%%%%%%%%%%%%%%%%%%%%%%%%%%%%%%%%%%%
                          
\usepackage[english,french]{babel} % Langue du document
\usepackage{graphicx}       % Inclusion d'images
\usepackage{float}
\usepackage{color}
\usepackage{comment}        % pour commenter plusieur ligne



\begin{document}

% En-tête
\begin{titlepage}
    \begin{center}
        % Logos
        \begin{figure}[t]
            \centering
            \includegraphics[width=0.35\textwidth]{logo/logo_LIRMM.jpg} 
        \end{figure}
        \begin{figure}[t]
            \centering
            \includegraphics[width=0.2\textwidth]{logo/logo_univ_mpt.png} 
        \end{figure}

        % Titre
        \vspace{2cm}
        \Huge{\textbf{Rapport de stage : trouver un titre intéréssant}} \\
        \vspace{0.5cm}
        \large{\textit{HAM2FDS4 : Stage académique}} \\
        
        % Espacement
        \vspace{2cm}

        % Auteurs
        \large{
            \textbf{Lucas Noirot} \\
            (\texttt{lucas.noirot@etu.umontpellier.fr})
        }
        
        % Espacement
        \vspace{1.5cm}
        
        % Encadrant
        \large{
            \textbf{Encadrant :} \textbf{Romain Lebreton} \\
            (\texttt{romain.lebreton@lirmm.fr})
        }

        % Espacement
        \vspace{2cm}
        
        % Date
        \normalsize{
            \textbf{Date :}  17 février - 26 juin 2025
        }
    \end{center}
\end{titlepage}


% Table des matières
\tableofcontents


\chapter{Introduction}

\chapter{Work in progress}

\begin{definition}
Soit $\mathbb{K}$ un corps et $M \in \mathbb{K}[x]_{1 \times n}$. On définit le \textbf{degré de ligne} du vecteur ligne M par : 
$$rdeg(M)=max(deg(m_i))_{i\in\{1, \cdots, n\}}$$
\end{definition}

\begin{definition}
Soit $\mathbb{K}$ un corps et $M \in \mathbb{K}[x]_{n \times n}$. On définit le \textbf{degré de ligne} de la matrice M par :
$$rdeg(M)=max(rdeg(ligne_i))_{i\in\{1, \cdots, n\}}$$
\end{definition}

\begin{example}
Soit $M =
\begin{pmatrix}
 & b & c \\
5 & 0 & -3\\
 & h & i
\end{pmatrix}
$

\end{example}

Je cite \cite{clef_unique_0} et \cite{clef_unique_1}  je cite \cite{clef_unique_2} et puis \cite{clef_unique_3} et finalement \cite{clef_unique_4}

\chapter{Conclusion}


\printbibliography

\end{document}
