\section{Orthogonalisation de Gram--Schmidt}

\begin{definition}[Base orthogonale]
    Une base \( (\bb_i)_{1 \leq i \leq n} \) de \( \R^n \) est dite \textbf{orthogonale} si

    \[
        \langle \bb_i, \bb_j\rangle = 0 \quad \text{pour tout } i \neq j.
    \]
\end{definition}

Soit \( B = (\bb_1,\ldots,\bb_n) \) une base de \( \R^n \). On construit une base orthogonale associée \( B^* = (\bb_1^*,\ldots,\bb_n^*) \) de la façon suivante, appelée \textbf{procédé d'orthogonalisation de Gram-Schmidt} :
\[
    \bb_1^* \coloneqq \bb_1,
    \quad
    \bb_i^* \coloneqq \bb_i - \sum_{j=1}^{i-1} \mu_{i,j}\, \bb_j^*,
    \quad
    \mu_{i,j} \coloneqq \frac{\langle \bb_i,\bb_j^*\rangle}{\|\bb_j^*\|^2}.
\]

Les coefficients \( \mu_{i,j} \) sont appelés \textbf{coefficients de Gram-Schmidt}.


\begin{proposition}
    La famille \( (\bb_i^*)_{1 \leq i \leq n} \) obtenue est orthogonale.
\end{proposition}

\begin{proof}
    Pour \( n = 1 \), la famille est clairement orthogonale.

    Supposons que \( (\bb_i^*)_{1 \leq i \leq k} \) est orthogonale pour un certain \( k < n \).

    Alors
    \[
        \ps{\bb_{k+1}^*}{\bb_i^*} = \ps{\bb_{k+1}- \sum_{j=1}^{k} \gscoeff{\bb_{k+1}}{\bb_j^*}}{\bb_i^*}
    \]

    \[
        = \ps{\bb_{k+1}}{\bb_i^*} - \ps{\sum_{j=1}^{k} \gscoeff{\bb_{k+1}}{\bb_j^*}}{\bb_i^*}
    \]

    \[
        = \ps{\bb_{k+1}}{\bb_i^*} - \ps{\gscoeff{\bb_{k+1}}{\bb_i^*}}{\bb_i^*} = 0
    \]

    donc la famille \( (\bb_i^*)_{1 \leq i \leq k+1} \) est orthogonale et par le procédé de récurrence la famille \( (\bb_i^*)_{1 \leq i \leq n} \) est orthogonale.
\end{proof}



\begin{definition}
    Le \textbf{complément orthogonal} de \( U \), noté \( U^\perp \), est défini par
    \[
        \left\{ x \in \R^n \mid \ps{x}{d} = 0 \quad \forall\, d \in U \right\}.
    \]
\end{definition}

\begin{proposition}
    Pour tout \( k \leq n \), ~ \( \bg_k^* \) est la projection de \( \bg_k \) sur \( \displaystyle \left(\sum_{1 \leq i < k} \R \bg_i \right)^\perp\).\footnote{On rappelle que \( \R \bg_i = \{ x \bg_i | x \in \R\}\).}
\end{proposition}

On en déduit directement que \( \mu_{i, i} = 1 \) et \( \mu_{i, j} = 0 \) pour \( i < j \). En notant respectivement \( B \) et \( B^* \) les matrices dont les lignes sont les vecteurs \( (\bb_i)_{1 \leq i \leq n} \) et \( (\bb_i^*)_{1 \leq i \leq n} \), et en posant
\[
    U \;=\; \begin{pmatrix}
        1         & 0      & \cdots      & 0      \\
        \mu_{2,1} & \ddots & \ddots      & \vdots \\
        \vdots    & \ddots & \ddots      & 0      \\
        \mu_{n,1} & \cdots & \mu_{n,n-1} & 1
    \end{pmatrix},
\]
on a la relation suivante :
\begin{equation}
    B = U\,B^*
\end{equation}

\begin{remark}
    \leavevmode\vspace{0.3\baselineskip}
    \begin{itemize}
        \item \( U \) est triangulaire inférieure, en particulier \( \det(U)=1 \).
        \item On ne fait pas de normalisation, afin d'éviter l'introduction de racines irrationnelles et de conserver l'information volumétrique du réseau associé.
    \end{itemize}
\end{remark}

\begin{example}
    Soit
    \[
        B =
        \begin{pmatrix}
            1  & 1 & 1 \\
            -1 & 0 & 2 \\
            3  & 5 & 6
        \end{pmatrix}
        \in GL_3(\Z).
    \]

    \begin{itemize}
        \item[$\bullet$] Calcul de \( \bb_1^* \) :
              \[
                  \bb_1^* = \bb_1 = (1, 1, 1)\text{ , } \quad \| \bb_1^* \|^2 = 3.
              \]

        \item[$\bullet$] Calcul de \( \bb_2^* \) :
              \[
                  \begin{aligned}
                      \bb_2^*   & = \bb_2 - \gscoeff{\bb_2}{\bb_1^*} = (-1, 0, 2 ) - \frac{1}{3} (1, 1, 1) \\
                                & = \left(-\frac{4}{3}, -\frac{1}{3}, \frac{5}{3} \right)                  \\
                      \mu_{2,1} & = \frac{1}{3}, \quad \|\bb_2^*\|^2 = \frac{14}{3}.
                  \end{aligned}
              \]

        \item[$\bullet$] Calcul de \( \bb_3^* \) :
              \[
                  \begin{aligned}
                      \bb_3^*   & = \bb_3 - \gscoeff{\bb_3}{\bb_1^*} - \gscoeff{\bb_3}{\bb_2^*}                                           \\
                                & = (3, 5, 6) - \frac{14}{3}(1,1,1) - \frac{13}{14} \left(-\frac{4}{3}, -\frac{1}{3}, \frac{5}{3} \right) \\
                                & =\left(-\frac{3}{7}, \frac{9}{14}, - \frac{3}{14}\right)                                                \\
                      \mu_{3,1} & = \frac{14}{3}, \quad \mu_{3,2} = \frac{13}{14}.
                  \end{aligned}
              \]
    \end{itemize}

    On a donc finalement
    \[
        \begin{aligned}
            U & =
            \begin{pmatrix}
                1            & 0             & 0 \\
                \frac{1}{3}  & 1             & 0 \\
                \frac{14}{3} & \frac{13}{14} & 1
            \end{pmatrix}, \quad
            B^* =
            \begin{pmatrix}
                1            & 1             & 1              \\
                -\frac{4}{3} & - \frac{1}{3} & \frac{5}{3}    \\
                -\frac{3}{7} & \frac{9}{14}  & - \frac{3}{14}
            \end{pmatrix}.
        \end{aligned}
    \]
\end{example}

\begin{proposition}
    On a \( \displaystyle \det(B)=\det(B^*) = \prod_{i=1}^n \| \bb_i^*\| \).
\end{proposition}

\begin{proof}
    \( \displaystyle \det(B) \eqjust{D.1} \det(UB^*) = \det(U) \det(B^*) = \det(B^*) \eqjust{1.3} \prod_{i=1}^n \| \bb_i^*\|\)
\end{proof}


Une implémentation de l’algorithme d’orthogonalisation de Gram-Schmidt en SageMath est disponible sur le dépôt GitHub du projet.

\begin{smallalgo}{Orthogonalisation de Gram–Schmidt}{alg:GSO}
    \LinesNumbered
    \KwIn{Une famille libre \( B = (\bb_1, \ldots, \bb_n) \in M_n(\Q)\).}
    \KwOut{La famille libre orthogonale \( B^* = (\bb_1^*, \ldots, \bb_n^*) \) et la matrice \( U \) des coefficients de Gram–Schmidt.}

    \For{\( k = 1 \) \KwTo \( n \)}{
        \( \bb_k^* \gets \bb_k \)\;
        \For{\( j = 1 \) \KwTo \( k-1 \)}{
            \( U_{k,j} \gets \dfrac{\langle \bb_k^*, \bb_j^* \rangle}{\|\bb_j^*\|^2} \)\;
            \( \bb_k^* \gets \bb_k^* - U_{k,j}\,\bb_j^* \)\;
        }
    }
\end{smallalgo}

\begin{theoreme}\label{thm:complexite-GSO}
    L'algorithme \hyperref[alg:GSO]{\emph{Orthogonalisation de Gram-Schmidt}} effectue au plus \( \OO(n^3) \) opérations arithmétiques dans \( \Q \), où \( n \) est la taille de la famille.
\end{theoreme}