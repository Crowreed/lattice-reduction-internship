\section{Matrice de Gram}

\begin{definition}
    On note \( M \in M_{n}(\R) \) la matrice dont les colonnes sont les vecteurs \( e_1, \ldots, e_n \).

    La \textbf{matrice de Gram} associée est la matrice symétrique définie par :
    \[
        \Gram(M) \;=\;\bigl(\langle e_i, e_j \rangle\bigr)_{1 \le i,j \le n} \;\in\; M_n(\R),
    \]
\end{definition}

Autrement dit,
\[
    \Gram(M) \;= M^t M
\]

\begin{remark}
    \leavevmode\vspace{0.3\baselineskip}
    \begin{itemize}
        \item Les éléments diagonaux de \( G \) sont les carrés des normes \( \|e_i\|^2 \).
        \item \( G \) est une matrice symétrique réelle.
    \end{itemize}
\end{remark}

\begin{proposition}
    Soit \( M \in M_n(\R)\). $\Gram(M)$ est orthogonale si et seulement si ses colonnes forment une famille orthogonale.
\end{proposition}


\begin{example}
    Soit

    \[
        B^* =
        \begin{pmatrix}
            1            & 1            & 1             \\
            -\frac{4}{3} & -\frac{1}{3} & \frac{5}{3}   \\
            -\frac{3}{7} & \frac{9}{14} & -\frac{3}{14}
        \end{pmatrix}
    \]

    Alors
    \[
        \Gram(B^*) = (B^*)^t \cdot B^* =
        \begin{pmatrix}
            1 & -\frac{4}{3} & -\frac{3}{7}  \\
            1 & -\frac{1}{3} & \frac{9}{14}  \\
            1 & \frac{5}{3}  & -\frac{3}{14}
        \end{pmatrix}
        \cdot
        \begin{pmatrix}
            1            & 1            & 1             \\
            -\frac{4}{3} & -\frac{1}{3} & \frac{5}{3}   \\
            -\frac{3}{7} & \frac{9}{14} & -\frac{3}{14}
        \end{pmatrix}
    \]
    \[
        =
        \begin{pmatrix}
            3 & 0            & 0            \\
            0 & \frac{14}{3} & 0            \\
            0 & 0            & \frac{9}{14}
        \end{pmatrix}
    \]

    On en conclut que la famille est orthogonale, il s’agit en réalité de celle obtenue précédemment par le procédé d’orthogonalisation de Gram-Schmidt.
\end{example}

Rappelons enfin que si \( u \) et \( v \) sont des vecteurs unitaires, alors \( \langle u, v\rangle = \cos(\theta) \), où \( \theta \) est l'angle entre \( u \) et \( v \). Ainsi :

\begin{itemize}
    \item \( \langle u,v\rangle=1 \) signifie que \( u \) et \( v \) sont alignés et de même sens,
    \item \( \langle u,v\rangle=0 \) signifie que \( u \) et \( v \) sont orthogonaux,
    \item \( \langle u,v\rangle=-1 \) signifie que \( u \) et \( v \) sont alignés, mais de sens opposé.
\end{itemize}

Ce point de vue fait de la matrice de Gram un outil privilégié pour évaluer la similarité directionnelle dans un ensemble de vecteurs.